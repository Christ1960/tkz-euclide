% tkz-obj-eu-points-with.tex
% Copyright 2020  Alain Matthes
% This work may be distributed and/or modified under the
% conditions of the LaTeX Project Public License, either version 1.3
% of this license or (at your option) any later version.
% The latest version of this license is in
%   http://www.latex-project.org/lppl.txt
% and version 1.3 or later is part of all distributions of LaTeX
% version 2005/12/01 or later.
%
% This work has the LPPL maintenance status “maintained”.
%
% The Current Maintainer of this work is Alain Matthes.
%  utf8 encoding
\def\fileversion{3.06c}
\def\filedate{2020/04/06}
\typeout{2020/04/06 3.06c tkz-obj-eu-points-with.tex}
\makeatletter
%<--------------------------------------------------------------------------–>
%                          Vectors
%<--------------------------------------------------------------------------–>
% Objet :  outils mathématiques pour la géométrie euclideienne avec pgf/tikz
% utilisable de préférence avec un repère orthonormé et le cm comme unité
% utile pour la compatibilité avec pgf 2
%<--------------------------------------------------------------------------–>
%             Duplicate Length      à revoir pas de pt pas de global
% ||v(CN)||= ||v(AB)|| et v(CN) colineaire à v(CD)
% A-->#1 B-->#2 C-->#3 D-->#4 N-->#5   ?????
%<--------------------------------------------------------------------------–>
%<--------------------------------------------------------------------------–>
%                 Outils pour les vecteurs
%<--------------------------------------------------------------------------–>
%  ce sont des outils élémentaires qui à partir de deux points en définissent
% un troisième
% #1 si c'est une option alors c'est un nombre réel
% #2 et #3 sont deux points
% #4 est le nom du point qui résulte de la transformation
% exemple : \tkzVecKNorm (A,B){C} définit un point C tel que AC = 1 et C est %  %  un point de la droite (AC). #1 peut être négatif

\def\tkzDuplicateSegment(#1,#2)(#3,#4){%
\begingroup
    \tkzCalcLength(#1,#2)\tkzGetLength{tkz@firstlen}%
    \tkzCalcLength(#3,#4)\tkzGetLength{tkz@secondlen}%
	 \edef\tkz@ratio{\fpeval{\tkz@firstlen/\tkz@secondlen}}
    \tkz@VecKCoLinear[\tkz@ratio](#3,#4,#3)%
\endgroup
}
\let\tkzDuplicateLength\tkzDuplicateSegment
%<--------------------------------------------------------------------------–>
%                    Coordonnées d'un vecteur  (couple de points)
% Deux points A et B donc un vecteur on récupère les coordonnées de v(AB)
% en cm
% tkzGetVecCoord en cm ou en pt ???
%<--------------------------------------------------------------------------–>
%result in #3x et #3y #1 et #2 sont les points
% passage en cm avec fp ?
% 28.45274 =1 cm
\def\tkzGetVectxy(#1,#2)#3{%
\begingroup
\pgfpointdiff{\pgfpointanchor{#1}{center}}%
             {\pgfpointanchor{#2}{center}}%
\pgfmathparse{\pgf@sys@tonumber{\pgf@x}/1cm}%
\let\tkzresultx\pgfmathresult
\pgfmathparse{\pgf@sys@tonumber{\pgf@y}/1cm}%
\let\tkzresulty\pgfmathresult
\global\expandafter\edef\csname #3x\endcsname{\tkzresultx}%
\global\expandafter\edef\csname #3y\endcsname{\tkzresulty}%
\endgroup
}
%<--------------------------------------------------------------------------–>
% options #1 two points #2,#3 result in #4
%
%<--------------------------------------------------------------------------–>
%<--------------------------------------------------------------------------–>
%          tkzDefPointWith
%<--------------------------------------------------------------------------–>
\def\tkz@numv{0}
\pgfkeys{/@pointwith/.cd,
      colinear/.code     args     = {at #1}{\def\tkz@numv{0}\def\tkz@frompoint{#1}},
      orthogonal/.code            = {\def\tkz@numv{1}},
      linear/.code                = {\def\tkz@numv{2}},
      orthogonal normed/.code     = {\def\tkz@numv{3}},
      linear normed/.code         = {\def\tkz@numv{4}},
      colinear normed/.code  args = {at #1}{\def\tkz@numv{5}\def\tkz@frompoint{#1}},
      K/.code                     = {\edef\tkz@coeff{\fpeval{#1}}},
      K                           = 1,
      normed/.is if               =  tkz@line@normed,
      normed/.default             =  true,
      normed                      =  false,
      orthogonal
}
\def\tkzDefPointWith{\pgfutil@ifnextchar[{\tkz@DefPointWith}{\tkz@DefPointWith[]}}

\def\tkz@DefPointWith[#1](#2,#3){%
\begingroup
\pgfkeys{/@pointwith/.cd,K=1}
\pgfqkeys{/@pointwith}{#1}
\ifcase\tkz@numv%
 % first case 0
   \tkz@DefVectorColinearat[\tkz@coeff](#2,#3)
  \or% 1
   \tkz@VecKOrth[\tkz@coeff](#2,#3)
  \or% 2
   \tkz@VecK[\tkz@coeff](#2,#3)
  \or% 3
   \tkz@VecKOrthNorm[\tkz@coeff](#2,#3)
  \or% 4
   \tkz@VecKNorm[\tkz@coeff](#2,#3)
  \or% 5
   \tkz@VecKColinearNorm[\tkz@coeff](#2,#3)
  \fi
\endgroup
}
%<--------------------------------------------------------------------------–>
%         tkzDefVectorfrom
%<--------------------------------------------------------------------------–>
% tkz@numv 0
\def\tkz@DefVectorColinearat[#1](#2,#3){%
\iftkz@line@normed
   \tkz@VecKColinearNorm[#1](#2,#3)
\else
\begingroup
  \pgfpointdiff{\pgfpointanchor{#2}{center}}{\pgfpointanchor{#3}{center}}%
  \pgf@xa=\pgf@x\relax%
  \pgf@ya=\pgf@y\relax%
  \pgfinterruptboundingbox
  \path (\tkz@frompoint)--++(\tkz@coeff\pgf@xa,\tkz@coeff\pgf@ya) coordinate (tkzPointResult);
  \endpgfinterruptboundingbox
\endgroup
\fi
}
%<--------------------------------------------------------------------------–>
%     tkzVector K Orth   coeff dans #1
%     v(AN) perp v(AB)  (v(AB) , v(AN) ) sens direct cercle trigo
%     ||v(AN)||=||v(AB)||
%<--------------------------------------------------------------------------–>
% tkz@numv 1
\def\tkzVecKOrth{\pgfutil@ifnextchar[{\tkz@VecKOrth}{\tkz@VecKOrth[1]}}
\def\tkz@VecKOrth[#1](#2,#3){%
\iftkz@line@normed
   \tkz@VecKOrthNorm[#1](#2,#3)
\else
\begingroup
  \pgfpointdiff{\pgfpointanchor{#2}{center}}{\pgfpointanchor{#3}{center}}%
  \pgf@xa=-\pgf@y%
  \pgf@ya=\pgf@x%
  \pgfinterruptboundingbox
  \path (#2)--++(\tkz@coeff\pgf@xa,\tkz@coeff\pgf@ya) coordinate (tkzPointResult);
  \endpgfinterruptboundingbox
\endgroup
\fi
}%

%<--------------------------------------------------------------------------–>
%         v(AN)=#1 x v(AB)
%   #1 le coeff; #2--> A #3--> B   #4-->N  tq    #4-#2 = #1*(#3-#2)
%<--------------------------------------------------------------------------–>
% tkz@numv 2
\def\tkzVecK{\pgfutil@ifnextchar[{\tkz@VecK}{\tkz@VecK[1]}}
\def\tkz@VecK[#1](#2,#3){%
\iftkz@line@normed
   \tkz@VecKNorm[#1](#2,#3)
\else
\begingroup
   \pgfpointdiff{\pgfpointanchor{#2}{center}}%
                {\pgfpointanchor{#3}{center}}%
   \pgf@xa=\pgf@x\relax%
   \pgf@ya=\pgf@y\relax%
   \pgfmathparse{#1}
   \let\tkz@coeff\pgfmathresult
      \pgfinterruptboundingbox
   \path (#2)--++(\tkz@coeff\pgf@xa,\tkz@coeff\pgf@ya) coordinate (tkzPointResult);
    \endpgfinterruptboundingbox
\endgroup
\fi
}%
%<--------------------------------------------------------------------------–>
%              tkzVecKOrthNorm   coeff dans #1
%    v(AN) perp v(AB)  v(AB) v(AN) sens direct cercle trigo
% ||v(AN||=1 si #1 est vide ou =1 sinon ||v(AN||=K
%<--------------------------------------------------------------------------–>
% tkz@numv 3
\def\tkzVecKOrthNorm{\pgfutil@ifnextchar[{\tkz@VecKOrthNorm}%
                                         {\tkz@VecKOrthNorm[1]}}
\def\tkz@VecKOrthNorm[#1](#2,#3){%
\begingroup
 \tkzpointnormalised{\pgfpointdiff{\pgfpointanchor{#2}{center}}%
                                  {\pgfpointanchor{#3}{center}}}
  \pgf@xa=-\pgf@y\relax%
  \pgf@ya=\pgf@x\relax%
  \edef\tkz@x{\fpeval{28.45274*(#1)*\pgf@xa}}
  \edef\tkz@y{\fpeval{28.45274*(#1)*\pgf@ya}}
  \pgfinterruptboundingbox
  \path (#2)--++(\tkz@x pt,\tkz@y pt) coordinate (tkzPointResult);
  \endpgfinterruptboundingbox
\endgroup
}%
%<--------------------------------------------------------------------------–>
%              VectorNormalised ou K*VectorNormalised
% A-->#2 B-->#3 N-->#4      v(AB)  devient v(AN) tq ||v(AN)||=1 si #1=1
%  sinon ||v(AN)||=#1
%<--------------------------------------------------------------------------–>
% tkz@numv 4
\def\tkzVecKNorm{\pgfutil@ifnextchar[{\tkz@VecKNorm}{\tkz@VecKNorm[1]}}
\def\tkz@VecKNorm[#1](#2,#3){%
\begingroup
    \tkzpointnormalised{%
    \pgfpointdiff{\pgfpointanchor{#2}{center}}
                 {\pgfpointanchor{#3}{center}}}
    \pgf@xa=\pgf@x\relax%
    \pgf@ya=\pgf@y\relax%
    \edef\tkz@x{\fpeval{28.45274*(#1)*\pgf@xa}}
    \edef\tkz@y{\fpeval{28.45274*(#1)*\pgf@ya}}
    \pgfinterruptboundingbox
    \path (#2)--++(\tkz@x pt,\tkz@y pt) coordinate (tkzPointResult);
    \endpgfinterruptboundingbox
\endgroup
}%
%<--------------------------------------------------------------------------–>
%            \tkz@VecKColinearNorm
%<--------------------------------------------------------------------------–>
%% tkz@numv 5
\def\tkz@VecKColinearNorm[#1](#2,#3){%
\begingroup
    \tkzpointnormalised{%
    \pgfpointdiff{\pgfpointanchor{#2}{center}}
                 {\pgfpointanchor{#3}{center}}}
    \pgf@xa=\pgf@x\relax%
    \pgf@ya=\pgf@y\relax%
    \edef\tkz@x{\fpeval{28.45274*(#1)*\pgf@xa}}
    \edef\tkz@y{\fpeval{28.45274*(#1)*\pgf@ya}}
    \pgfinterruptboundingbox
    \path (\tkz@frompoint)--++(\tkz@x pt,\tkz@y pt) coordinate (tkzPointResult);
    \endpgfinterruptboundingbox
\endgroup
}%
%<--------------------------------------------------------------------------–>
%  VecKCoLinear  CN = K x AB  #1 pt #2 pt #3 pt #4 nb #5 pt result
% il faut modifier cette macro : on supprime #3 pour la colinéarité
% Il suffit d'utiliser Replicate ou Duplicate     coeff dans #1
% v(CD)=#1 x v(AB) #1 le coeff; #2-->A #3-->B #4-->C
%<--------------------------------------------------------------------------–>
\def\tkz@VecKCoLinear[#1](#2,#3,#4){%
\begingroup
   \pgfpointdiff{\pgfpointanchor{#2}{center}}%
                {\pgfpointanchor{#3}{center}}%
   \pgf@xa=\pgf@x\relax%
   \pgf@ya=\pgf@y\relax%
   \edef\tkz@coeff{\fpeval{#1}}
   \pgfinterruptboundingbox
   \path (#4)--+(\tkz@coeff\pgf@xa,\tkz@coeff\pgf@ya)  coordinate (tkzPointResult);
   \endpgfinterruptboundingbox
\endgroup
}%
\def\tkz@VecCoLinear(#1,#2,#3){%
\begingroup
   \pgfpointdiff{\pgfpointanchor{#1}{center}}%
                {\pgfpointanchor{#2}{center}}%
   \pgf@xa=\pgf@x\relax%
   \pgf@ya=\pgf@y\relax%
   \pgfinterruptboundingbox
   \path (#3)--+(\pgf@xa,\pgf@ya)  coordinate (tkzPointResult);
   \endpgfinterruptboundingbox
\endgroup
}%
%<-------------------------------------------------------------------------–>
\makeatother
\endinput
