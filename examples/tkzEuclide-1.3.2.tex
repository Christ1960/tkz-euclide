\input{preamble-standalone.ltx}
\begin{document}

% Ex. No. 8 (Section  1.3.2 : Example Part II: two others methods gold and euclide triangle)

\begin{tikzpicture}
  \tkzDefPoint(0,0){C} % possible
  % \tkzDefPoint[label=below:$C$](0,0){C}
  % but don't do this
  \tkzDefPoint(2,6){B}
  % We get D and E with a rotation
  \tkzDefPointBy[rotation= center B angle 36](C) \tkzGetPoint{D}
  \tkzDefPointBy[rotation= center B angle 72](C) \tkzGetPoint{E}
  % To get A we use an intersection of lines
  \tkzInterLL(B,E)(C,D) \tkzGetPoint{A}
  \tkzInterLL(C,E)(B,D) \tkzGetPoint{H}
  % drawing
  \tkzDrawArc[delta=10](B,C)(E)
  \tkzDrawPolygon(C,B,D)
  \tkzDrawSegments(D,A B,A C,E)
  % angles
  \tkzMarkAngles(C,B,D E,A,D) %this is to draw the arcs
  \tkzLabelAngles[pos=1.5](C,B,D E,A,D){$\alpha$}
  \tkzMarkRightAngle(B,H,C)
  \tkzDrawPoints(A,...,E)
  % Label only now
  \tkzLabelPoints[below left](C,A)
  \tkzLabelPoints[below right](D)
  \tkzLabelPoints[above](B,E)
\end{tikzpicture}

\end{document}