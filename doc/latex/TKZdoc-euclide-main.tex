%  encoding : utf8 
%  doc de tkz-euclide.sty
%  Created by Alain Matthes  on 2010-04-04.
%  Copyright (C) 2010 Alain Matthes  
%
% This file may be distributed and/or modified
%
% 1. under the LaTeX Project Public License , either version 1.3
% of this license or (at your option) any later version and/or
% 2. under the GNU Public License.
%
% See the file doc/generic/pgf/licenses/LICENSE for more details.%
% See http://www.latex-project.org/lppl.txt for details.
%
%
% TKZdoc-euclide-main is the french  doc of tkz-euclide

\documentclass[DIV         = 12,
               fontsize    = 10,
               headinclude = false,
               index       = totoc,
               footinclude = false,
               twoside,
               headings    = small]{tkz-doc}
%\usepackage{svn-multi}
\usepackage{tkz-euclide}
\usetkzobj{all}
\tkzSetUpColors[background=fondpaille,text=Maroon]
\usepackage[frenchb]{babel}
\usepackage[autolanguage]{numprint}
\usepackage[pdftex,
            unicode,
            colorlinks    = true,
            pdfpagelabels, 
            urlcolor      = blue,
            filecolor     = pdffilecolor,
            linkcolor     = blue,
            breaklinks    = false,
            hyperfootnotes= false,
            bookmarks     = false,
            bookmarksopen = false, 
            linktocpage   = true,
            pdfsubject    ={Euclidean geometry},
            pdfauthor     ={Alain Matthes},
            pdftitle      ={tkz-euclide},
            pdfkeywords   ={euclide,compass,rule,point,line},
            pdfcreator    ={pdfeTeX}
            ]{hyperref}    
\usepackage{url}
\def\UrlFont{\small\ttfamily}
\usepackage[protrusion = true,
            expansion,
            final,
            verbose    = false]{microtype}

\DisableLigatures{encoding = T1, family = tt*} 
\usepackage{tkzexample}  
% \usepackage[saved]{tkzexample}  
% \def\tkzFileSavedPrefix{tkzEucl}%    
\usepackage[parfill]{parskip}
\gdef\nameofpack{tkz-euclide}
\gdef\versionofpack{1.13 c}
\gdef\dateofpack{2011/01/20}
\gdef\nameofdoc{doc-tkz-euclide}
\gdef\dateofdoc{2011/02/18}
\gdef\authorofpack{Alain Matthes}
\gdef\adressofauthor{}
\gdef\namecollection{AlterMundus}
\gdef\urlauthor{http://altermundus.fr}
\gdef\urlauthorcom{http://altermundus.com}
\title{The package : tkz-euclide.sty}
\author{Alain Matthes}

%\usepackage{hvindex}
   
\usepackage{shortvrb,fancyvrb}
\makeatletter
\renewcommand*\l@subsubsection{\bprot@dottedtocline{3}{3.8em}{4em}}
\makeatother
\AtBeginDocument{\MakeShortVerb{\|}}

\pdfcompresslevel=9
\pdfinfo{
    /Title (doc-tkz-euclide.pdf)
    /Creator (TeX)
    /Producer (pdfeTeX)
    /Author (Alain Matthes)
    /CreationDate (18 février 2011)
    /Subject (Documentation du package tkz-euclide v 1.13 c)
    /Keywords (pdfeTeX, geometry, compass, triangle, segment, line, pdflatex) }

%<---------------------------------------------------------------------------> 
\begin{document}

\title{\nameofpack}
\date{\today}
\clearpage
\thispagestyle{empty}
\maketitle

\clearpage
\pagecolor{fondpaille} 
\color{Maroon}    
\colorlet{graphicbackground}{fondpaille}
\colorlet{codebackground}{Peach!30}
\colorlet{codeonlybackground}{Peach!30}    
\colorlet{numbackground}{fondpaille}
\colorlet{textcodecolor}{Maroon}
\colorlet{numcolor}{gray}  

\nameoffile{\nameofpack} 
\defoffile{Le package \textbf{tkz-euclide.sty} est un ensemble de macros spécialisées permettant de construire des objets géométriques en 2D dans un plan muni d'un repère. Il est construit au-dessus de PGF et son interface TikZ. Ce document fournit les définitions des différentes macros ainsi que des exemples dont la complexité est graduée. \textbf{tkz-euclide.sty} remplace \textbf{tkz-2d.sty} dont le code n'est plus maintenu. Ce package nécessite la version 2.1 de \TIKZ.}

\presentation  

\vspace*{1cm}  
\noindent\lefthand\ Je souhaite remercier \textbf{Till Tantau} pour avoir créé le merveilleux outil \href{http://sourceforge.net/projects/pgf/}{Ti\emph{k}Z}, ainsi que \tkzimp{Michel Bovani} pour \tkzname{fourier}, dont l'association avec \tkzname{utopia} est excellente. 

\vspace*{12pt} 
\noindent\lefthand\ Je  remercie \textbf{Yve Combe} pour avoir partagé son travail sur le rapporteur et les constructions à l'aide du compas. Je souhaite remercier également,  \tkzimp{David Arnold} qui a corrigé un grand nombre d'erreurs et qui a testé de nombreux exemples,  \tkzimp{Wolfgang Büchel} qui a corrigé également des erreurs et a construit de superbes scripts pour obtenir les fichiers d'exemples,  \tkzimp{John Kitzmiller} et \tkzimp{Dimitri Kapetas}  pour leurs exemples, et enfin  \tkzimp{Gaétan Marris} pour ses remarques et corrections.

\vspace*{12pt}
\noindent\lefthand\ Vous trouverez de nombreux exemples sur mes sites~: 
\href{http://altermundus.com/pages/download.html}{altermundus.com} ou 
\href{http://altermundus.fr/pages/download.html}{altermundus.fr}    

\vfill   
Vous pouvez envoyer vos remarques, et les rapports sur des erreurs que vous aurez constatées à l'adresse suivante~: \href{mailto:al.ma@mac.com}{\textcolor{blue}{Alain Matthes}}.
 
This file can be redistributed and/or modified under the terms of the LATEX 
Project Public License Distributed from \href{http://www.ctan.org/}{CTAN}\  archives.

\vspace{1cm}
\begin{center}
  \begin{tikzpicture}[decoration=snake,color=Peach,line width=1pt]
    \draw[decorate] (0,0)--(\textwidth-1cm,0);
  \end{tikzpicture}
\end{center}

\clearpage
\tableofcontents

\vspace{1cm}
\begin{center}
  \begin{tikzpicture}[decoration=snake,color=Peach,line width=1pt]
    \draw[decorate] (0,0)--(\textwidth-1cm,0);
  \end{tikzpicture}
\end{center}
\clearpage   \newpage

\setlength{\parskip}{1ex plus 0.5ex minus 0.2ex}
 \section{Installation}

\tkzNamePack{tkz-euclide} and \tkzNamePack{tkz-base} are now on the server of the \tkzname{CTAN}\footnote{\tkzNamePack{tkz-base} and \tkzNamePack{tkz-euclide} are part of \NameDist{TeXLive} and \tkzname{tlmgr} allows you to install them. These packages are also part of \NameDist{MiKTeX} under \NameSys{Windows}.}. If you want to test a beta version, just put the following files in a texmf folder that your system can find.
You will have to check several points:

\begin{itemize}\setlength{\itemsep}{5pt}
\item  The \tkzNamePack{tkz-base} and \tkzNamePack{tkz-euclide} folders must be located on a path recognized by \tkzname{latex}.
\item  The \tkzNamePack{xfp}\footnote{\tkzNamePack{xfp} replaces \tkzNamePack{fp}.}, \tkzNamePack{numprint} and \tkzNamePack{tikz 3.00} must be installed as they are mandatory, for the proper functioning of \tkzNamePack{tkz-euclide}.
\item This documentation and all examples were obtained with \tkzname{lualatex-dev} but \tkzname{pdflatex} should be suitable.
\end{itemize}

\subsection{List of folder files \tkzname{tkzbase}  and \tkzname{tkzeuclide}}

In the folder \tkzname{base}:

\begin{itemize}
\item  \tkzname{tkz-base.cfg}
\item  \tkzname{tkz-base.sty}
\item  \tkzname{tkz-lib-marks.tex}
\item  \tkzname{tkz-obj-axes.tex}
\item  \tkzname{tkz-obj-grids.tex}
\item  \tkzname{tkz-obj-marks.tex}
\item  \tkzname{tkz-obj-points.tex}
\item  \tkzname{tkz-obj-rep.tex}
\item  \tkzname{tkz-tools-arith.tex}
\item  \tkzname{tkz-tools-base.tex}
\item  \tkzname{tkz-tools-BB.tex}
\item  \tkzname{tkz-tools-misc.tex}
\item  \tkzname{tkz-tools-modules.tex}
\item  \tkzname{tkz-tools-print.tex}
\item  \tkzname{tkz-tools-text.tex}
\item  \tkzname{tkz-tools-utilities.tex}
\end{itemize}

In the folder \tkzname{euclide}:

\begin{itemize}
\item   \tkzname{tkz-euclide.sty}
\item   \tkzname{tkz-obj-eu-angles.tex}
\item   \tkzname{tkz-obj-eu-arcs.tex}
\item   \tkzname{tkz-obj-eu-circles.tex}
\item   \tkzname{tkz-obj-eu-compass.tex}
\item   \tkzname{tkz-obj-eu-draw-circles.tex}
\item   \tkzname{tkz-obj-eu-draw-lines.tex}
\item   \tkzname{tkz-obj-eu-draw-polygons.tex}
\item   \tkzname{tkz-obj-eu-draw-triangles.tex}
\item   \tkzname{tkz-obj-eu-lines.tex}
\item   \tkzname{tkz-obj-eu-points-by.tex}
\item   \tkzname{tkz-obj-eu-points-rnd.tex}
\item   \tkzname{tkz-obj-eu-points-with.tex}
\item   \tkzname{tkz-obj-eu-points.tex}
\item   \tkzname{tkz-obj-eu-polygons.tex}
\item   \tkzname{tkz-obj-eu-protractor.tex}
\item   \tkzname{tkz-obj-eu-sectors.tex}
\item   \tkzname{tkz-obj-eu-show.tex}
\item   \tkzname{tkz-obj-eu-triangles.tex}
\item   \tkzname{tkz-tools-angles.tex}
\item   \tkzname{tkz-tools-intersections.tex}
\item   \tkzname{tkz-tools-math.tex}
\end{itemize}
\tkzHandBomb\ Now \tkzname{tkz-euclide} loads all the files. 
\endinput

\section{Presentation and Overview}

\begin{tkzexample}[latex=5cm,small]
\begin{tikzpicture}[scale=.25]
  \tkzDefPoints{00/0/A,12/0/B,6/12*sind(60)/C}
  \foreach \density in {20,30,...,240}{%
  \tkzDrawPolygon[fill=teal!\density](A,B,C)
  \pgfnodealias{X}{A}
  \tkzDefPointWith[linear,K=.15](A,B) \tkzGetPoint{A}
  \tkzDefPointWith[linear,K=.15](B,C) \tkzGetPoint{B}
  \tkzDefPointWith[linear,K=.15](C,X) \tkzGetPoint{C}}
\end{tikzpicture}
\end{tkzexample}

\vspace*{12pt}

\subsection{Why \tkzname{\tkznameofpack}?}

My initial goal was to provide other mathematics teachers and myself with a tool
to quickly create Euclidean geometry figures without investing too much effort
in learning a new programming language.
Of course, \tkzname{\tkznameofpack}  is for math teachers who use \LATEX\ and
makes it possible to easily create correct  drawings by means of \LATEX.

It appeared that the simplest method was to reproduce the one used to obtain
construction by hand.
To describe a construction, you must, of course, define the objects but also the
actions that you perform. It seemed to me that syntax close to the language of
mathematicians and their students would be more easily understandable; moreover,
it also seemed to me that this syntax should be close to that of \LaTeX.
The objects, of course, are points, segments, lines, triangles, polygons and
circles. As for actions, I considered five to be sufficient, namely: define,
create, draw, mark and label.

The syntax is perhaps too verbose but it is, I believe, easily accessible.
As a result, the students like teachers were able to easily access this tool.

\subsection{\tkzname{\tkznameofpack}  vs \tkzname{\TIKZ}}

I love programming with  \TIKZ,  and without  \TIKZ\  I would never have had the
idea to create \tkzname{\tkznameofpack}  but never forget that behind it there
is  \TIKZ\  and that it is always possible to insert code from  \TIKZ.
\tkzname{\tkznameofpack}  doesn't prevent you from using  \TIKZ.
That said, I don't think mixing syntax is a good thing.

There is no need to compare \TIKZ\  and \tkzname{\tkznameofpack}.  The latter is
not addressed to the same audience as  \TIKZ. The first one allows you to do a
lot of things, the second one only does geometry drawings. The first one can do
everything the second one does, but the second one will more easily do what you
want.

\subsection{How it works}

\subsubsection{Example Part I: gold triangle}

\begin{center}
\begin{tikzpicture}
\tkzDefPoint(0,0){C} % possible \tkzDefPoint[label=below:$C$](0,0){C} but don't do this
\tkzDefPoint(2,6){B}
% We get D and E with a rotation
\tkzDefPointBy[rotation= center B angle 36](C) \tkzGetPoint{D}
\tkzDefPointBy[rotation= center B angle 72](C) \tkzGetPoint{E}
% Toget A we use an intersection of lines
\tkzInterLL(B,E)(C,D) \tkzGetPoint{A}
\tkzInterLL(C,E)(B,D) \tkzGetPoint{H}
% drawing
\tkzDrawArc[delta=10](B,C)(E)
\tkzDrawPolygon(C,B,D)
\tkzDrawSegments(D,A B,A C,E)
% angles
\tkzMarkAngles(C,B,D E,A,D) %this is to draw the arcs
\tkzLabelAngles[pos=1.5](C,B,D E,A,D){$\alpha$}
\tkzMarkRightAngle(B,H,C)
\tkzDrawPoints(A,...,E)
% Label only now
\tkzLabelPoints[below left](C,A)
\tkzLabelPoints[below right](D)
\tkzLabelPoints[above](B,E)
\end{tikzpicture}
\end{center}

Let's analyze the figure
\begin{enumerate}
\item $CBD$ and $DBE$ are isosceles triangles;

\item $BC=BE$ and $(BD)$ is a bisector of the angle $CBE$;

\item From this we deduce that the $CBD$ and $DBE$ angles are equal and have
the same measure $\alpha$
   \[\widehat{BAC} +\widehat{ABC} + \widehat{BCA}=180^\circ \ \text{in the
triangle}\ BAC \]
   \[3\alpha + \widehat{BCA}=180^\circ\  \text{in the triangle}\ CBD\]
   then
     \[\alpha + 2\widehat{BCA}=180^\circ \]
   or
     \[\widehat{BCA}=90^\circ -\alpha/2 \]

\item Finally \[\widehat{CBD}=\alpha=36^\circ \]
     the triangle $CBD$ is a \enquote{gold} triangle.
\end{enumerate}

\vspace*{24pt}
How construct a gold triangle or an angle of $36^\circ$?

\begin{enumerate}
\item We place the fixed points $C$ and $D$. |\tkzDefPoint(0,0){C}| and
|\tkzDefPoint(4,0){D}|;

\item  We construct a square $CDef$ and we construct the midpoint $m$ of $[Cf]$;

We can do all of this with a compass and a rule;
\item Then we trace an arc with center $m$ through $e$. This arc cross the
line $(Cf)$ at $n$;

\item Now the two arcs with center $C$ and $D$ and radius $Cn$ define the
point $B$.
\end{enumerate}

\begin{minipage}{0.4\textwidth}
\begin{tikzpicture}
\tkzDefPoint(0,0){C}
\tkzDefPoint(4,0){D}
\tkzDefSquare(C,D)
\tkzGetPoints{e}{f}
\tkzDefMidPoint(C,f)
\tkzGetPoint{m}
\tkzInterLC(C,f)(m,e)
\tkzGetSecondPoint{n}
\tkzInterCC[with nodes](C,C,n)(D,C,n)
\tkzGetFirstPoint{B}
\tkzDrawSegment[brown,dashed](f,n)
\pgfinterruptboundingbox
\tkzDrawPolygon[brown,dashed](C,D,e,f)
\tkzDrawArc[brown,dashed](m,e)(n)
\tkzCompass[brown,dashed,delta=20](C,B)
\tkzCompass[brown,dashed,delta=20](D,B)
\endpgfinterruptboundingbox
\tkzDrawPoints(C,D,B)
\tkzDrawPolygon(B,...,D)
\end{tikzpicture}
\end{minipage}
\begin{minipage}{0.58\textwidth}
\begin{tkzexample}[code only,small]
\begin{tikzpicture}
  \tkzDefPoint(0,0){C}
  \tkzDefPoint(4,0){D}
  \tkzDefSquare(C,D)
  \tkzGetPoints{e}{f}
  \tkzDefMidPoint(C,f)
  \tkzGetPoint{m}
  \tkzInterLC(C,f)(m,e)
  \tkzGetSecondPoint{n}
  \tkzInterCC[with nodes](C,C,n)(D,C,n)
  \tkzGetFirstPoint{B}
  \tkzDrawSegment[brown,dashed](f,n)
  \pgfinterruptboundingbox
  \tkzDrawPolygon[brown,dashed](C,D,e,f)
  \tkzDrawArc[brown,dashed](m,e)(n)
  \tkzCompass[brown,dashed,delta=20](C,B)
  \tkzCompass[brown,dashed,delta=20](D,B)
  \endpgfinterruptboundingbox
  \tkzDrawPoints(C,D,B)
  \tkzDrawPolygon(B,...,D)
\end{tikzpicture}
\end{tkzexample}
\end{minipage}

After building the golden triangle $BCD$, we build the point $A$ by noticing
that $BD=DA$. Then we get the point $E$ and finally the point $F$. This is done
with already intersections of defined objects  (line and circle).

\begin{center}
\begin{tikzpicture}
\tkzDefPoint(0,0){C}
\tkzDefPoint(4,0){D}
\tkzDefSquare(C,D)
\tkzGetPoints{e}{f}
\tkzDefMidPoint(C,f)
\tkzGetPoint{m}
\tkzInterLC(C,f)(m,e)
\tkzGetSecondPoint{n}
\tkzInterCC[with nodes](C,C,n)(D,C,n)
\tkzGetFirstPoint{B}
\tkzInterLC(C,D)(D,B) \tkzGetSecondPoint{A}
\tkzInterLC(B,A)(B,D) \tkzGetSecondPoint{E}
\tkzInterLL(B,D)(C,E) \tkzGetPoint{F}
\tkzDrawPoints(C,D,B)
\tkzDrawPolygon(B,...,D)
\tkzDrawPolygon(B,C,D)
\tkzDrawSegments(D,A A,B C,E)
\tkzDrawArc[delta=10](B,C)(E)
\tkzMarkRightAngle[fill=blue!20](B,F,C)
\tkzFillAngles[fill=blue!10](C,B,D E,A,D)
\tkzMarkAngles(C,B,D E,A,D)
\tkzLabelAngles[pos=1.5](C,B,D E,A,D){$\alpha$}
\tkzLabelPoints[below](A,C,D,E)
\tkzLabelPoints[above right](B,F)
\tkzDrawPoints(A,...,F)
\end{tikzpicture}
\end{center}

\begin{tkzexample}[code only,small,pre={},post={}]
\begin{tikzpicture}
  \tkzDefPoint(0,0){C}
  \tkzDefPoint(4,0){D}
  \tkzDefSquare(C,D)
  \tkzGetPoints{e}{f}
  \tkzDefMidPoint(C,f)
  \tkzGetPoint{m}
  \tkzInterLC(C,f)(m,e)
  \tkzGetSecondPoint{n}
  \tkzInterCC[with nodes](C,C,n)(D,C,n)
  \tkzGetFirstPoint{B}
  \tkzInterLC(C,D)(D,B) \tkzGetSecondPoint{A}
  \tkzInterLC(B,A)(B,D) \tkzGetSecondPoint{E}
  \tkzInterLL(B,D)(C,E) \tkzGetPoint{F}
  \tkzDrawPoints(C,D,B)
  \tkzDrawPolygon(B,...,D)
  \tkzDrawPolygon(B,C,D)
  \tkzDrawSegments(D,A A,B C,E)
  \tkzDrawArc[delta=10](B,C)(E)
  \tkzDrawPoints(A,...,F)
  \tkzMarkRightAngle[fill=blue!20](B,F,C)
  \tkzFillAngles[fill=blue!10](C,B,D E,A,D)
  \tkzMarkAngles(C,B,D E,A,D)
  \tkzLabelAngles[pos=1.5](C,B,D E,A,D){$\alpha$}
  \tkzLabelPoints[below](A,C,D,E)
  \tkzLabelPoints[above right](B,F)
\end{tikzpicture}
\end{tkzexample}

\subsubsection{Example Part II: two others methods gold and euclide triangle}

\tkzname{\tkznameofpack} knows how to define a \enquote{gold} or \enquote{euclide} triangle. We
can define $BCD$ and $BCA$ like gold triangles.

\begin{tkzexample}[code only,small,pre={},post={}]
\begin{tikzpicture}
  \tkzDefPoint(0,0){C}
  \tkzDefPoint(4,0){D}
  \tkzDefTriangle[euclide](C,D)
  \tkzGetPoint{B}
  \tkzDefTriangle[euclide](B,C)
  \tkzGetPoint{A}
  \tkzInterLC(B,A)(B,D) \tkzGetSecondPoint{E}
  \tkzInterLL(B,D)(C,E) \tkzGetPoint{F}
  \tkzDrawPoints(C,D,B)
  \tkzDrawPolygon(B,...,D)
  \tkzDrawPolygon(B,C,D)
  \tkzDrawSegments(D,A A,B C,E)
  \tkzDrawArc[delta=10](B,C)(E)
  \tkzDrawPoints(A,...,F)
  \tkzMarkRightAngle[fill=blue!20](B,F,C)
  \tkzFillAngles[fill=blue!10](C,B,D E,A,D)
  \tkzMarkAngles(C,B,D E,A,D)
  \tkzLabelAngles[pos=1.5](C,B,D E,A,D){$\alpha$}
  \tkzLabelPoints[below](A,C,D,E)
  \tkzLabelPoints[above right](B,F)
\end{tikzpicture}
\end{tkzexample}

Here is a final method that uses rotations:

\begin{tkzexample}[code only,small,pre={},post={}]
\begin{tikzpicture}
  \tkzDefPoint(0,0){C} % possible
  % \tkzDefPoint[label=below:$C$](0,0){C}
  % but don't do this
  \tkzDefPoint(2,6){B}
  % We get D and E with a rotation
  \tkzDefPointBy[rotation= center B angle 36](C) \tkzGetPoint{D}
  \tkzDefPointBy[rotation= center B angle 72](C) \tkzGetPoint{E}
  % To get A we use an intersection of lines
  \tkzInterLL(B,E)(C,D) \tkzGetPoint{A}
  \tkzInterLL(C,E)(B,D) \tkzGetPoint{H}
  % drawing
  \tkzDrawArc[delta=10](B,C)(E)
  \tkzDrawPolygon(C,B,D)
  \tkzDrawSegments(D,A B,A C,E)
  % angles
  \tkzMarkAngles(C,B,D E,A,D) %this is to draw the arcs
  \tkzLabelAngles[pos=1.5](C,B,D E,A,D){$\alpha$}
  \tkzMarkRightAngle(B,H,C)
  \tkzDrawPoints(A,...,E)
  % Label only now
  \tkzLabelPoints[below left](C,A)
  \tkzLabelPoints[below right](D)
  \tkzLabelPoints[above](B,E)
\end{tikzpicture}
\end{tkzexample}

\subsubsection{Complete but minimal example}

A unit of length being chosen, the example shows how to obtain a segment of
length $\sqrt{a}$ from a segment of length $a$, using a ruler and a compass.

$IB=a$, $AI=1$

\vspace{12pt}
\hypertarget{firstex}{}

\begin{tikzpicture}[scale=1,ra/.style={fill=gray!20}]
  % fixed points
  \tkzDefPoint(0,0){A}
  \tkzDefPoint(1,0){I}
  % calculation
  \tkzDefPointBy[homothety=center A ratio  10 ](I) \tkzGetPoint{B}
  \tkzDefMidPoint(A,B)              \tkzGetPoint{M}
  \tkzDefPointWith[orthogonal](I,M) \tkzGetPoint{H}
  \tkzInterLC(I,H)(M,B)             \tkzGetSecondPoint{C}
  \tkzDrawSegment[style=orange](I,C)
  \tkzDrawArc(M,B)(A)
  \tkzDrawSegment[dim={$1$,-16pt,}](A,I)
  \tkzDrawSegment[dim={$a/2$,-10pt,}](I,M)
  \tkzDrawSegment[dim={$a/2$,-16pt,}](M,B)
  \tkzMarkRightAngle[ra](A,I,C)
  \tkzDrawPoints(I,A,B,C,M)
  \tkzLabelPoint[left](A){$A(0,0)$}
  \tkzLabelPoints[above right](I,M)
  \tkzLabelPoints[above left](C)
  \tkzLabelPoint[right](B){$B(10,0)$}
  \tkzLabelSegment[right=4pt](I,C){$\sqrt{a^2}=a \ (a>0)$}
\end{tikzpicture}

\emph{Comments}

\begin{itemize}
\item The Preamble

Let us first look at the preamble. If you need it, you have to load
\tkzname{xcolor} before \tkzname{tkz-euclide}, that is, before \TIKZ. \TIKZ\ may
cause problems with the active characters, but\dots
provides a library in its latest version that's supposed to solve these
problems \NameLib{babel}.

\begin{tkzltxexample}[small]
\documentclass{standalone} % or another class
% \usepackage{xcolor} % before tikz or tkz-euclide if necessary
\usepackage{tkz-euclide} % no need to load TikZ
% \usetkzobj{all}  is no longer necessary
% \usetikzlibrary{babel}  if there are problems with the active characters
\end{tkzltxexample}

The following code consists of several parts:

\item Definition of fixed points: the first part includes the definitions of
the points necessary for the construction, these are the fixed points. The
macros \tkzcname{tkzInit} and \tkzcname{tkzClip} in most cases are not
necessary.

\begin{tkzltxexample}[]
  \tkzDefPoint(0,0){A}
  \tkzDefPoint(1,0){I}
\end{tkzltxexample}

\item The second part is dedicated to the creation of new points from the
fixed points;
a $B$ point is placed at $10$~cm from $A$. The middle of $[AB]$ is defined
by $M$ and then the orthogonal line to the $(AB)$ line is searched for at the
$I$ point. Then we look for the intersection of this line with the semi-circle
of center $M$ passing through $A$.

\begin{tkzltxexample}[small]
  \tkzDefPointBy[homothety=center A ratio 10](I)
    \tkzGetPoint{B}
  \tkzDefMidPoint(A,B)
    \tkzGetPoint{M}
  \tkzDefPointWith[orthogonal](I,M)
    \tkzGetPoint{H}
  \tkzInterLC(I,H)(M,A)
    \tkzGetSecondPoint{B}
\end{tkzltxexample}

\item The third one includes the different drawings;
\begin{tkzltxexample}[small]
  \tkzDrawSegment[style=orange](I,H)
  \tkzDrawPoints(O,I,A,B,M)
  \tkzDrawArc(M,A)(O)
  \tkzDrawSegment[dim={$1$,-16pt,}](O,I)
  \tkzDrawSegment[dim={$a/2$,-10pt,}](I,M)
  \tkzDrawSegment[dim={$a/2$,-16pt,}](M,A)
\end{tkzltxexample}

\item  Marking: the fourth is devoted to marking;

\begin{tkzltxexample}[small]
  \tkzMarkRightAngle(A,I,B)
\end{tkzltxexample}

\item Labelling: the latter only deals with the placement of labels.
\begin{tkzltxexample}[small]
  \tkzLabelPoint[left](O){$A(0,0)$}
  \tkzLabelPoint[right](A){$B(10,0)$}
  \tkzLabelSegment[right=4pt](I,B){$\sqrt{a^2}=a \ (a>0)$}
\end{tkzltxexample}

\item The full code:

\begin{tkzexample}[code only,small,pre={},post={},small]
\begin{tikzpicture}[scale=1,ra/.style={fill=gray!20}]
  % fixed points
  \tkzDefPoint(0,0){A}
  \tkzDefPoint(1,0){I}
  % calculation
  \tkzDefPointBy[homothety=center A ratio 10](I) \tkzGetPoint{B}
  \tkzDefMidPoint(A,B)        \tkzGetPoint{M}
  \tkzDefPointWith[orthogonal](I,M) \tkzGetPoint{H}
  \tkzInterLC(I,H)(M,B)       \tkzGetSecondPoint{C}
  \tkzDrawSegment[style=orange](I,C)
  \tkzDrawArc(M,B)(A)
  \tkzDrawSegment[dim={$1$,-16pt,}](A,I)
  \tkzDrawSegment[dim={$a/2$,-10pt,}](I,M)
  \tkzDrawSegment[dim={$a/2$,-16pt,}](M,B)
  \tkzMarkRightAngle[ra](A,I,C)
  \tkzDrawPoints(I,A,B,C,M)
  \tkzLabelPoint[left](A){$A(0,0)$}
  \tkzLabelPoints[above right](I,M)
  \tkzLabelPoints[above left](C)
  \tkzLabelPoint[right](B){$B(10,0)$}
  \tkzLabelSegment[right=4pt](I,C){$\sqrt{a^2}=a \ (a>0)$}
  \end{tikzpicture}
\end{tkzexample}
\end{itemize}

\newpage

\subsection{The Elements of tkz code}

In this paragraph, we start looking at the \enquote{rules} and \enquote{symbols} used to create
a figure with \tkzname{\tkznameofpack}.

The primitive objects are points. You can refer to a point at any time using
the name given when defining it. (it is possible to assign a different name
later on).

\medskip
In general, \tkzname{\tkznameofpack} macros have a name beginning with tkz.
There are four main categories starting with:
|\tkzDef|\dots|\tkzDraw|\dots|\tkzMark|\dots{} and |\tkzLabel|\dots

Among the first category, |\tkzDefPoint| allows you to define fixed points. It
will be studied in detail later. Here we will see in detail the macro
|\tkzDefTriangle|.

This macro makes it possible to associate to a pair of points a third point in
order to define a certain triangle |\tkzDefTriangle(A,B)|. The obtained point is
referenced |tkzPointResult| and it is possible to choose another reference with
|\tkzGetPoint{C}| for example.
Parentheses are used to pass arguments. In |(A,B)| $A$ and $B$ are the points
with which a third will be defined.

However, in |{C}| we use braces to retrieve the new point.
In order to choose a certain type of triangle among the following choices:
|equilateral|, |half|, |pythagoras|, |school|, |golden| or |sublime|, |euclide|,
|gold|, |cheops|\dots{} and |two angles| you just have to choose between hooks,
for example can be used: |\tkzDefTriangle[euclide](A,B)| |\tkzGetPoint{C}|

\begin{minipage}{0.48\textwidth}
\begin{tikzpicture}[scale=0.75]
  \tkzDefPoints{0/0/A,8/0/B}
  \foreach \tr in {equilateral,half,pythagore,%
          school,golden,euclide, gold,cheops}
  {\tkzDefTriangle[\tr](A,B) \tkzGetPoint{C}
  \tkzDrawPoint(C)
  \tkzLabelPoint[right](C){\tr}
  \tkzDrawSegments(A,C C,B)}
  \tkzDrawPoints(A,B)
  \tkzDrawSegments(A,B)
\end{tikzpicture}
\end{minipage}
\begin{minipage}{0.51\textwidth}
\begin{tkzexample}[code only,small]
\begin{tikzpicture}[scale=0.75]
  \tkzDefPoints{0/0/A,8/0/B}
  \foreach \tr in {equilateral,half,pythagore,%
           school,golden,euclide, gold,cheops}
    {\tkzDefTriangle[\tr](A,B) \tkzGetPoint{C}
  \tkzDrawPoint(C)
  \tkzLabelPoint[right](C){\tr}
  \tkzDrawSegments(A,C C,B)}
  \tkzDrawPoints(A,B)
  \tkzDrawSegments(A,B)
\end{tikzpicture}
\end{tkzexample}
\end{minipage}

\subsection{Notations and conventions}

I deliberately chose to use the geometric French and personal  conventions  to
describe the geometric objects represented. The objects defined and represented
by \tkzname{\tkznameofpack} are points, lines and circles located in a plane.
They are the primary objects of Euclidean geometry from which we will construct
figures.

According to \tkzimp{Euclidian} these figures will only illustrate pure ideas
produced by our brain.
Thus a point has no dimension and therefore no real existence. In the same way
the line has no width and therefore no existence in the real world. The objects
that we are going to consider are only representations of ideal mathematical
objects. \tkzname{\tkznameofpack} will follow the steps of the ancient Greeks to
obtain geometrical constructions using the ruler and the compass.

Here are the notations that will be used:

\begin{itemize}
\item The points are represented geometrically either by a small disc or by the
intersection of two lines (two straight lines, a straight line and a circle or
two circles). In this case, the point is represented by a cross.

\newpage

\begin{tkzexample}[latex=6cm, small]
\begin{tikzpicture}
  \tkzDefPoints{0/0/A,4/2/B}
  \tkzDrawPoints(A,B)
  \tkzLabelPoints(A,B)
\end{tikzpicture}
\end{tkzexample}

or else

\begin{tkzexample}[latex=6cm, small]
\begin{tikzpicture}
  \tkzSetUpPoint[shape=cross, color=red]
  \tkzDefPoints{0/0/A,4/2/B}
  \tkzDrawPoints(A,B)
  \tkzLabelPoints(A,B)
\end{tikzpicture}
\end{tkzexample}

The existence of a point being established, we can give it a label which will be
a capital letter (with some exceptions) of the Latin alphabet such as $A$, $B$
or $C$. For example:

\begin{itemize}
\item $O$ is a center for a circle, a rotation, etc.;
\item $M$ defined a midpoint;
\item $H$ defined the foot of an altitude;
\item $P'$ is the image of $P$ by a transformation ;
\end{itemize}

It is important to note that the reference name of a point in the code may be
different from the label to designate it in the text. So we can define a point A
and give it as label $P$. In particular the style will be different, point A
will be labeled $A$.

\begin{tkzexample}[latex=6cm, small]
\begin{tikzpicture}
  \tkzDefPoints{0/0/A}
  \tkzDrawPoints(A)
  \tkzLabelPoint(A){$P$}
\end{tikzpicture}
\end{tkzexample}

Exceptions: some points such as the middle of the sides of a triangle share a
characteristic, so it is normal that their names also share a common character.
We will designate these points by $M_a$, $M_b$ and $M_c$ or $M_A$, $M_B$ and
$M_C$.

In the code, these points will be referred to as: M\_A, M\_B and M\_C.

Another exception relates to intermediate construction points which will not be
labelled. They will often be designated by a lowercase letter in the code.

\item The line segments are designated by two points representing their ends in
square brackets: $[AB]$.

\item The straight lines are in Euclidean geometry defined by two points so $A$
and $B$ define the straight line $(AB)$. We can also designate this stright line
using the Greek alphabet and name it $(\delta)$ or $(\Delta)$. It is also
possible to designate the straight line with lowercase letters such as $d$ and
$d'$.

\item The semi-straight line is designated as follows $[AB)$.

\item Relation between the straight lines. Two perpendicular $(AB)$ and $(CD)$
lines will be written $(AB) \perp (CD)$ and if they are parallel we will write
$(AB) \parallelslant (CD)$.

\item The lengths of the sides of triangle ABC are $AB$, $AC$ and $BC$. The
numbers are also designated by a lowercase letter so we will write: $AB=c$,
$AC=b$ and $BC=a$. The letter $a$ is also used to represent an angle, and $r$ is
frequently used to represent a radius, $d$ a diameter, $l$ a length, $d$ a
distance.

\item Polygons are designated afterwards by their vertices so $ABC$ is a
triangle, $EFGH$ a quadrilateral.

\item Angles are generally measured in degrees (ex $60^\circ$) and in an
equilateral $ABC$ triangle we will write $\widehat{ABC}=\widehat{B}=60^\circ$.

\item The arcs are designated by their extremities. For example if $A$ and $B$
are two points of the same circle then $\widearc{AB}$.

\item Circles are noted either $\mathcal{C}$ if there is no possible confusion
or $\mathcal{C}$ $(O~;~A)$ for a circle with center $O$ and passing through the
point $A$ or $\mathcal{C}$ $(O~;~1)$ for a circle with center O and radius 1 cm.

\item  Name of the particular lines of a triangle: I used the terms bisector,
bisector out, mediator (sometimes called perpendicular bisectors), altitude,
median and symmedian.

\item ($x_1$,$y_1$) coordinates of the point $A_1$, ($x_A$,$y_A$) coordinates of
the point $A$.

\end{itemize}


\subsection{How to use the \tkzname{\tkznameofpack} package?}

\subsubsection{Let's look at a classic example}

In order to show the right way, we will see how to build an equilateral
triangle. Several possibilities are open to us, we are going to follow the steps
of Euclid.

\begin{itemize}
\item First of all you have to use a document class. The best choice to test
your code is to create a single figure with the class \tkzname{standalone}\index{standalone}.

\begin{verbatim}
\documentclass{standalone}
\end{verbatim}

\item Then load the \tkzname{\tkznameofpack} package:

\begin{verbatim}
\usepackage{tkz-euclide}
\end{verbatim}

You don't need to load \TIKZ\ because the \tkzname{\tkznameofpack} package
works on top of TikZ and loads it.
\item {\color{red} \bomb \sout{|\BS usetkzobj{all}|}}
With the new version 3.03 you don't need this line anymore. All objects are now
loaded.
\item Start the document and open a TikZ picture environment:

\begin{verbatim}
\begin{document}
\begin{tikzpicture}
\end{verbatim}

\item Now we define two fixed points:

\begin{verbatim}
\tkzDefPoint(O,O){A}
\tkzDefPoint(5,2){B}
\end{verbatim}

\item Two points define two circles, let's use these circles:

circle with center $A$ through $B$ and circle with center $B$ through $A$.
These two circles have two points in common.

\begin{verbatim}
\tkzInterCC(A,B)(B,A)
\end{verbatim}

We can get the points of intersection with

\begin{verbatim}
\tkzGetPoints{C}{D}
\end{verbatim}

\item All the necessary points are obtained, we can move on to the final steps
including the plots.

\begin{verbatim}
\tkzDrawCircles[gray,dashed](A,B B,A)
\tkzDrawPolygon(A,B,C)% The triangle
\end{verbatim}

\item Draw all points $A$, $B$, $C$ and $D$:

\begin{verbatim}
\tkzDrawPoints(A,...,D)
\end{verbatim}

\item The final step, we print labels to the points and use options for
positioning:\par

\begin{verbatim}
\tkzLabelSegments[swap](A,B){$c$}
\tkzLabelPoints(A,B,D)
\tkzLabelPoints[above](C)
\end{verbatim}

\item We finally close both environments

\begin{verbatim}
\end{tikzpicture}
\end{document}
\end{verbatim}

\item The complete code

\begin{tkzexample}[latex=8cm,small]
\begin{tikzpicture}[scale=0.5]
  % fixed points
  \tkzDefPoint(0,0){A}
  \tkzDefPoint(5,2){B}
  % calculus
  \tkzInterCC(A,B)(B,A)
  \tkzGetPoints{C}{D}
  % drawings
  \tkzDrawCircles[gray,dashed](A,B B,A)
  \tkzDrawPolygon(A,B,C)
  \tkzDrawPoints(A,...,D)
  % marking
  \tkzMarkSegments[mark=s||](A,B B,C C,A)
  % labelling
  \tkzLabelSegments[swap](A,B){$c$}
  \tkzLabelPoints(A,B,D)
  \tkzLabelPoints[above](C)
\end{tikzpicture}
\end{tkzexample}

\end{itemize}

\subsubsection{\tkzname{Set, Calculate, Draw, Mark, Label}}

The title could have been: \texttt{Separation of Calculus and Drawings}

When a document is prepared using the \LATEX\ system, the source code of the
document can be divided into two parts: the document body and the preamble.
Under this methodology,  publications can be structured, styled and typeset with
minimal effort.
I propose a similar methodology for creating figures with \tkzname{\tkznameofpack}.

The first part defines the fixed points, the second part allows the creation of
new points. These are the two main parts. All that is left to do is to draw,
mark and label.

\endinput

\section{Summary of tkz-base}

\subsection{Utility of \tkzname{tkz-base}} 

First of all, you don't have to deal with \TIKZ\ the size of the bounding box. Early versions of \tkzNamePack{tkz-euclide} did not control the size of the bounding box, now the size of the bounding box is limited.

 However, it is sometimes necessary to control the size of what will be displayed.
 To do this, you need to have prepared the bounding box you are going to work in, this is the role of \tkzNamePack{tkz-base} and its main macro \tkzNameMacro{tkzInit}. It is recommended to leave the graphic unit equal to 1 cm. For some drawings, it is interesting to fix the extreme values (xmin,xmax,ymin and ymax) and to "clip" the definition rectangle in order to control the size of the figure as well as possible.

The two macros in \tkzNamePack{tkz-base} that are useful for \tkzNamePack{tkz-euclide} are:
\begin{itemize}
   \item \tkzcname{tkzInit}
   \item \tkzcname{tkzClip}
\end{itemize}
\vspace{20pt}

To this, I added macros directly linked to the bounding box. You can now view it, backup it, restore it (see the documentation of \tkzNamePack{tkz-base} section Bounding Box).

\subsection{\tkzcname{tkzInit} and \tkzcname{tkzShowBB}}
The rectangle around the figure shows you the bounding box.
\begin{tkzexample}[latex=8cm,small]
\begin{tikzpicture}
 \tkzInit[xmin=-1,xmax=3,ymin=-1, ymax=3]
 \tkzGrid
 \tkzShowBB[red,line width=2pt]
\end{tikzpicture}
\end{tkzexample} 

\subsection{\tkzcname{tkzClip}}
The role of this macro is to "clip" the initial rectangle so that only the paths contained in this rectangle are drawn.

\begin{tkzexample}[latex=8cm,small]
\begin{tikzpicture}
 \tkzInit[xmax=4, ymax=3]
 \tkzAxeXY 
 \tkzGrid
 \tkzClip
 \draw[red] (-1,-1)--(5,2);
\end{tikzpicture}
\end{tkzexample} 

It is possible to add a bit of space
\begin{tkzltxexample}[]
  \tkzClip[space=1]
\end{tkzltxexample} 

\subsection{\tkzcname{tkzClip} and the option \tkzname{space}} 
This option allows you to add some space around the "clipped" rectangle.
\begin{tkzexample}[latex=8cm,small]
\begin{tikzpicture}
 \tkzInit[xmax=4, ymax=3]
 \tkzAxeXY 
 \tkzGrid
 \tkzClip[space=1]
 \draw[red] (-1,-1)--(5,2);
\end{tikzpicture}
\end{tkzexample}   
The dimensions of the "clipped" rectangle are \tkzname{xmin-1}, \tkzname{ymin-1}, \tkzname{xmax+1} and \tkzname{ymax+1}. 


\endinput
\section{Definition of a point}

 Points can be specified in any of the following ways:
\begin{itemize}
\item Cartesian coordinates;
\item Polar coordinates;
\item Named points;
\item Relative points.
\end{itemize}

Even if it's possible, I think it's a bad idea to work directly with coordinates. Preferable is to use named points.
A point is defined if it has a name linked to a unique pair of decimal numbers.
 Let $(x,y)$ or $(a:d)$  i.e. ($x$ abscissa, $y$ ordinate) or  ($a$ angle: $d$ distance).
 This is possible because the plan has been provided with an orthonormed Cartesian coordinate system.   The working axes are supposed to be (ortho)normed with unity equal to $1$~cm or something equivalent like $0.39370$~in.
 Now by default if you use a grid or axes, the rectangle used is defined by the coordinate points: $(0,0)$ and $(10,10)$. It's the macro \tkzcname{tkzInit} of the package \tkzNamePack{tkz-base} that creates this rectangle. Look at the following two codes and the result of their compilation:

\begin{tkzexample}[latex=10cm,small]
\begin{tikzpicture}
\tkzGrid
\tkzDefPoint(0,0){O}
\tkzDrawPoint[red](O)
\tkzShowBB[line width=2pt,teal]
\end{tikzpicture}
\end{tkzexample}


\begin{tkzexample}[latex=7cm,small]
\begin{tikzpicture}
 \tkzDefPoint(0,0){O}
 \tkzDefPoint(5,5){A}
 \tkzDrawSegment[blue](O,A)
 \tkzDrawPoints[red](O,A)
 \tkzShowBB[line width=2pt,teal]
\end{tikzpicture}
\end{tkzexample}

 The Cartesian coordinate $(a,b)$ refers to the
 point $a$ centimeters in the $x$-direction and $b$ centimeters in the
 $y$-direction.

 A point in polar coordinates requires an angle $\alpha$, in degrees,
 and a distance  $d$ from the origin with a dimensional
 unit by default it's the \texttt{cm}.


\begin{minipage}[b]{0.5\textwidth}
 Cartesian coordinates
\begin{tkzexample}[vbox,small]
\begin{tikzpicture}[scale=1]
  \tkzInit[xmax=5,ymax=5]
  \tkzDefPoints{0/0/O,1/0/I,0/1/J}
  \tkzDrawXY[noticks,>=latex]
  \tkzDefPoint(3,4){A}
  \tkzDrawPoints(O,A)
  \tkzLabelPoint(A){$A_1 (x_1,y_1)$}
  \tkzShowPointCoord[xlabel=$x_1$,
                     ylabel=$y_1$](A)
  \tkzLabelPoints(O,I)
  \tkzLabelPoints[left](J)
  \tkzDrawPoints[shape=cross](I,J)
\end{tikzpicture}
\end{tkzexample}
\end{minipage}
\begin{minipage}[b]{0.5\textwidth}
 Polar coordinates
\begin{tkzexample}[vbox,small]
\begin{tikzpicture}[,scale=1]
  \tkzInit[xmax=5,ymax=5]
  \tkzDefPoints{0/0/O,1/0/I,0/1/J}
  \tkzDefPoint(40:4){P}
  \tkzDrawXY[noticks,>=triangle 45]
  \tkzDrawSegment[dim={$d$,
                 16pt,above=6pt}](O,P)
  \tkzDrawPoints(O,P)
  \tkzMarkAngle[mark=none,->](I,O,P)
  \tkzFillAngle[fill=blue!20,
                opacity=.5](I,O,P)
  \tkzLabelAngle[pos=1.25](I,O,P){$\alpha$}
  \tkzLabelPoint(P){$P  (\alpha : d )$}
  \tkzDrawPoints[shape=cross](I,J)
  \tkzLabelPoints(O,I)
  \tkzLabelPoints[left](J)
\end{tikzpicture}
\end{tkzexample}
\end{minipage}%

The \tkzNameMacro{tkzDefPoint} macro is used to define a point by assigning coordinates to it. This macro is based on \tkzNameMacro{coordinate}, a macro of \TIKZ. It can use \TIKZ-specific options such as \tkzname{shift}. If calculations are required then the \tkzNamePack{xfp} package is chosen. We can use Cartesian or polar coordinates.

\subsection{Defining a named point  \tkzcname{tkzDefPoint}}

\begin{NewMacroBox}{tkzDefPoint}{\oarg{local options}\parg{$x,y$}\marg{name} or \parg{$\alpha$:$d$}\marg{name}}%
\begin{tabular}{lll}%
arguments &  default & definition  \\
\midrule
\TAline{($x,y$)}{no default}{$x$ and $y$ are two dimensions, by default in cm.}
\TAline{($\alpha$:$d$)}{no default}{$\alpha$ is an angle in degrees, $d$ is a dimension}
\TAline{\{name\}}{no default}{Name assigned to the point: $A$, $T_a$ ,$P1$ etc ...}
\bottomrule
\end{tabular}

\medskip
The obligatory arguments of this macro are two dimensions expressed with decimals, in the first case they are two measures of length, in the second case they are a measure of length and the measure of an angle in degrees.

\medskip
\begin{tabular}{lll}%
\toprule
options             & default & definition  \\
\midrule
\TOline{label} {no default} {allows you to place a label at a predefined distance}
\TOline{shift} {no default} {adds $(x,y)$ or $(\alpha:d)$ to all coordinates}
\end{tabular}
\end{NewMacroBox}

\subsubsection{Cartesian coordinates }

\begin{tkzexample}[latex=7cm,small]
  \begin{tikzpicture}
  \tkzInit[xmax=5,ymax=5]
  \tkzDefPoint(0,0){A}
  \tkzDefPoint(4,0){B}
  \tkzDefPoint(0,3){C}
  \tkzDrawPolygon(A,B,C)
  \tkzDrawPoints(A,B,C)
  \end{tikzpicture}
\end{tkzexample}

\subsubsection{Calculations with \tkzNamePack{xfp}}

 \begin{tkzexample}[latex=7cm,small]
\begin{tikzpicture}[scale=1]
  \tkzInit[xmax=4,ymax=4]
  \tkzGrid
  \tkzDefPoint(-1+2,sqrt(4)){O}
  \tkzDefPoint({3*ln(exp(1))},{exp(1)}){A}
  \tkzDefPoint({4*sin(pi/6)},{4*cos(pi/6)}){B}
  \tkzDrawPoints[color=blue](O,B,A)
\end{tikzpicture}
\end{tkzexample}


\subsubsection{Polar coordinates }

\begin{tkzexample}[latex=7cm,small]
  \begin{tikzpicture}
  \foreach \an [count=\i] in {0,60,...,300}
   { \tkzDefPoint(\an:3){A_\i}}
  \tkzDrawPolygon(A_1,A_...,A_6)
  \tkzDrawPoints(A_1,A_...,A_6)
  \end{tikzpicture}
\end{tkzexample}

\subsubsection{Calculations and coordinates}
You must follow the syntax of \tkzNamePack{xfp} here. It is always possible to go through \tkzNamePack{pgfmath} but in this case, the coordinates must be calculated before using the macro \tkzcname{tkzDefPoint}.

\begin{tkzexample}[latex=6cm,small]
  \begin{tikzpicture}[scale=.5]
  \foreach \an [count=\i] in {0,2,...,358}
   { \tkzDefPoint(\an:sqrt(sqrt(\an mm))){A_\i}}
   \tkzDrawPoints(A_1,A_...,A_180)
  \end{tikzpicture}
\end{tkzexample}


\subsubsection{Relative points}
First, we can use the \tkzNameEnv{scope} environment from \TIKZ.
In the following example, we have a way to define an equilateral triangle.

\begin{tkzexample}[latex=7cm,small]
\begin{tikzpicture}[scale=1]
  \tkzSetUpLine[color=blue!60]
 \begin{scope}[rotate=30]
  \tkzDefPoint(2,3){A}
  \begin{scope}[shift=(A)]
     \tkzDefPoint(90:5){B}
     \tkzDefPoint(30:5){C}
  \end{scope}
 \end{scope}
 \tkzDrawPolygon(A,B,C)
\tkzLabelPoints[above](B,C)
\tkzLabelPoints[below](A)
\tkzDrawPoints(A,B,C)
\end{tikzpicture}
\end{tkzexample}

%<--------------------------------------------------------------------------->
\subsection{Point relative to another: \tkzcname{tkzDefShiftPoint}}
\begin{NewMacroBox}{tkzDefShiftPoint}{\oarg{Point}\parg{$x,y$}\marg{name} or \parg{$\alpha$:$d$}\marg{name}}%
\begin{tabular}{lll}%
arguments &  default & definition \\
\midrule
\TAline{($x,y$)}{no default}{$x$ and $y$ are two dimensions, by default in cm.}
\TAline{($\alpha$:$d$)}{no default}{$\alpha$ is an angle in degrees, $d$ is a dimension}

\midrule
options &  default & definition \\

\midrule
\TOline{[pt]} {no default} {\tkzcname{tkzDefShiftPoint}[A](0:4)\{B\}}
\end{tabular}
\end{NewMacroBox}

\subsubsection{Isosceles triangle with  \tkzcname{tkzDefShiftPoint}}
This macro allows you to place one point relative to another. This is equivalent to a translation. Here is how to construct an isosceles triangle with main vertex $A$ and angle at vertex of $30^{\circ} $.

\begin{tkzexample}[latex=7cm,small]
\begin{tikzpicture}[rotate=-30]
 \tkzDefPoint(2,3){A}
 \tkzDefShiftPoint[A](0:4){B}
 \tkzDefShiftPoint[A](30:4){C}
 \tkzDrawSegments(A,B B,C C,A)
 \tkzMarkSegments[mark=|,color=red](A,B A,C)
 \tkzDrawPoints(A,B,C)
 \tkzLabelPoints(B,C)
 \tkzLabelPoints[above left](A)
\end{tikzpicture}
\end{tkzexample}

\subsubsection{Equilateral triangle}
Let's see how to get an equilateral triangle (there is much simpler)

\begin{tkzexample}[latex=7cm,small]
\begin{tikzpicture}[scale=1]
 \tkzDefPoint(2,3){A}
 \tkzDefShiftPoint[A](30:3){B}
 \tkzDefShiftPoint[A](-30:3){C}
 \tkzDrawPolygon(A,B,C)
 \tkzDrawPoints(A,B,C)
 \tkzLabelPoints(B,C)
 \tkzLabelPoints[above left](A)
 \tkzMarkSegments[mark=|,color=red](A,B A,C B,C)
\end{tikzpicture}
\end{tkzexample}

\subsubsection{Parallelogram}
There's a simpler way
\begin{tkzexample}[latex=7cm,small]
\begin{tikzpicture}
 \tkzDefPoint(0,0){A}
 \tkzDefPoint(30:3){B}
 \tkzDefShiftPointCoord[B](10:2){C}
 \tkzDefShiftPointCoord[A](10:2){D}
 \tkzDrawPolygon(A,...,D)
 \tkzDrawPoints(A,...,D)
\end{tikzpicture}
\end{tkzexample}

%<--------------------------------------------------------------------------->
\subsection{Definition of multiple points: \tkzcname{tkzDefPoints}}

\begin{NewMacroBox}{tkzDefPoints}{\oarg{local options}\marg{$x_1/y_1/n_1,x_2/y_2/n_2$, ...}}%
$x_i$ and $y_i$ are the coordinates of a referenced point $n_i$

\begin{tabular}{lll}%
\toprule
arguments &  default  & example  \\
\midrule
\TAline{$x_i/y_i/n_i$}{}{\tkzcname{tkzDefPoints\{0/0/O,2/2/A\}}}
\end{tabular}

\medskip
\begin{tabular}{lll}%
options             & default & definition   \\
\midrule
\TOline{shift} {no default} {Adds $(x,y)$ or $(\alpha:d)$ to all coordinates}
\end{tabular}
\end{NewMacroBox}

\subsection{Create a triangle}
\begin{tkzexample}[latex=6cm,small]
\begin{tikzpicture}[scale=1]
 \tkzDefPoints{0/0/A,4/0/B,4/3/C}
 \tkzDrawPolygon(A,B,C)
 \tkzDrawPoints(A,B,C)
\end{tikzpicture}
\end{tkzexample}

\subsection{Create a square}
Note here the syntax for drawing the polygon.
\begin{tkzexample}[latex=6cm,small]
\begin{tikzpicture}[scale=1]
 \tkzDefPoints{0/0/A,2/0/B,2/2/C,0/2/D}
 \tkzDrawPolygon(A,...,D)
 \tkzDrawPoints(A,B,C,D)
\end{tikzpicture}
\end{tkzexample}

\section{Special points}
The introduction of the dots was done in \tkzname{tkz-base}, the most important macro being \tkzcname{tkzDefPoint}. Here are some special points.
%<--------------------------------------------------------------------------->
\subsection{Middle of a segment \tkzcname{tkzDefMidPoint}}
It is a question of determining the middle of a segment.

\begin{NewMacroBox}{tkzDefMidPoint}{\parg{pt1,pt2}}%
The result is in \tkzname{tkzPointResult}. We can access it with \tkzcname{tkzGetPoint}.

 \medskip
\begin{tabular}{lll}%
\toprule
arguments & default & definition \\
\midrule
\TAline{(pt1,pt2)}{no default}{pt1 and pt2 are two points}
\end{tabular}
\end{NewMacroBox}

\subsubsection{Use of \tkzcname{tkzDefMidPoint}}
Review the use of \tkzcname{tkzDefPoint} in \tkzNamePack{tkz-base}.
\begin{tkzexample}[latex=7cm,small]
\begin{tikzpicture}[scale=1]
 \tkzDefPoint(2,3){A}
 \tkzDefPoint(4,0){B}
 \tkzDefMidPoint(A,B) \tkzGetPoint{C}
 \tkzDrawSegment(A,B)
 \tkzDrawPoints(A,B,C)
 \tkzLabelPoints[right](A,B,C)
\end{tikzpicture}
\end{tkzexample}

\subsection{Barycentric coordinates }

$pt_1$, $pt_2$, \dots, $pt_n$ being $n$ points, they define $n$ vectors $\overrightarrow{v_1}$, $\overrightarrow{v_2}$, \dots, $\overrightarrow{v_n}$ with the origin of the referential as the common endpoint. $\alpha_1$, $\alpha_2$,
\dots $\alpha_n$ are $n$ numbers, the vector obtained by:
\begin{align*}
  \frac{\alpha_1 \overrightarrow{v_1} + \alpha_2 \overrightarrow{v_2} + \cdots + \alpha_n \overrightarrow{v_n}}{\alpha_1
    + \alpha_2 + \cdots + \alpha_n}
\end{align*}
defines a single point.

\begin{NewMacroBox}{tkzDefBarycentricPoint}{\parg{pt1=$\alpha_1$,pt2=$\alpha_2$,\dots}}%
\begin{tabular}{lll}%
arguments & default & definition \\
\midrule
\TAline{(pt1=$\alpha_1$,pt2=$\alpha_2$,\dots)}{no default}{Each point has a assigned weight}
\bottomrule
\end{tabular}

\medskip
You need at least two points.
\end{NewMacroBox}


\subsubsection{Using \tkzcname{tkzDefBarycentricPoint} with two points}
In the following example, we obtain the barycentre of points $A$ and $B$ with coefficients $1$ and $2$, in other words:
\[
  \overrightarrow{AI}= \frac{2}{3}\overrightarrow{AB}
\]

\begin{tkzexample}[latex=7cm,small]
\begin{tikzpicture}
  \tkzDefPoint(2,3){A}
  \tkzDefShiftPointCoord[2,3](30:4){B}
  \tkzDefBarycentricPoint(A=1,B=2)
  \tkzGetPoint{I}
  \tkzDrawPoints(A,B,I)
  \tkzDrawLine(A,B)
  \tkzLabelPoints(A,B,I)
\end{tikzpicture}
\end{tkzexample}

\subsubsection{Using \tkzcname{tkzDefBarycentricPoint} with three points}
This time $M$ is simply the centre of gravity of the triangle. For reasons of simplification and homogeneity, there is also \tkzcname{tkzCentroid}.
\begin{tkzexample}[latex=7cm,small]
\begin{tikzpicture}[scale=.8]
  \tkzDefPoint(2,1){A}
  \tkzDefPoint(5,3){B}
  \tkzDefPoint(0,6){C}
  \tkzDefBarycentricPoint(A=1,B=1,C=1)
  \tkzGetPoint{M}
  \tkzDefMidPoint(A,B)  \tkzGetPoint{C'}
  \tkzDefMidPoint(A,C)  \tkzGetPoint{B'}
  \tkzDefMidPoint(C,B)  \tkzGetPoint{A'}
  \tkzDrawPolygon(A,B,C)
  \tkzDrawPoints(A',B',C')
  \tkzDrawPoints(A,B,C,M)
  \tkzDrawLines[add=0 and 1](A,M B,M C,M)
  \tkzLabelPoint(M){$M$}
  \tkzAutoLabelPoints[center=M](A,B,C)
  \tkzAutoLabelPoints[center=M,above right](A',B',C')
\end{tikzpicture}
\end{tkzexample}

\subsection{Internal Similitude Center}
The centres of the two homotheties in which two circles correspond are called external and internal centres of similitude.

\begin{tkzexample}[latex=6cm,small]
\begin{tikzpicture}[scale=.75,rotate=-30]
 \tkzDefPoint(0,0){O}
 \tkzDefPoint(4,-5){A}
 \tkzDefIntSimilitudeCenter(O,3)(A,1)
 \tkzGetPoint{I}
 \tkzExtSimilitudeCenter(O,3)(A,1)
 \tkzGetPoint{J}
 \tkzDefTangent[from with R= I](O,3 cm)
 \tkzGetPoints{D}{E}
 \tkzDefTangent[from with R= I](A,1 cm)
 \tkzGetPoints{D'}{E'}
 \tkzDefTangent[from  with R= J](O,3 cm)
 \tkzGetPoints{F}{G}
 \tkzDefTangent[from with R= J](A,1 cm)
 \tkzGetPoints{F'}{G'}
 \tkzDrawCircle[R,fill=red!50,opacity=.3](O,3 cm)
 \tkzDrawCircle[R,fill=blue!50,opacity=.3](A,1 cm)
 \tkzDrawSegments[add = .5 and .5,color=red](D,D' E,E')
 \tkzDrawSegments[add= 0 and 0.25,color=blue](J,F J,G)
 \tkzDrawPoints(O,A,I,J,D,E,F,G,D',E',F',G')
 \tkzLabelPoints[font=\scriptsize](O,A,I,J,D,E,F,G,D',E',F',G')
\end{tikzpicture}
\end{tkzexample}

\endinput


 
%!TEX root = /Users/ego/Boulot/TKZ/tkz-euclide/doc_fr/TKZdoc-euclide-main.tex


\section{Définition aléatoire de points}
Il y a pour le moment quatre possibilités :
\begin{enumerate}
  \item point dans un rectangle,
  \item sur un segment,
  \item sur une droite,
  \item sur un cercle.
\end{enumerate}

\begin{NewMacroBox}{tkzGetRandPointOn}{\oarg{local options}\marg{name} }


\medskip
\begin{tabular}{lll}
\toprule
options     &     & définition                         \\ 
\midrule
\TOline{rectangle =  \#1 and \#2}{}{\#1 et \#2 sont des noms de points}
\TOline{segment =  \#1--\#2}{}{\#1 et \#2 sont des noms de points}
\TOline{line =  \#1--\#2}{}{\#1 et \#2 sont des noms de points}
\TOline{circle = center \#1 radius \#1 }{}{\#1 est un point et \#1 une mesure}
 \bottomrule
\end{tabular}

\medskip
\noindent\emph{Cette macro est assez simple à utiliser, voyez les exemples.}
\end{NewMacroBox} 

\subsection{Point aléatoire dans un rectangle} 

\begin{center}
\begin{tkzexample}[vbox]
\begin{tikzpicture}
  \tkzInit[xmax=5,ymax=5]  \tkzGrid   
  \tkzDefPoint(0,0){A}  \tkzDefPoint(2,2){B}
  \tkzDefPoint(5,5){C}
  \tkzGetRandPointOn[rectangle = A and B]{a}
  \tkzGetRandPointOn[rectangle = B and C]{d}
  \tkzDrawLine(a,d)
  \tkzDrawPoints(A,B,C,a,d) 
  \tkzLabelPoints(A,B,C,a,d)  
\end{tikzpicture} 
\end{tkzexample} 
\end{center}


\subsection{Point aléatoire sur un segment}  
\begin{tkzexample}[latex=6cm] 
\begin{tikzpicture}  
  \tkzInit[xmax=5,ymax=5] \tkzGrid   
  \tkzDefPoint(0,0){A} \tkzDefPoint(2,2){B}
  \tkzDefPoint(3,3){C} \tkzDefPoint(5,5){D}
  \tkzGetRandPointOn[segment = A--B]{a}
  \tkzGetRandPointOn[segment = C--D]{d}
  \tkzDrawPoints(A,B,C,D,a,d) 
  \tkzLabelPoints(A,B,C,D,a,d)
\end{tikzpicture} 
\end{tkzexample}

\subsection{Point aléatoire sur une droite}  
\begin{tkzexample}[latex=6cm] 
\begin{tikzpicture} 
  \tkzInit[xmax=5,ymax=5] \tkzGrid   
  \tkzDefPoint(0,0){A}  \tkzDefPoint(2,2){B}
  \tkzDefPoint(3,3){C}  \tkzDefPoint(5,5){D}
  \tkzGetRandPointOn[line = A--B]{a}
  \tkzGetRandPointOn[line = C--D]{d}
  \tkzDrawPoints(A,B,C,D,a,d) 
  \tkzLabelPoints(A,B,C,D,a,d)   
\end{tikzpicture}    
\end{tkzexample}

\subsection{Point aléatoire sur un cercle}  

\begin{tkzexample}[latex=5cm] 
\begin{tikzpicture} 
  \tkzInit[xmax=5,ymax=5]  \tkzGrid   
  \tkzDefPoint(3,2){A}  \tkzDefPoint(1,1){B}
  \tkzCalcLength[cm](A,B) \tkzGetLength{rAB}
  \tkzDrawCircle[R](A,\rAB cm) 
  \tkzGetRandPointOn[circle = center A radius \rAB cm]{a}
  \tkzDrawSegment(A,a)
  \tkzDrawPoints(A,B,a) 
  \tkzLabelPoints(A,B,a)  
\end{tikzpicture}
\end{tkzexample}


\newpage
\subsection{Milieu d'un segment au compas}  
 Pour terminer cette section, voici un exemple plus complexe. Il s'agit de déterminer le milieu d'un segment, uniquement avec un compas. 
 
\begin{center}
\begin{tkzexample}[vbox]
\begin{tikzpicture}[scale=.75]
  \tkzDefPoint(0,0){A}  
  \tkzGetRandPointOn[circle= center A radius 4cm]{B}
  \tkzDrawPoints(A,B)
  \tkzDefPointBy[rotation= center A angle 180](B) 
  \tkzGetPoint{C}
  \tkzInterCC[R](A,4 cm)(B,4 cm) 
  \tkzGetPoints{I}{I'}
  \tkzInterCC[R](A,4 cm)(I,4 cm) 
  \tkzGetPoints{J}{B}
  \tkzInterCC(B,A)(C,B) 
  \tkzGetPoints{D}{E}
  \tkzInterCC(D,B)(E,B) 
  \tkzGetPoints{M}{M'} 
  \tikzset{arc/.style={color=brown,style=dashed,delta=10}} 
  \tkzDrawArc[arc](C,D)(E) 
  \tkzDrawArc[arc](B,E)(D)
  \tkzDrawCircle[color=brown,line width=.2pt](A,B) 
  \tkzDrawArc[arc](D,B)(M) 
  \tkzDrawArc[arc](E,M)(B)
  \tkzCompasss[color=red,style=solid](B,I I,J J,C) 
  \tkzDrawPoints(B,C,D,E,M)    
 \end{tikzpicture}  
 \end{tkzexample}
\end{center}

\endinput
%!TEX root = /Users/ego/Boulot/TKZ/tkz-euclide/doc_fr/TKZdoc-euclide-main.tex


\section{Définition de points par transformation; \tkzcname{tkzDefPointBy} }
Ces transformations sont au nombre de sept :

\begin{enumerate}
   \item la translation;
   \item l'homothetie;
   \item la réflexion  ou symétrie orthogonale;
   \item la symétrie centrale;
   \item la projection orthogonale;
   \item la rotation;
   \item la rotation en radian;
   \item l'inversion par rapport à un cercle
\end{enumerate}

Le choix des transformations se fait par l'intermédiaire des options. Il y a deux macros l'une pour la transformation d'un unique point \tkzcname{tkzDefPointBy} et l'autre pour la transformation d'une liste de points \tkzcname{tkzDefPointsBy}. Dans le second cas, il faut donner en argument, les noms des images ou bien encore indiquer que le nom des images est formé à partir du nom des antécédents. Par défaut l'image de $A$ est $A'$. Par exemple, on écrira~:
\begin{tkzltxexample}[]
\tkzDefPointBy[translation= from A to A'](B) le résultat est dans tkzPointResult}
\tkzDefPointsBy[translation= from A to A'](B,C){} les images sont B' et C'
\tkzDefPointsBy[translation= from A to A'](B,C){D,E} les images sont D et E
\tkzDefPointsBy[translation= from A to A'](B) l'image est B'
\end{tkzltxexample}

La variante sans (s), évite l'usage d'une boucle et d'un test et est donc plus efficace.
 
\bigskip
\begin{NewMacroBox}{tkzDefPointBy}{\oarg{local options}\parg{pt}}
\emph{L'argument est un simple point existant et son image est stockée dans \tkzname{tkzPointResult}. Soit la création est une étape intermédiaire et vous n'avez pas besoin de conserver ce point alors tant qu'aucune macro ne modifie  l'attribution de \tkzname{tkzPointResult}, vous pouvez utiliser ce nom pour faire référence au point obtenu. Si vous voulez conserver ce point alors la macro  \tkzcname{tkzGetPoint\{M\}} permet d'attribuer le nom \tkzname{M} au point.}

\medskip
\begin{tabular}{lll}
\toprule
arguments &  définition  &   exemples               \\ 
\midrule
\TAline{pt}   {nom d'un point existant}   {$(A)$}
\bottomrule
\end{tabular}


\medskip
\begin{tabular}{lll}
options     &     & exemples                         \\ 
\midrule
\TOline{translation}{= from \#1 to \#2}{[translation=from A to B](E)}
\TOline{homothety}  {= center \#1 ratio \#2}{[homothety=center A ratio .5](E)}
\TOline{reflection} {= over \#1--\#2}{[reflection=over A--B](E)}
\TOline{symmetry }  {= center \#1}{[symmetry=center A](E)}
\TOline{projection }{= onto \#1--\#2}{[projection=onto A--B](E)}
\TOline{rotation }  {= center \#1 angle \#2}{[rotation=center O angle 30](E)}
\TOline{rotation in rad}{= center \#1 angle \#2}{rotation=center O angle pi/3} 
\TOline{inversion}{= center \#1 through \#2}{[inversion =center O through A](E)} 
\bottomrule
\end{tabular}

\medskip
\noindent\emph{ L'image est seulement définie et non tracée.}
\end{NewMacroBox} 

\newpage 
\subsection{La réflexion ou symétrie orthogonale } 

\subsubsection{Exemple de réflexion} 
\begin{center}
\begin{tkzexample}[vbox]
\begin{tikzpicture}[scale=1]
 \tkzInit[ymin=-4,ymax=6,xmin=-7,xmax=3]
 \tkzClip
 \tkzDefPoints{1.5/-1.5/C,-4.5/2/D}    
 \tkzDefPoint(-4,-2){O} 
 \tkzDefPoint(-2,-2){A}
 \foreach \i in {0,1,...,4}{%
 \pgfmathparse{0+\i * 72}
 \tkzDefPointBy[rotation=center O angle \pgfmathresult](A) \tkzGetPoint{A\i} 
 \tkzDefPointBy[reflection = over C--D](A\i) \tkzGetPoint{A\i'}}
 \tkzDrawPolygon(A0, A2, A4, A1, A3)    
 \tkzDrawPolygon(A0', A2', A4', A1', A3')
 \tkzDrawLine[add= .5 and .5](C,D)
\end{tikzpicture}
\end{tkzexample}
\end{center}
 
\newpage 
\subsection{L'homothétie}  
\subsubsection{Exemple d'homothétie et de projection}

\begin{center}
\begin{tkzexample}[vbox] 
\begin{tikzpicture}[scale=1.25] 
  \tkzInit   \tkzClip 
  \tkzDefPoint(0,1){A}   \tkzDefPoint(6,3){B}   \tkzDefPoint(3,6){C} 
  \tkzDrawLines[add= 0 and .3](A,B A,C) 
  \tkzDefLine[bisector](B,A,C)                     \tkzGetPoint{a} 
  \tkzDrawLine[add=0 and 0,color=magenta!50 ](A,a) 
  \tkzDefPointBy[homothety=center A ratio .5](a)   \tkzGetPoint{a'} 
  \tkzDefPointBy[projection = onto A--B](a')       \tkzGetPoint{k}   
  \tkzDrawSegment[style=dashed](a',k) 
  \tkzShowLine[bisector,size=2,gap=3](B,A,C) 
  \tkzDrawCircle(a',k)  
\end{tikzpicture}
\end{tkzexample}  
\end{center}


\newpage  
\subsection{La projection }  
\subsubsection{Exemple de projection}

\begin{center}
\begin{tkzexample}[vbox] 
\begin{tikzpicture}[scale=1.5]  
 \tkzInit[xmin=-3,xmax=5,ymax=4] \tkzClip[space=.5]
 \tkzDefPoint(0,0){A}
 \tkzDefPoint(0,4){B}
 \tkzDrawTriangle[pythagore](B,A) \tkzGetPoint{C}
 \tkzDefLine[bisector](B,C,A) \tkzGetPoint{c}
 \tkzInterLL(C,c)(A,B)        \tkzGetPoint{D}
 \tkzDrawSegment(C,D)
 \tkzDrawCircle(D,A)
 \tkzDefPointBy[projection=onto B--C](D) \tkzGetPoint{G}
 \tkzInterLC(C,D)(D,A) \tkzGetPoints{E}{F}
 \tkzDrawPoints(A,C,F) \tkzLabelPoints(A,C,F)
 \tkzDrawPoints(B,D,E,G)   
 \tkzLabelPoints[above right](B,D,E,G)
 \end{tikzpicture}
 \end{tkzexample} 
\end{center}


\newpage 
\subsection{La symétrie }  
\subsubsection{Exemple de symétrie}
 
 \begin{center}
\begin{tkzexample}[vbox] 
\begin{tikzpicture}[scale=2]
  \tkzDefPoint(0,0){O}
  \tkzDefPoint(2,-1){A}
  \tkzDefPoint(2,2){B}
  \tkzDefPointsBy[symmetry=center O](B,A){}
  \tkzDrawLine(A,A')
  \tkzDrawLine(B,B')
  \tkzMarkAngle[mark=s,arc=lll,size=2 cm,mkcolor=red](A,O,B) 
  \tkzLabelAngle[pos=1,circle,draw,fill=blue!10](A,O,B){$60^{\circ}$}  
\end{tikzpicture}  
\end{tkzexample}
\end{center}

\newpage  
\subsection{La rotation }  
\subsubsection{Exemple de rotation} 

 \begin{center}
\begin{tkzexample}[vbox] 
 \begin{tikzpicture}[scale=1.2,rotate=-90] 
 \tkzInit
 \tkzPoint(0,0){A} \tkzPoint(5,0){B}
 \tkzDrawSegment(A,B)
 \tkzDefPointBy[rotation= center A angle 60](B) 
 \tkzGetPoint{C} 
 \tkzDefPointBy[symmetry= center C](A) 
 \tkzGetPoint{D} 
 \tkzDrawSegment(A,tkzPointResult) 
 \tkzDrawLine(B,D)
 \tkzDrawArc[delta=10](A,B)(C) 
 \tkzDrawArc[delta=10](B,C)(A)
 \tkzDrawArc[delta=10](C,D)(D)  
 \tkzMarkRightAngle(D,B,A)  
\end{tikzpicture}  
\end{tkzexample}  
 \end{center} 

\newpage 
\subsection{La rotation en radian }  
\subsubsection{Exemple de rotation en radian} 
 
\begin{center}
\begin{tkzexample}[vbox]
\begin{tikzpicture} 
  \tkzInit\tkzGrid[sub]
  \tkzPoint[pos=left](1,5){A} 
  \tkzPoint(5,2){B}
  \tkzDrawSegment(A,B)
  \tkzDefPointBy[rotation in rad= center A angle pi/3](B)
  \tkzGetPoint{C}  
  \tkzCompass[color=red](A,C)
  \tkzCompass[color=red](B,C) 
\end{tikzpicture}
\end{tkzexample} 
\end{center}

\newpage 
\subsection{L'inversion par rapport à un cercle }
\subsubsection{Inversion de points}

\begin{center}
\begin{tkzexample}[vbox]  
\begin{tikzpicture}[scale=2]
  \tkzDefPoint(0,0){O}
  \tkzDefPoint(1,0){A}
  \tkzDrawCircle(O,A) 
  \tkzDefPoint(-1.5,-1.5){z1}
  \tkzDefPoint(0.35,0){z2} 
  \tkzDrawPoints[fill=red,color=black,size=8](O,z1,z2)   
  \tkzDefPointBy[inversion = center O through A](z1)
  \tkzGetPoint{Z1} 
  \tkzDefPointBy[inversion = center O through A](z2)
  \tkzGetPoint{Z2} 
  \tkzDrawPoints[fill=red,color=black,size=8](Z1,Z2)    
  \tkzDrawSegments(z1,Z1 z2,Z2)
  \tkzLabelPoints(O,A,z1,z2,Z1,Z2)  
\end{tikzpicture}
\end{tkzexample} 
\end{center}  

\subsubsection{Inversion de point : cercles orthogonaux} 

\begin{center}
\begin{tkzexample}[vbox]
\begin{tikzpicture}[scale=3]
  \tkzDefPoint(0,0){O}
  \tkzDefPoint(1,0){A}
  \tkzDrawCircle(O,A) 
  \tkzDefPoint(0.5,-0.25){z1}
  \tkzDefPoint(-0.5,-0.5){z2}
  \tkzDefPointBy[inversion = center O through A](z1)
  \tkzGetPoint{Z1} 
  \tkzCircumCenter(z1,z2,Z1)\tkzGetPoint{c}
  \tkzDrawCircle(c,Z1)
  \tkzDrawPoints[color=black,fill=red,size=12](O,z1,z2,Z1,O,A) 
\end{tikzpicture}
\end{tkzexample}
\end{center}


\newpage
Il existe une variante de cette macro pour la définition de multiples images

\begin{NewMacroBox}{tkzDefPointsBy}{\oarg{local options}\parg{liste de pts}\marg{liste de pts}}
\begin{tabular}{lll}
\toprule
arguments &  exemples  &                  \\ 
\midrule
\TAline{\parg{liste de pts}\marg{liste de pts}}{(A,B)\{E,F\}}{E est l'image de A et F celle de B.}   \\
\bottomrule
\end{tabular}

\medskip
\emph{Si la liste des images est vide alors le nom de l'image est le nom de l'antécédent auquel on ajoute « ' »}

\medskip
\begin{tabular}{lll}
\toprule
options     &     & exemples                         \\ 
\midrule
\TOline{translation = from \#1 to \#2}{}{[translation=from A to B](E)\{\}}
\TOline{homothety = center \#1 ratio \#2}{}{[homothety=center A ratio .5](E)\{F\}}
\TOline{reflection = over \#1--\#2}{}{[reflection=over A--B](E)\{F\}}
\TOline{symmetry = center \#1}{}{[symmetry=center A](E)\{F\}}
\TOline{projection = onto \#1--\#2}{}{[projection=onto A--B](E)\{F\}}
\TOline{rotation = center \#1 angle \#2}{}{[rotation=center  angle 30](E)\{F\}}
\TOline{rotation in rad = center \#1 angle \#2}{}{par exemple angle pi/3}
\bottomrule
\end{tabular}

\medskip
\noindent\emph{ Les points sont seulement définis et non tracés.}
\end{NewMacroBox}

\subsection{Exemple de translation}

\begin{tkzexample}[vbox,small]
\begin{tikzpicture} 
 \tkzDefPoint(0,0){A}  \tkzDefPoint(5,2){A'}
 \tkzDefPoint(3,0){B}  \tkzDefPoint(1,2){C} 
 \tkzDefPointsBy[translation= from A to A'](B,C){} 
 \tkzDrawPolygon[color=blue](A,B,C)
 \tkzDrawPolygon[color=red](A',B',C')
 \tkzDrawPoints[color=blue](A,B,C)
 \tkzDrawPoints[color=red](A',B',C') 
 \tkzLabelPoints(A,B,A',B')  \tkzLabelPoints[above](C,C')
 \tkzDrawSegments[color = gray,->,style=dashed](A,A' B,B' C,C')   
\end{tikzpicture}
\end{tkzexample}

\newpage
\subsection{Fruit of Life}
\begin{center}
\begin{tkzexample}[vbox] 
\begin{tikzpicture}[scale=.8]
 \tkzDefPoint(0,0){O}  \tkzDefPoint(1.5,0){A}
 \tkzDrawCircle(O,A)
 \foreach \i in {0,...,5}{
  \tkzDefPointBy[rotation  = center O  angle 30+60*\i](A) \tkzGetPoint{a\i}
  \tkzDefPointBy[homothety = center O  ratio 2](a\i) \tkzGetPoint{b\i}
  \tkzDefPointBy[homothety = center O  ratio 3](a\i) \tkzGetPoint{c\i}
  \tkzDefPointBy[homothety = center O  ratio 4](a\i) \tkzGetPoint{d\i}
  \tkzDrawCircle(b\i,a\i) \tkzDrawCircle(d\i,c\i)
  }
\tkzDrawPolygon[color=red!50!Gold,ultra thick](d0,d1,d2,d3,d4,d5) 
\tkzDrawPolygon[color=red!50!Gold,ultra thick](b0,b2,b4)
\tkzDrawSegments[color=red!50!Gold,ultra thick](b0,d5 b0,d0 b0,d1 %
                              b2,d1 b2,d2 b2,d3 b4,d3 b4,d4 b4,d5)
\tkzDrawPoints[color=red!50!Gold,size=20](b0,b2,b4,d0,d1,d2,d3,d4,d5)
\end{tikzpicture}
\end{tkzexample} 
\end{center}

\newpage
\subsection{Flower of Life}

\begin{center}
\begin{tkzexample}[vbox]
\begin{tikzpicture}[scale=.6]
 \tkzSetUpLine[line width=2pt,color=orange!80!black] 
 \tkzSetUpCompass[line width=2pt,color=orange!80!black]
 \tkzDefPoint(0,0){O} \tkzDefPoint(2.25,0){A}
 \tkzDrawCircle(O,A)
 \foreach \i in {0,...,5}{
  \tkzDefPointBy[rotation= center O angle 30+60*\i](A)   \tkzGetPoint{a\i}
  \tkzDefPointBy[rotation= center {a\i} angle  120](O)   \tkzGetPoint{b\i}
  \tkzDefPointBy[rotation= center {a\i} angle  180](O)   \tkzGetPoint{c\i}
  \tkzDefPointBy[rotation= center {c\i} angle  120](a\i) \tkzGetPoint{d\i}
  \tkzDefPointBy[rotation= center {c\i} angle   60](d\i) \tkzGetPoint{f\i}
  \tkzDefPointBy[rotation= center {d\i} angle   60](b\i) \tkzGetPoint{e\i} 
  \tkzDefPointBy[rotation= center {f\i} angle   60](d\i) \tkzGetPoint{g\i} 
  \tkzDefPointBy[rotation= center {d\i} angle   60](e\i) \tkzGetPoint{h\i}
  \tkzDefPointBy[rotation= center {e\i} angle  180](b\i) \tkzGetPoint{k\i}
  
  \tkzDrawCircle(a\i,O) \tkzDrawCircle(b\i,a\i)
  \tkzDrawCircle(c\i,a\i)
  \tkzDrawArc[rotate](f\i,d\i)(-120)
  \tkzDrawArc[rotate](e\i,d\i)(180)
  \tkzDrawArc[rotate](d\i,f\i)(180)
  \tkzDrawArc[rotate](g\i,f\i)(60)
  \tkzDrawArc[rotate](h\i,d\i)(60)
  \tkzDrawArc[rotate](k\i,e\i)(60) }
 \tkzClipCircle(O,f0)
\end{tikzpicture} 
\end{tkzexample}
\end{center}

\clearpage\newpage
\subsection{Sangaku cercle et carré}
Dans cet exemple, on peut voir comment utiliser un point sans le nommer

\begin{center}
\begin{tkzexample}[vbox]
\begin{tikzpicture}[scale = 1]
   \tkzInit[xmax = 8] \tkzClip
   \tkzDefPoint(0,0){B}
   \tkzDefPoint(0,8){A}
   \tkzDefSquare(A,B)
   \tkzGetPoints{C}{D}
   \tkzDrawSquare(A,B)
   \tkzClipPolygon(A,B,C,D)
   \tkzDefPoint(4,8){F}
   \tkzDefPoint(4,0){E}
   \tkzDefPoint(4,4){Q}
   \tkzFillPolygon[color = green](A,B,C,D)
   \tkzDrawCircle[fill   = orange](B,A)
   \tkzDrawCircle[fill   = purple](E,B)  
   \tkzTgtFromP(F,A)(B)
   \tkzInterLL(F,tkzFirstPointResult)(C,D)
   \tkzInterLL(A,tkzPointResult)(F,E) 
   \tkzDrawCircle[fill = yellow](tkzPointResult,Q)  
   \tkzDefPointBy[projection= onto B--A](tkzPointResult)
   \tkzDrawCircle[fill = blue!50!black](tkzPointResult,A)
\end{tikzpicture}
\end{tkzexample}
\end{center}   

\newpage
\subsection{Constructions de certaines  transformations \addbs{tkzShowTransformation}}

 \begin{NewMacroBox}{tkzShowTransformation}{\oarg{local options}\parg{pt1,pt2} ou \parg{pt1,pt2,pt3}}
\emph{Ces constructions concernent les symétries  orthogonales, les symétries centrales, les projections orthogonales et les translations. Plusieurs options permettent l'ajustement des constructions. L'idée de cette macro revient à \tkzimp{Yves Combe}}
  

\medskip 
\begin{tabular}{lll}
\toprule
options             & défaut & définition                         \\ 
\midrule
\TOline{reflection= over pt1--pt2}{reflection}{constructions d'une symétrie orthogonale} 
\TOline{symmetry=center pt}{reflection}{constructions d'une symétrie centrale} 
\TOline{projection=onto pt1--pt2}{reflection}{constructions d'une projection}
\TOline{translation=from pt1 to pt2}{reflection}{constructions d'une translation}
\TOline{K}{1}{cercle inscrit dans à un triangle }
\TOline{length}{1}{longueur d'un arc}
\TOline{ratio} {.5}{rapport entre les longueurs des arcs}
\TOline{gap}{2}{placement le point de construction}
\TOline{size}{1}{rayon d'un arc (voir bissectrice)}
 \bottomrule
\end{tabular}

\emph{Il faut ajouter bien sûr tous les styles de \TIKZ\ pour les tracés}
\end{NewMacroBox}

\subsubsection{Exemple d'utilisation de \tkzcname{tkzShowTransformation}} 

\begin{center}
\begin{tkzexample}[latex=6cm,small]
\begin{tikzpicture}[scale=.8]
  \tkzInit[xmin=-4,xmax=4,ymin=-5,ymax=5]
  \tkzGrid \tkzClip \tkzPoint(0,0){O} \tkzPoint(2,-2){A}
  \tkzDefPoint(70:4){B} \tkzDrawPoints(A,O,B)
  \tkzLabelPoints(A,O,B)
  \tkzDrawLine[add= 2 and 2](O,A)
  \tkzDefPointBy[translation=from O to A](B) 
  \tkzGetPoint{C}
  \tkzDrawPoint[color=orange](C)  \tkzLabelPoints(C)
  \tkzShowTransformation[translation=from O to A,%
             length=2](B) 
  \tkzDrawVectors[color=orange](O,A B,C)  
  \tkzDefPointBy[reflection=over O--A](B) \tkzGetPoint{E}
  \tkzDrawSegment[blue](B,E)
  \tkzDrawPoint[color=blue](E)\tkzLabelPoints(E) 
  \tkzShowTransformation[reflection=over O--A,size=2](B)   
  \tkzDefPointBy[symmetry=center O](B) \tkzGetPoint{F} 
  \tkzDrawSegment[color=green](B,F)
  \tkzDrawPoint[color=green](F)\tkzLabelPoints(F)
  \tkzShowTransformation[symmetry=center O,%
                      length=2](B) 
  \tkzDefPointBy[projection=onto O--A](C) 
  \tkzGetPoint{H}    
  \tkzDrawSegments[color=magenta](C,H)
  \tkzDrawPoint[color=magenta](H)\tkzLabelPoints(H)
  \tkzShowTransformation[projection=onto O--A,%
                         color=red,size=3,gap=-2](C)   
\end{tikzpicture}
\end{tkzexample}
\end{center}

\subsubsection{Autre exemple d'utilisation de \tkzcname{tkzShowTransformation}} 

Vous retouverez cette figure, mais sans les traits de construction
\begin{tkzexample}[vbox]  
  \begin{tikzpicture}[scale=1.25]
  % on définit les points nécessaires 
  \tkzInit[ymin=-3]
  \tkzClip[space=1]
  \tkzDefPoint(0,0){A}
  \tkzDefPoint(8,0){B}
  \tkzDefPoint(3.5,10){I}
  \tkzDefMidPoint(A,B) \tkzGetPoint{O} 
  % syntaxe (liste de points) {liste des images} si vide on met des '
  \tkzDefPointBy[projection=onto A--B](I) \tkzGetPoint{J}
  \tkzInterLC(I,A)(O,A) \tkzGetPoints{M'}{M}
  \tkzInterLC(I,B)(O,A)  \tkzGetPoints{N}{N'}    
  \tkzDrawCircle[diameter](A,B)
   % attention plusieurs segments donc (s) espace entre les objets 
   % virgule entre les points
  \tkzDrawSegments(I,A I,B A,B B,M A,N) 
  % idem (s) et espace entre les objets
  \tkzMarkRightAngles(A,M,B A,N,B)  
  \tkzDrawSegment[style=dashed,color=blue](I,J)
  % tkzShowTransformation il y a aussi tkzShowLine 
  \tkzShowTransformation[projection=onto A--B,color=red,size=3,gap=-3](I)
  % on trace les points à la fin ainsi c'est plus propre, il n'y a rien 
  % par-dessus 
  \tkzDrawPoints[color=red](M,N)
  \tkzDrawPoints[color=blue](O,A,B,I) 
  %  \tkzLabelPoints version rapide de  \tkzLabelPoint on met automatiquement
  % $O$ etc ... sinon on traite chaque point l'un après l'autre avec
  %  \tkzLabelPoint(le point){son label}
  \tkzLabelPoints(O)  \tkzLabelPoints[above right](N,I) 
  \tkzLabelPoints[below left](M,A) 
\end{tikzpicture} 
\end{tkzexample} 
\endinput
%!TEX root = /Users/ego/Boulot/TKZ/tkz-euclide/doc_fr/TKZdoc-euclide-main.tex


\section{Intersections}



Il est possible de déterminer les coordonnées des points d'intersection entre deux droites, une droite et un cercle et deux cercles.

Les commandes associées n'ont pas d'arguments optionnels et l'usager doit lui même déterminer l'existence des points d'intersection.


\subsection{Intersection de deux droites}


 \begin{NewMacroBox}{tkzInterLL}{\parg{$A,B$}\parg{$C,D$}}
\emph{Définit le point d'intersection \tkzname{tkzPointResult} des deux droites $(AB)$ and $(CD)$. Les points connus sont donnés en couple (deux par droite) entre parenthèses, quant au point obtenu, son nom est placé entre accolades.}       

 \end{NewMacroBox}   
% 

\medskip
\subsubsection{exemple d'intersection entre deux droites}
\begin{center}
\begin{tkzexample}[vbox]
\begin{tikzpicture}[rotate=-30]
   \tkzDefPoint(2,1){A}   \tkzDefPoint(6,5){B}
   \tkzDefPoint(3,6){C}   \tkzDefPoint(5,2){D}
   \tkzDrawLines(A,B C,D)
   \tkzInterLL(A,B)(C,D)  \tkzGetPoint{I}
   \tkzDrawPoints[color=blue](A,B,C,D) \tkzDrawPoint[color=red](I)
\end{tikzpicture}
\end{tkzexample}
\end{center}  

De nombreux points particuliers sont obtenus avec cette macro par exemple l'orthocentre (OrthoCenter) voir \tkzcname{tkzOrthoCenter}, le centre du cercle circonscrit à un triangle \tkzcname{tkzCircumCenter}. 

\newpage
\subsection{Intersection d'une droite et d'un cercle} % (fold)
\label{sub:intersection_d_une_droite_et_d_un_cercle}
Pour avoir une syntaxe homogène, l'option pour définir le cercle à l'aide de la mesure du rayon est \tkzname{R} comme pour les macros pour  le cercle , les arcs et les secteurs.    

Comme précédemment, la droite est définie par un couple de points. Le cercle
 est aussi défini par un un couple :
 \begin{itemize}
  \item $(O,C)$ qui est un couple de points, le premier désigne le centre et le second est un point quelconque du cercle.
  \item $(O,r)$  La mesure $r$ est celle du rayon. Elle est exprimée soint en \emph{cm}, soit en \emph{pt}.
 \end{itemize}
 

\begin{NewMacroBox}{tkzInterLC}{\parg{$A,B$}\parg{$O,C/r$}\marg{$I$}\marg{$J$}}
Les arguments sont donc deux couples. Le premier couple est un couple de points, le second est soit un couple de points si aucune option n'est utilisée ou bien si l'option \tkzname{N} est utilisée sinon le couple est constitué d'un point (le centre du cercle et d'une mesure, celle du rayon).

\medskip
\begin{tabular}{lll}
\toprule
options            & défaut  & définition                         \\ 
\midrule
\TOline{N}        {N}    { (O,C) détermine le cercle}
\TOline{R}        {N}    { (O, 1 cm) ou (O, 120 pt)}  
\bottomrule
\end{tabular}

\medskip   
\emph{La macro définit les points d' intersection $I$ et $J$ de la droite $(AB)$ et du cercle de centre $O$ de rayon $r$ s'ils existent; dans le cas contraire, une erreur sera signalée dans le fichier .log}
\end{NewMacroBox}

\subsubsection{Exemple simple d'intersection droite-cercle}

Dans l'exemple suivant, le tracé du cercle utilise deux points et  l'intersection de la droite et du cercle utilise deux couples de points

\begin{tkzexample}[latex=7cm]
\begin{tikzpicture}
   \tkzInit[xmax=5,ymax=4]
 \tkzDefPoint(1,1){O} 
 \tkzDefPoint(0,4){A} 
 \tkzDefPoint(5,0){B} 
 \tkzDefPoint(3,3){C}
 \tkzInterLC(A,B)(O,C)  \tkzGetPoints{D}{E}  
 \tkzDrawCircle(O,C)
 \tkzDrawPoints[color=blue](O,A,B,C)
 \tkzDrawPoints[color=red](D,E)
 \tkzDrawLine(A,B)
 \tkzLabelPoints[above right](O,A,B,C,D,E)
\end{tikzpicture} 
\end{tkzexample}  

\subsubsection{Exemple plus complexe d'intersection droite-cercle}
\url{http://gogeometry.com/problem/p190_tangent_circle}

\begin{center}
\begin{tkzexample}[vbox]
\begin{tikzpicture}[scale=1.25]
  \tkzInit[xmin=0,xmax=8,ymin=-4,ymax=4]  \tkzClip[space=.4]
  \tkzDefPoint(0,0){A}  \tkzDefPoint(8,0){B}
  \tkzDefMidPoint(A,B)  \tkzGetPoint{O}
  \tkzDrawCircle(O,B)
  \tkzDefMidPoint(O,B)  \tkzGetPoint{O'}
  \tkzDrawCircle(O',B)
  \tkzTangent[from=A](O',B) \tkzGetSecondPoint{E}
  \tkzInterLC(A,E)(O,B)     \tkzGetSecondPoint{D}
  \tkzDefPointBy[projection=onto A--B](D)  \tkzGetPoint{F}
  \tkzMarkRightAngle(D,F,B)
  \tkzDrawSegments(A,D A,B D,F) 
  \tkzDrawSegments[color=red,line width=1pt,opacity=.4](A,O F,B)
  \tkzDrawPoints(A,B,O,O',E,D)  \tkzLabelPoints(A,B,O,O',E,D) 
\end{tikzpicture}
\end{tkzexample}
\end{center}

 

\newpage
\subsubsection{Cercle défini par un centre et une mesure, et cas particuliers}
Regardons quelques cas particuliers comme des droites tangentes au cercle. 

\begin{center}
 
\begin{tkzexample}[vbox]
\begin{tikzpicture}[scale=.75]
  \tkzDefPoint(0,8){A}  \tkzDefPoint(8,0){B}
  \tkzDefPoint(8,8){C}  \tkzDefPoint(4,4){I}
  \tkzDefPoint(2,7){E}  \tkzDefPoint(6,4){F}  
  \tkzDrawCircle[R](I,4 cm)
  \tkzInterLC[R](A,C)(I,4 cm)  \tkzGetPoints{I1}{I2}
  \tkzInterLC[R](B,C)(I,4 cm)  \tkzGetPoints{J1}{J2}
  \tkzInterLC[R](A,B)(I,4 cm)  \tkzGetPoints{K1}{K2}
  \tkzDrawPoints[color=red](I1,J1,K1,K2)
  \tkzDrawLines(A,B B,C A,C)
  \tkzInterLC[R](E,F)(I,4 cm)  \tkzGetPoints{I2}{J2}  
  \tkzDrawPoints[color=blue](E,F)
  \tkzDrawPoints[color=red](I2,J2)
  \tkzDrawLine(I2,J2)\end{tikzpicture}
\end{tkzexample}  
 
\end{center}

\newpage
\subsubsection{Exemple plus complexe}
Attention à la syntaxe. Tout d'abord, les calculs pour les points peuvent être faits pendant le passage des arguments, mais il faut respecter la syntaxe de \tkzname{fp}. Vous pouvez constater que j'utilise la macro  \tkzcname{FPpi} car \tkzname{fp} travaille en radians !. De plus quand des calculs nécéssitent l'emploi de parenthèses, celles-ci doivent être insérées dans un groupe \TEX \{ \dots \}.


\begin{center}
\begin{tkzexample}[vbox]
\begin{tikzpicture}[scale=2.5,rotate=180]
  \tkzDefPoint(0,1){J} \tkzDefPoint(0,0){O}
  \tkzDrawCircle[R](O,1 cm)
  \tkzDrawArc[R,line width=1pt,color=Gold](J,2.5 cm)(180,0)
  \foreach \i in {0,-5,-10,...,-85}{
     \tkzDefPoint({2.5*cos(\i*\FPpi/180)},{1+2.5*sin(\i*\FPpi/180)}){P}
     \tkzDrawSegment[color=orange](J,P)
     \tkzInterLC[R](P,J)(O,1 cm) \tkzGetPoints{M}{N}
     \tkzDrawPoints(N)} 
  \foreach \i in {-90,-95,...,-175,-180}{
    \tkzDefPoint({2.5*cos(\i*\FPpi/180)},{1+2.5*sin(\i*\FPpi/180)}){P} 
    \tkzDrawSegment[color=orange](J,P)
    \tkzInterLC[R](P,J)(O,1 cm) \tkzGetPoints{M}{N}
    \tkzDrawPoints(M)}   
\end{tikzpicture}
\end{tkzexample} 
\end{center}

\newpage
\subsubsection{Calcul de la mesure du rayon} 
 Avec \tkzname{pgfmath} et \tkzcname{pgfmathsetmacro}   
 
La mesure du rayon peut être le résultat d'un calcul que l'on ne fera pas au sein de la macro d'intersection, mais avant. 
On peut calculer une longueur de plusieurs façons. Il est possible bien sûr,
 d'utiliser le module \tkzname{pgfmath} et la macro \tkzcname{pgfmathsetmacro}. Dans certains, les résultats obtenus ne sont pas assez précis ainsi le calcul suivant $0.0002 \div 0.0001$ donne 1.98 avec pgfmath alors que fp.sty donnera 2. C'est pour cela que j'ai préféré interdire le calcul pendant le passage de paramètres, cela permet à chacun de choisir sa méthode.
   
\begin{tkzexample}[latex=7cm]
\begin{tikzpicture}  
  \tkzDefPoint(2,2){A}
  \tkzDefPoint(5,4){B}
  \tkzDefPoint(4,4){O}
  \pgfmathsetmacro{\tkzLen}{0.0002/0.0001}
  \tkzDrawCircle[R](O,\tkzLen cm)
  \tkzInterLC[R](A,B)(O, \tkzLen cm) 
  \tkzGetPoints{I}{J}
  \tkzDrawPoints[color=blue](A,B)
  \tkzDrawPoints[color=red](I,J)
  \tkzDrawLine(I,J) 
\end{tikzpicture}
\end{tkzexample}

\subsubsection{Calcul de la mesure du rayon} 
Avec \tkzname{fp} et \tkzcname{FPeval}
  
\begin{tkzexample}[latex=7cm]
  \begin{tikzpicture}  
  \tkzDefPoint(2,2){A}
  \tkzDefPoint(5,4){B}
  \tkzDefPoint(4,4){O}
  \FPeval{\tkzLen}{0.0002/0.0001} 
  \tkzDrawCircle[R](O,\tkzLen cm)
  \tkzInterLC[R](A,B)(O, \tkzLen cm) 
  \tkzGetPoints{I}{J}
  \tkzDrawPoints[color=blue](A,B)
  \tkzDrawPoints[color=red](I,J)
  \tkzDrawLine(I,J) 
\end{tikzpicture}
  \end{tkzexample}

\subsubsection{Calcul de la mesure du rayon} 
 Avec \TEX\ et \tkzcname{tkzLength}. 
 
 Cette dimension a été créée avec \tkzcname{newdimen}. 2 cm a été transformé en points. Il est bien sûr possible  d'utiliser \TEX\ pour calculer.

\begin{tkzexample}[latex=7cm]   
\begin{tikzpicture}
  \tkzDefPoint(2,2){A}
  \tkzDefPoint(5,4){B}
  \tkzDefPoint(4,4){O}
  \tkzLength=2cm 
  \tkzDrawCircle[R](O,\tkzLength pt)
  \tkzInterLC[R](A,B)(O, \tkzLength pt)
   \tkzGetPoints{I}{J}
  \tkzDrawPoints[color=blue](A,B)
  \tkzDrawPoints[color=red](I,J)
  \tkzDrawLine(I,J) 
\end{tikzpicture}
\end{tkzexample} 



\subsubsection{Des carrés dans un demi-disque}
Un air de Sangaku ! Il s'agit de prouver que l'on peut inscrire dans un demi-disque, deux carrés, et de déterminer la longueur de leurs côtés respectifs en fonction du rayon.

\begin{center}
\begin{tkzexample}[vbox]
\begin{tikzpicture}[scale=1.5]
 \tkzInit[xmax=8,ymax=5]\tkzClip[space=.25] 
 \tkzDefPoint(0,0){A}
 \tkzDefPoint(8,0){B}
 \tkzDefPoint(4,0){I}
 \tkzDefSquare(A,B)    
   \tkzGetPoints{C}{D}
 \tkzInterLC(I,C)(I,B) 
   \tkzGetPoints{E'}{E}
 \tkzInterLC(I,D)(I,B) 
   \tkzGetPoints{F'}{F} 
 \tkzDefPointsBy[projection = onto A--B](E,F){H,G}
 \tkzDefPointsBy[symmetry   = center H](I){J}
 \tkzDefSquare(H,J)
   \tkzGetPoints{K}{L}
 \tkzDrawSector[fill=Maroon!30](I,B)(A)
 \tkzFillPolygon[color=red!40](H,E,F,G)
 \tkzFillPolygon[color=blue!40](H,J,K,L)
 \tkzDrawPolySeg[color=red](H,E,F,G) 
 \tkzDrawPolySeg[color=red](J,K,L)
 \tkzDrawPoints(E,G,H,F,J,K,L)
\end{tikzpicture}
\end{tkzexample}      
\end{center}


\clearpage \newpage
\subsection{Intersection de deux cercles} 

Le cas le plus fréquent est celui de deux cercles définis par leur centre et un point, mais comme précédemment l'option \tkzname{R} permet d'utiliser les mesures des rayons

\begin{NewMacroBox}{tkzInterCC}{\oarg{options}\parg{$O,A/r$}\parg{$O',A'/r'$}\marg{$I$}\marg{$J$}}

\medskip
\begin{tabular}{lll}
\toprule
options       & défaut  & définition                         \\ 
\midrule
\TOline{N}   {N}    {OA et O'A' sont des rayons, O et O' les centres}
\TOline{R}   {N}    {$r$ et $r'$ sont des dimensions et mesurent les rayons}   
\bottomrule
\end{tabular}

\medskip
\noindent
\emph{Cette macro définit le(s) point(s) d' intersection $I$ et $J$ des deux cercles de centre $O$ et $O'$. Si les deux cercles n'ont pas de point commun alors la macro se termine par une erreur qui n'est pas gérée. \\ 
Il est également possible d'utiliser directement \tkzcname{tkzInterCCN} et  \tkzcname{tkzInterCCR}.}
\end{NewMacroBox}   

\subsubsection{Construction d'un triangle connaissant les mesures des côtés}
On veut obtenir le triangle de Pythagore (3,4,5)  
\begin{center}
\begin{tkzexample}[vbox]  
\begin{tikzpicture}[scale=.8]
  \tkzDefPoint(0,0){A} \tkzDefPoint(5,0){B}
  \tkzDrawCircle[R,dashed](A,4 cm) \tkzDrawCircle[R,dashed](B,3 cm)
  \tkzInterCC[R](A,4 cm)(B,3 cm) \tkzGetPoints{C}{D}
  \tkzDrawPolygon(A,B,C)
  \tkzCompasss(A,C B,C) 
  \tkzLabelSegment[below](A,B){$5$ cm}
  \tkzLabelSegment[above left](A,C){$4$ cm}
  \tkzLabelSegment[above right](B,C){$3$ cm}
  \tkzDrawPoints[color=red](C) 
  \tkzDrawPoints[color=blue](A,B)
\end{tikzpicture}
\end{tkzexample}
\end{center}

\subsubsection{Dupliquer un triangle} 
Trois segments étant donnés, construire un triangle. Il s'agit de récupérer les mesures des longueurs avec \tkzcname{tkzCalcLength}.

\begin{tkzexample}[vbox]
\begin{tikzpicture}
 \tkzDefPoint(1,0){A}  \tkzDefPoint(4,0){B}   % On place les points   
 \tkzDefPoint(1,1){C}  \tkzDefPoint(5,1){D}
 \tkzDefPoint(1,2){E}  \tkzDefPoint(6,2){F}
 \tkzDefPoint(0,4){A'} \tkzDefPoint(3,4){B'}
 \tkzCalcLength[cm](C,D)\tkzGetLength{rCD}
 \tkzCalcLength[cm](E,F)\tkzGetLength{rEF}
 \tkzInterCC[R](A',\rCD cm)(B',\rEF cm)\tkzGetPoints{I}{J}
 \tkzDrawSegments[red](A,B C,D E,F) % Les tracés   
 \tkzDrawLine(A',B')    
 \tkzDrawPoints(D,E,I,J)
 \tkzDrawPolygon[color=red](A',B',I)
 \tkzSetUpLine[color=gray]
 \tkzCompass(A',B')
 \tkzDrawCircle[R](A',\rCD cm)
 \tkzDrawCircle[R](B',\rEF cm)
 \tkzDrawPoints(A,B,C,D,E,F,A',B',I)
 \tkzLabelPoints[left](A,C,E)
 \tkzLabelPoints[right](B,D,F)
 \tkzLabelPoints[below](A',B')
 \tkzLabelPoint[above left](I){$C'$}   
\end{tikzpicture} 
\end{tkzexample}

\subsubsection{Construction d'un triangle équilatéral}

\begin{tkzexample}[vbox] 
\begin{tikzpicture}[rotate=30] 
 \tkzDefPoint(1,1){A}
 \tkzDefPoint(5,1){B}
 \tkzInterCC(A,B)(B,A)\tkzGetPoints{C}{D}
 \tkzDrawPoint[color=black](C)
 \tkzDrawCircle[dashed](A,B)
 \tkzDrawCircle[dashed](B,A)
 \tkzCompass[color=red](A,C)
 \tkzCompass[color=red](B,C)
 \tkzDrawPolygon(A,B,C)
 \tkzLabelSegment[above left](A,C){$4$ cm}
 \tkzLabelSegment[above right](B,C){$4$ cm}
 \tkzLabelSegment[below](A,B){$4$ cm} 
 \tkzLabelPoints[](A,B)
 \tkzLabelPoint[above](C){$C$} 
\end{tikzpicture}
\end{tkzexample}

\subsubsection{Un triangle isocèle.}

\begin{tkzexample}[vbox]
\begin{tikzpicture}[rotate=30] 
 \tkzDefPoint(1,2){A}
 \tkzDefPoint(5,1){B}
 \tkzInterCC[R](A,5cm)(B,5cm)\tkzGetPoints{C}{D}
 \tkzDrawCircle[R,dashed](A,5 cm)
 \tkzDrawCircle[R,dashed](B,5 cm) 
 \tkzDrawPoint[color=blue](C) 
 \tkzCompass[color=red](A,C)
 \tkzCompass[color=red](B,C)
 \tkzDrawPolygon(A,B,C)
 \tkzLabelSegment[above left](A,C){$5$ cm}
 \tkzLabelSegment[above right](B,C){$5$ cm}
 \tkzLabelPoints[](A,B)
 \tkzLabelPoint[above](C){$C$}     
\end{tikzpicture}
\end{tkzexample} 

\subsubsection{Exemple une médiatrice}

\begin{center}
\begin{tkzexample}[]
\begin{tikzpicture}
  \tkzDefPoint(0,0){A} 
  \tkzDefPoint(3,3){B}  
  \tkzDrawCircle[color=blue](B,A)
  \tkzDrawCircle[color=blue](A,B)
  \tkzInterCC(B,A)(A,B)\tkzGetPoints{M}{N}
  \tkzDrawLine(A,B)
  \tkzDrawPoints(M,N)
  \tkzDrawLine[color=red](M,N)
\end{tikzpicture}
\end{tkzexample}
\end{center} 

\newpage
\subsubsection{Trisection d'un segment}
Voici un exemple complet utilisant toutes les macros précédentes. Il s'agit de partager avec une règle et un compas, un segment en trois segments de même longueur. 

\begin{center}
\begin{tkzexample}[vbox]
\begin{tikzpicture}[scale=.8] 
 \tkzDefPoint(0,0){A}  \tkzDefPoint(3,2){B}
 \tkzInterCC(A,B)(B,A) \tkzGetPoints{C}{D}
 \tkzInterCC(D,B)(B,A) \tkzGetPoints{A}{E}  
 \tkzInterCC(D,B)(A,B) \tkzGetPoints{F}{B}
 \tkzInterLC(E,F)(F,A) \tkzGetPoints{D}{G}   
 \tkzInterLL(A,G)(B,E) \tkzGetPoint{O}      
 \tkzInterLL(O,D)(A,B) \tkzGetPoint{J}
 \tkzInterLL(O,F)(A,B) \tkzGetPoint{I}
 \tkzDrawCircle(D,A)    \tkzDrawCircle(A,B)
 \tkzDrawCircle(B,A)    \tkzDrawCircle(F,A)
 \tkzDrawSegments[color=red](O,G O,B O,D O,F)
 \tkzDrawPoints(A,B,D,E,F,G,I,J)  \tkzLabelPoints(A,B,D,E,F,G,I,J)
 \tkzDrawSegments[blue](A,B B,D A,D A,F F,G E,G B,E)
 \tkzMarkSegments[mark=s|](A,I I,J J,B)
\end{tikzpicture}
\end{tkzexample} 
\end{center}
 
 \endinput 


%!TEX root = /Users/ego/Boulot/TKZ/tkz-euclide/doc_fr/TKZdoc-euclide-main.tex

\section{Les droites}

Il est bien sûr essentiel de tracer des droites, mais avant il faut pouvoir définir certaines droites particulières comme des médiatrices, des bissectrices, des parallèles ou encore des perpendiculaires. Le principe consiste à déterminer deux points de la droite. 
   

\subsection{Définition de droites}

\begin{NewMacroBox}{tkzDefLine}{\oarg{local options}\parg{pt1,pt2} ou \parg{pt1,pt2,pt3}}
\noindent\emph{L' argument est une liste de deux  ou trois points.    Suivant les cas, la macro définit un ou deux points nécessaires pour obtenir la droite cherchée. Il faut utiliser soit la macro \tkzcname{tkzGetPoint}, soit la macro \tkzcname{tkzGetPoints}.}
  

\medskip
\begin{tabular}{lll}
\toprule
options             & défaut & définition                         \\ 
\midrule
\TOline{mediator}{}{médiatrice. Deux points sont définis} 
\TOline{perpendicular=through\ldots}{}{perpendiculaire à une droite passant par un point} 
\TOline{orthogonal=through\ldots}{}{voir ci-dessus }
\TOline{parallel=through\ldots}{}{parallèle à une droite passant par un point}
\TOline{bisector}{}{bissectrice d'un angle défini par trois points}
\TOline{bisector out}{}{bissectrice extérieure}
\TOline{K}{1}{Coefficient  pour la droite perpendiculaire}
 \bottomrule
\end{tabular}
\end{NewMacroBox}  

\subsubsection{Exemple avec \tkzname{mediator}}  
\begin{tkzexample}[latex=5 cm]
\begin{tikzpicture}[rotate=25]
  \tkzInit
  \tkzDefPoints{-2/0/A,1/2/B}
  \tkzDefLine[mediator](A,B)          \tkzGetPoints{C}{D}
  \tkzDefPointWith[linear,K=.75](C,D) \tkzGetPoint{D}
  \tkzDefMidPoint(A,B)                \tkzGetPoint{I}
  \tkzFillPolygon[color=orange!30](A,C,B,D)
  \tkzDrawSegments(A,B C,D)
  \tkzMarkRightAngle(B,I,C) 
  \tkzDrawSegments(D,B D,A)
  \tkzDrawSegments(C,B C,A)
\end{tikzpicture}
\end{tkzexample}  

\subsubsection{Exemple avec \tkzname{orthogonal} et \tkzname{parallel}}    
\begin{tkzexample}[latex=5 cm]
\begin{tikzpicture}
   \tkzDefPoints{-1.5/-0.25/A,1/-0.75/B,-0.7/1/C}
   \tkzDrawLine[end   = $(d_1)$](A,B)
   \tkzDrawPoints(A,B,C)
   \tkzDefLine[orthogonal=through C](B,A) \tkzGetPoint{c}
   \tkzDrawLine[end   = $(\delta)$](C,c)
   \tkzInterLL(A,B)(C,c) \tkzGetPoint{I}
   \tkzMarkRightAngle(C,I,B) 
   \tkzDefLine[parallel=through C](A,B) \tkzGetPoint{c'}
   \tkzDrawLine[end   = $(d_2)$](C,c') 
   \tkzMarkRightAngle(I,C,c')   
\end{tikzpicture}
\end{tkzexample}

\subsection{Tracer une droite}

Pour tracer une droite, il suffit de donner les deux points et d'utiliser l'option \tkzname{add}. Cette option est due à Mark Wibrow 

\begin{tkzltxexample}[]
  \tikzset{%
    add/.style args={#1 and #2}{
        to path={%
 ($(\tikztostart)!-#1!(\tikztotarget)$)--($(\tikztotarget)!-#2!(\tikztostart)$)%
  \tikztonodes}}}
\end{tkzltxexample}
  
  Cela permet de tracer une partie d'une droite définie par deux points. On utilise pour cela deux valeurs, qui sont des pourcentages par rapport à la longueur du segment défini par les deux points.
  
\begin{tkzexample}[]
\begin{tikzpicture}
   \tkzDefPoints{0/0/A,5/0/B}
   \tkzDrawLine[color=blue,thin, add=1 and 1,end   = $(\delta)$](A,B) 
   \tkzDrawLine[color=red,thick, add=.5 and .5](A,B)
   \tkzDrawPoints(A,B)  \tkzLabelPoints(A,B)
    \tkzDrawLine[color=Maroon,line width=2pt, add=-.2 and -.2 ](A,B)  
  \end{tikzpicture} 
\end{tkzexample} 

 \begin{NewMacroBox}{tkzDrawLine}{\oarg{local options}\parg{pt1,pt2}}
\emph{Les arguments sont une liste de deux points.}

\begin{tabular}{lll}
\toprule
options             & défaut & définition                         \\ 
\midrule
\TOline{add= nb1 and nb2}{.2 and .2}{Permet de prolonger le segment} 
 \bottomrule
\end{tabular}

\medskip 
\emph{\tkzname{add} permet de définir la longueur du trait passant par les points pt1 et pt2. Les deux nombres sont des pourcentages. Les styles de \TIKZ\ sont accessibles pour les tracés}
\end{NewMacroBox}

\subsubsection{Exemple de tracer de droite avec \tkzname{add}}

\begin{tkzexample}[latex=5cm]
\begin{tikzpicture}
 \tkzInit[xmin=-2,xmax=3,ymin=-2.25,ymax=2.25]
 \tkzClip[space=.25]
 \tkzDefPoint(0,0){A} \tkzDefPoint(2,0.5){B}
 \tkzDefPoint(0,-1){C}\tkzDefPoint(2,-0.5){D} 
 \tkzDefPoint(0,1){E} \tkzDefPoint(2,1.5){F} 
 \tkzDefPoint(0,-2){G} \tkzDefPoint(2,-1.5){H}
  \tkzDrawLine(A,B)    \tkzDrawLine[add = 0 and .5](C,D) 
 \tkzDrawLine[add = 1 and 0](E,F)
  \tkzDrawLine[add = 0 and 0](G,H) 
 \tkzDrawPoints(A,B,C,D,E,F,G,H)    
 \tkzLabelPoints(A,B,C,D,E,F,G,H)  
\end{tikzpicture}
\end{tkzexample} 

\newpage
Il est possible de tracer plusieurs droites, mais avec les mêmes options.
\begin{NewMacroBox}{tkzDrawLines}{\oarg{local options}\parg{pt1,pt2 pt3,pt4 ...}}
\emph{Les arguments sont une liste de couples de deux points séparés par des espaces. Les styles de \TIKZ\ sont accessibles pour les tracés.}
\end{NewMacroBox}      

\subsubsection{Exemple avec \tkzcname{tkzDrawLines}}    
\begin{center}
\begin{tkzexample}[latex=7cm]
\begin{tikzpicture}
  \tkzDefPoint(0,0){A}
  \tkzDefPoint(2,0){B}
  \tkzDefPoint(1,2){C}
  \tkzDefPoint(3,2){D}   
  \tkzDrawLines(A,B C,D A,C B,D)
  \tkzLabelPoints(A,B,C,D)
\end{tikzpicture}
\end{tkzexample}
\end{center} 
 
\begin{center}
\begin{tkzexample}[vbox] 
\begin{tikzpicture}
 \tkzInit[xmin=-3,xmax=6, ymin=-1,ymax=6]
 \tkzClip
 \tkzDefPoint(0,0){O}
 \tkzDefPoint(3,1){I}
 \tkzDefPoint(1,4){J}
 \tkzDefLine[bisector](I,O,J)     \tkzGetPoint{i}   
 \tkzDefLine[bisector out](I,O,J) \tkzGetPoint{j}
 \tkzDrawLines[add = 1 and 1,color=red](O,I O,J) 
 \tkzDrawLines[add = 5 and 5,color=blue](O,i O,j) 
\end{tikzpicture} 
\end{tkzexample}
\end{center} 

\newpage
\subsubsection{Une enveloppe}
D'après une figure d'O. Reboux  avec pst-eucl de D Rodriguez
\begin{center}
\begin{tkzexample}[vbox]
\begin{tikzpicture}[scale=1.25]
  \tkzInit[xmin=-6,ymin=-6,xmax=6,ymax=6]  
  \tkzClip 
  \tkzDefPoint(0,0){O} 
  \tkzDefPoint(132:4){A}
  \tkzDefPoint(5,0){B}
  \foreach \ang in {5,10,...,360}{%
    \tkzDefPoint(\ang:5){M}
    \tkzDefLine[mediator](A,M)
    \tkzDrawLine[color=magenta,add= 4 and 4](tkzFirstPointResult,tkzSecondPointResult)}
\end{tikzpicture}
\end{tkzexample}
\end{center}

\newpage
\subsubsection{Une parabole}
D'après une figure d'O. Reboux  avec pst-eucl de D Rodriguez.
Il n'est pas nécessaire de nommer les deux points qui définissent la médiatrice.

\begin{center}
\begin{tkzexample}[vbox]
\begin{tikzpicture}[scale=1.25]
  \tkzInit[xmin=-6,ymin=-6,xmax=6,ymax=6]  
  \tkzClip 
  \tkzDefPoint(0,0){O} 
  \tkzDefPoint(132:5){A}
  \tkzDefPoint(4,0){B}
  \foreach \ang in {5,10,...,360}{%
    \tkzDefPoint(\ang:4){M}
    \tkzDefLine[mediator](A,M) 
    \tkzDrawLine[color=magenta,
             add= 4 and 4](tkzFirstPointResult,tkzSecondPointResult)}
   \end{tikzpicture}
\end{tkzexample}
\end{center}


\subsection{Ajouter des labels aux  droites \tkzcname{tkzLabelLine}} 

 \begin{NewMacroBox}{tkzLabelLine}{\oarg{local options}\parg{pt1,pt2}\marg{label}}

 \begin{tabular}{lll}
 \toprule
 arguments &  défaut  & définition                 \\ 
 \midrule
 \TAline{label}{}{exemple \tkzcname{tkzLabelLine(A,B)\{$\delta$\}}}
 \bottomrule
 \end{tabular}

\medskip
\begin{tabular}{lll}
\toprule
options             & défaut & définition                         \\ 
\midrule
\TOline{pos}{.5}{pos est une option de \TIKZ\ mais essentielle dans ce cas} 
 \bottomrule
\end{tabular}

\medskip
\emph{En option et en plus de \tkzname{pos}, on peut utiliser tous les styles de \TIKZ\ , en particulier le placement avec \tkzname{above}, \tkzname{right}, \dots}

 \end{NewMacroBox}

\subsubsection{Exemple avec \tkzcname{tkzLabelLine}}
Une option importante est \tkzname{pos}, c'est elle qui permet de placer le label le long de la droite. La valeur de \tkzname{pos} peut être supérieure à 1 ou négative.

\begin{tkzexample}[latex=4cm]
\begin{tikzpicture}
   \tkzInit[ymin=-1,ymax=1.5,xmin=-2,xmax=2.5]
   \tkzDefPoints{0/0/A,3/0/B,1/1/C}
   \tkzDefLine[perpendicular=through C,K=-1](A,B)
   \tkzGetPoint{c}
   \tkzDrawLines(A,B C,c)
   \tkzLabelLine[pos=1.25,blue,right](C,c){$(\delta)$} 
   \tkzLabelLine[pos=-0.25,red,left](C,c){encore $(\delta)$} 
\end{tikzpicture}
\end{tkzexample}


\subsection{Configurer les options pour les lignes \tkzcname{tkzSetUpLine}}
voir  \ref{tkzsetupline}
 
\newpage
\subsection{Montrer les constructions de certaines  lignes \tkzcname{tkzShowLine}}

 \begin{NewMacroBox}{tkzShowLine}{\oarg{local options}\parg{pt1,pt2} ou \parg{pt1,pt2,pt3}}
\emph{Ces constructions concernent les médiatrices, les droites perpendiculaires ou parallèles passant par un point donné et les bissectrices. Les arguments sont donc des listes de deux ou bien de trois points. Plusieurs options permettent l'ajustement des constructions. L'idée de cette macro revient à \tkzimp{Yves Combe}}
  

\medskip 
\begin{tabular}{lll}
\toprule
options             & défaut & définition                         \\ 
\midrule
\TOline{mediator}{mediator}{affiche les constructions d'une médiatrice} 
\TOline{perpendicular}{mediator}{constructions pour une perpendiculaire} 
\TOline{orthogonal}{mediator}{idem}
\TOline{bisector}{mediator}{constructions pour une bissectrice}
\TOline{K}{1}{cercle inscrit dans à un triangle }
\TOline{length}{1}{ en cm, longueur d'un arc}
\TOline{ratio} {.5}{rapport entre les longueurs des arcs}
\TOline{gap}{2}{placement le point de construction}
\TOline{size}{1}{rayon d'un arc (voir bissectrice)}
 \bottomrule
\end{tabular}

\emph{Il faut ajouter bien sûr tous les styles de \TIKZ\ pour les tracés}
\end{NewMacroBox}

\subsubsection{Exemple de \tkzcname{tkzShowLine} et \tkzname{parallel}} 

\begin{tkzexample}[latex=5cm]
\begin{tikzpicture}
    \tkzDefPoints{-1.5/-0.25/A,1/-0.75/B,-1.5/2/C}
    \tkzDrawLine(A,B)
    \tkzDefLine[parallel=through C](A,B)  \tkzGetPoint{c} 
    \tkzShowLine[parallel=through C](A,B)
    \tkzDrawLine(C,c)
    \tkzDrawPoints(A,B,C,c)
\end{tikzpicture}
\end{tkzexample}



\subsubsection{Exemple de \tkzcname{tkzShowLine} et \tkzname{perpendicular}} 

\begin{tkzexample}[latex=6cm]
\begin{tikzpicture}
  \tkzInit[xmin=0,xmax=6,ymin=0,ymax=6]
  \tkzClip
  \tkzDefPoint(0,0){A}
  \tkzDefPoint(3,4){B}  
  \tkzDefPoint(2,4){C}
  \tkzDefLine[perpendicular=through C,%
              K=-.5](A,B)
  \tkzGetPoint{c}
  \tkzDefPointBy[projection=onto A--B](c) 
  \tkzGetPoint{h}
  \tkzMarkRightAngle[fill=lightgray](A,h,C)
  \tkzDrawLines[](A,B C,c)
  \tkzDrawPoints(A,B,C,h,c)
\end{tikzpicture}
\end{tkzexample}

\subsubsection{Exemple de \tkzcname{tkzShowLine} et \tkzname{bisector}} 

\begin{tkzexample}[latex=5.25 cm]
\begin{tikzpicture}
 \tkzInit[xmin=0,xmax=7,ymin=0,ymax=7]
 \tkzClip 
 \tkzDefPoints{0/0/A, 6/2/B, 1/6/C}
 \tkzDrawPolygon(A,B,C)  
 \tkzSetUpCompass[color=brown,line width=.1 pt]
 \tkzDefLine[bisector](B,A,C)  \tkzGetPoint{a}
 \tkzDefLine[bisector](C,B,A)  \tkzGetPoint{b}
 \tkzShowLine[bisector,size=2,gap=3](B,A,C)
 \tkzShowLine[bisector,size=1,gap=3](C,B,A)   
 \tkzInterLL(A,a)(B,b) \tkzGetPoint{I}
 \tkzDefPointBy[projection = onto A--B](I) 
 \tkzDrawCircle[radius,color=red,%
 line width=.2pt](I,tkzPointResult) 
 \tkzDrawSegments[color=Maroon!50](I,tkzPointResult)
 \tkzDrawLines[add=0 and 5,color=Maroon!50](A,a B,b) 
\end{tikzpicture}

\end{tkzexample}

\subsubsection{Exemple de \tkzcname{tkzShowLine} et \tkzname{mediator}} 
\begin{tkzexample}[latex=6 cm]
\begin{tikzpicture}
 \tkzInit[xmax=6,ymax=7]
 \tkzGrid
 \tkzDefPoint(2,2){A} 
 \tkzDefPoint(5,4){B}  
 \tkzDrawPoints(A,B)    
 \tkzShowLine[mediator,color=orange,length=1](A,B)
 \tkzGetPoints{i}{j}
 \tkzLabelPoints[below =3pt](A,B)
 \tkzDrawLines[](A,B i,j) 
\end{tikzpicture}
\end{tkzexample}
\endinput
%!TEX root = /Users/ego/Boulot/TKZ/tkz-euclide/doc_fr/TKZdoc-euclide-main.tex

\section{Les segments}

Il existe bien sûr, une macro pour tracer simplement un segment (il serait possible comme pour une demi-droite, de créer un style avec \tkzcname{add}) .

\subsection{Tracer un segment \tkzcname{tkzDrawSegment}} 
 \hypertarget{tds}{}      

 \begin{NewMacroBox}{tkzDrawSegment}{\oarg{local options}\parg{pt1,pt2}}
\emph{Les arguments sont une liste de deux points. Les styles de \TIKZ\ sont accessibles pour les tracés}
 
\medskip
\begin{tabular}{lll}
argument    & exemple & définition    \\
\midrule
\TAline{(pt1,pt2)}{(A,B)}{trace le segment $[A,B]$}
\bottomrule 
\end{tabular}

C'est bien sûr équivalent à \tkzcname{draw (A)--(B);} 
\end{NewMacroBox}

\subsubsection{Exemple avec des références de points}     

\begin{tkzexample}[latex=6cm]
\begin{tikzpicture}[scale=1.5]
  \tkzInit[xmin=-1,xmax=3,ymin=-1,ymax=2]
  \tkzClip
  \tkzDefPoint(0,0){A}
  \tkzDefPoint(2,1){B}
  \tkzDrawSegment[color=red,thin](A,B)
  \tkzDrawPoints(A,B)    
  \tkzLabelPoints(A,B)  
\end{tikzpicture}
\end{tkzexample}
  

\subsubsection{Exemple avec des références de points} 
 Il est préférable de référencer les points, car les points sont
 placées en tenant compte de  \tkzcname{tkzInit}.
 
\begin{tkzexample}[latex=6cm]
\begin{tikzpicture}[scale=1.5]
  \tkzInit[xmin=-1,xmax=3,ymin=-1,ymax=2]
  \tkzClip
  \tkzDrawSegment[color=red,thin]({0,0},{2,1})  
\end{tikzpicture}
\end{tkzexample} 

\bigskip
Si les options sont les mêmes on peut tracer plusieurs \hypertarget{segs}{segments} avec la même macro. 
 
\newpage
\subsection{Tracer des segments \tkzcname{tkzDrawSegments}} 
 \hypertarget{tdss}{}      

 \begin{NewMacroBox}{tkzDrawSegments}{\oarg{local options}\parg{pt1,pt2 pt3,pt4 ...}}
\emph{Les arguments sont une liste de couple de deux points. Les styles de \TIKZ\ sont accessibles pour les tracés}
\end{NewMacroBox}

\begin{center}
\begin{tkzexample}[latex=6cm]
\begin{tikzpicture}
  \tkzInit[xmin=-1,xmax=3,ymin=-1,ymax=2]
  \tkzClip[space=1]
  \tkzDefPoint(0,0){A}
  \tkzDefPoint(2,1){B} 
  \tkzDefPoint(3,0){C} 
  \tkzDrawSegments(A,B B,C)
  \tkzDrawPoints(A,B,C)    
  \tkzLabelPoints(A,C) 
  \tkzLabelPoints[above](B)  
\end{tikzpicture}
\end{tkzexample}
\end{center} 

\subsection{Marquer un segment \tkzcname{tkzMarkSegment}}
\hypertarget{tms}{}  
  
 \begin{NewMacroBox}{tkzMarkSegment}{\oarg{local options}\parg{pt1,pt2}} 
\emph{La macro permet de placer une marque sur un segment.}

\medskip
\begin{tabular}{lll}
\toprule
options             & défaut & définition    \\
\midrule
\TOline{pos}{.5}{position de la marque} 
\TOline{color}{black}{couleur de la marque} 
\TOline{mark}{none}{choix de la marque} 
\TOline{size}{4pt}{taille de la marque} 
\bottomrule
\end{tabular}

\emph{Les marques possibles sont celles fournies par \TIKZ, mais d'autres marques ont été crées d'après une idée de Yves Combe.}
\end{NewMacroBox} 

\subsubsection{Marques multiples}
\begin{tkzexample}[latex=6cm,small] 
\begin{tikzpicture}
  \tkzDefPoint(2,1){A}
  \tkzDefPoint(6,4){B}
  \tkzDrawSegment(A,B)
  \tkzMarkSegment[color=Maroon,size=2pt,
        pos=0.4, mark=z](A,B) 
  \tkzMarkSegment[color=blue,
        pos=0.2, mark=oo](A,B)
  \tkzMarkSegment[pos=0.8,
        mark=s,color=red](A,B) 
\end{tikzpicture}
\end{tkzexample}

\subsubsection{Utilisation de \tkzname{mark}}      
\begin{tkzexample}[latex=6cm,small] 
\begin{tikzpicture}
  \tkzDefPoint(2,1){A} 
  \tkzDefPoint(6,4){B}
  \tkzDrawSegment(A,B)
  \tkzMarkSegment[color=gray,
                  pos=0.2,mark=s|](A,B)
  \tkzMarkSegment[color=gray,
                  pos=0.4,mark=s||](A,B)
  \tkzMarkSegment[color=Maroon,
                  pos=0.6,mark=||](A,B)
  \tkzMarkSegment[color=red,
                  pos=0.8,mark=|||](A,B)
\end{tikzpicture}
\end{tkzexample}


\subsection{Marquer des segments \tkzcname{tkzMarkSegments}}
\hypertarget{tmss}{} 
 
\begin{NewMacroBox}{tkzMarkSegments}{\oarg{local options}\parg{pt1,pt2 pt3,pt4 ...}}
\emph{Les arguments sont une liste de couple de deux points séparés par des espaces. Les styles de \TIKZ\ sont accessibles pour les tracés.}
\end{NewMacroBox} 

\subsubsection{Marques pour un triangle isocèle}      
\begin{tkzexample}[latex=6cm,small]
\begin{tikzpicture}[scale=1]
 \tkzDefPoints{0/0/O,2/2/A,4/0/B,6/2/C}
 \tkzDrawSegments(O,A A,B)
 \tkzDrawPoints(O,A,B)
 \tkzDrawLine(O,B)   
 \tkzMarkSegments[mark=||,size=6pt](O,A A,B)
\end{tikzpicture}
\end{tkzexample} 

\subsection{Exemple de rotation}   
\begin{center}
\begin{tkzexample}[latex=7cm,small] 
 \begin{tikzpicture}[scale=0.5]
  \tkzDefPoint(0,0){A}\tkzDefPoint(3,2){B} 
  \tkzDefPoint(4,0){C}\tkzDefPoint(2.5,1){P}
  \tkzDrawPolygon(A,B,C)
  \tkzDefEquilateral(A,P) \tkzGetPoint{P'}
  \tkzDefPointsBy[rotation=center A angle 60](P,B){P',C'}
  \tkzDrawPolygon(A,P,P')
  \tkzDrawPolySeg(P',C',A,P,B)
  \tkzDrawSegment(C,P)
  \tkzDrawPoints(A,B,C,C',P,P')
  \tkzMarkSegments[mark=s|,mark size=6pt,
  color=blue](A,P P,P' P',A) 
  \tkzMarkSegments[mark=||,color=orange](B,P P',C')
  \tkzLabelPoints(A,C) \tkzLabelPoints[below](P) 
  \tkzLabelPoints[above right](P',C',B) 
  
\end{tikzpicture} 
\end{tkzexample}  
\end{center}  
\newpage  
\hypertarget{tls}{}  
 \begin{NewMacroBox}{tkzLabelSegment}{\oarg{local options}\parg{pt1,pt2}\marg{label}}
\emph{Cette macro permet de placer une étiquette le long d'un segment ou encore d'une ligne. Les options sont celles de \TIKZ\ par exemple \tkzname{pos} } 

\medskip
\begin{tabular}{lll}
argument    & exemple & définition    \\
\midrule
\TAline{label}{\tkzcname{tkzLabelSegment(A,B)\{$5$\}}}{texte de l'étiquette} 
\TAline{(pt1,pt2)}{(A,B)}{étiquette le long de $[A,B]$} 
\bottomrule
\end{tabular}


\medskip
\begin{tabular}{lll}
options  & défaut & définition    \\
\midrule
\TOline{pos}{.5}{position du label} 
\end{tabular}
\end{NewMacroBox}  

 \subsubsection{Labels multiples}      
\begin{tkzexample}[latex=6 cm,small]
\begin{tikzpicture}
\tkzInit
\tkzDefPoint(0,0){A}
\tkzDefPoint(6,0){B}
\tkzDrawSegment(A,B)
\tkzLabelSegment[above,pos=.8](A,B){$a$}
\tkzLabelSegment[below,pos=.2](A,B){$4$}
\end{tikzpicture} 
\end{tkzexample}  

\subsubsection{Labels et Pythagore}
Cet exemple nécessite \tkzcname{usetkzobj{polygons}}
      
\begin{tkzexample}[latex=7cm]
\begin{tikzpicture}[scale=.75]
\tkzInit[xmax=5,ymax=5]
\tkzDefPoint(0,0){C}
\tkzDefPoint(4,0){A}
\tkzDefPoint(0,3){B}
\tkzDefSquare(B,A)\tkzGetPoints{E}{F}
\tkzDefSquare(A,C)\tkzGetPoints{G}{H}
\tkzDefSquare(C,B)\tkzGetPoints{I}{J}
\tkzFillPolygon[draw,
                fill = red!50 ](A,C,G,H)
\tkzFillPolygon[draw,
                fill = blue!50 ](C,B,I,J)
\tkzFillPolygon[draw,
                fill = purple!50](B,A,E,F)
\tkzFillPolygon[draw,opacity=.5,
                fill = orange](A,B,C)
\tkzDrawPolygon[line width = 1pt](A,B,C)
\tkzLabelSegment[above](C,A){$a$}
\tkzLabelSegment[right](B,C){$b$}
\tkzLabelSegment[below left](B,A){$c$}
\end{tikzpicture} 
\end{tkzexample}

\newpage 
\hypertarget{tlss}{} 
 \begin{NewMacroBox}{tkzLabelSegments}{\oarg{local options}\parg{pt1,pt2 pt3,pt4 ...}}
\emph{Les arguments sont une liste de couple de deux points. Les styles de \TIKZ\ sont accessibles pour les tracés.}
\end{NewMacroBox} 
 
\subsubsection{Labels pour un triangle isocèle}      
\begin{center}
\begin{tkzexample}[latex=6cm,small]
\begin{tikzpicture}[scale=1]
 \tkzDefPoints{0/0/O,2/2/A,4/0/B,6/2/C}
 \tkzDrawSegments(O,A A,B)
 \tkzDrawPoints(O,A,B)
 \tkzDrawLine(O,B)   
 \tkzLabelSegments[color=red,above=4pt](O,A A,B){$a$}
\end{tikzpicture}
\end{tkzexample}  
\end{center}   
\endinput



%!TEX root = /Users/ego/Boulot/TKZ/tkz-euclide/doc_fr/TKZdoc-euclide-main.tex


\section{Définition de points à l'aide d'un vecteur}

\subsection{\tkzcname{tkzDefPointWith}}
Il y a plusieurs possibilités pour créer des points qui répondent à certaines conditions vectorielles.
Cela peut se faire avec  \tkzcname{tkzDefPointWith}. Le principe général est le suivant, deux points sont passés en argument, autrement dit un vecteur. Les différentes options permettent d'obtenir  un nouveau point formant avec le premier point (sauf exception) un vecteur colinéaire  ou bien orthogonal au premier vecteur. Ensuite la longueur est soit proportionnelle à celle du premier, ou bien proportionnelle à l'unité. Dans la mesure ou ce point n'est utilisé que temporairement, il n'est pas obligé de le nommer immédiatement. Le résultat est dans \tkzcname{tkzPointResult}. La macro \tkzNameMacro{tkzGetPoint} permet de récupérer le point et de le nommer différemment.

\begin{NewMacroBox}{tkzDefPointWith}{\parg{pt1,pt2}}
 Il s'agit en fait de la définition d'un point  répondant à des conditions vectorielles.

\medskip
  
\begin{tabular}{lll}
\toprule
arguments             & définition & explication                         \\ 
\midrule
\TAline{(pt1,pt2)} {couple de points}{le résultat est un point dans \tkzcname{tkzPointResult} } \\                                                                         
 \bottomrule
\end{tabular}

\medskip
Dans ce qui suit, on suppose que le point est récupéré par \tkzNameMacro{tkzGetPoint\{C\}}

\begin{tabular}{lll}
\toprule
options             & exemple & explication                         \\ 
\midrule
\TOline{orthogonal}{[orthogonal](A,B)}{$AC=AB$ et $\overrightarrow{AC} \perp \overrightarrow{AB}$}
\TOline{orthogonal normed}{[orthogonal normed](A,B)}{$AC=1$ et $\overrightarrow{AC} \perp \overrightarrow{AB}$ } 
\TOline{linear}{[linear](A,B)}{ $\overrightarrow{AC}=K \times \overrightarrow{AB}$}
\TOline{linear normed}{[linear normed](A,B)}{$AC=K$ et $\overrightarrow{AC}=k\times \overrightarrow{AB}$ }  
\TOline{colinear= at \#1}{[colinear= at C](A,B)}{$\overrightarrow{CD}= \overrightarrow{AB}$ }
\TOline{K}{[linear](A,B),K=2}{$\overrightarrow{AC}=2\times \overrightarrow{AB}$}     
  \bottomrule
\end{tabular}

\medskip
\noindent\emph{Pour la linéarité, K est obligatoire. Sa valeur par défaut est égale à 1.}   


\end{NewMacroBox}

\subsubsection{\tkzcname{tkzDefPointWith} et \tkzname{orthogonal}} 
$K=-1$ c'est pour que $(\overrightarrow{AC},\overrightarrow{AB})$ détermine un angle positif. AB=AC puisque $|K|=1$
\begin{tkzexample}[latex=6cm]
\begin{tikzpicture}[scale=1.2]
   \tkzInit[xmax=5,ymax=4] \tkzGrid
   \tkzDefPoint(2,3){A}   \tkzDefPoint(4,2){B}
   \tkzDefPointWith[orthogonal,K=-1](A,B)
   \tkzGetPoint{C}
   \tkzDrawPoints[color=red](A,B,C)
   \tkzLabelPoints[above right=3pt](A,B,C)
\end{tikzpicture} 
\end{tkzexample}

\subsubsection{\tkzcname{tkzDefPointWith}  \tkzname{orthogonal normed}} 
AC=1

\begin{tkzexample}[latex=6cm]
\begin{tikzpicture}[scale=1.2]
   \tkzInit[ymin=1,xmax=5,ymax=5] \tkzGrid
   \tkzDefPoint(2,3){A}   \tkzDefPoint(4,2){B}
   \tkzDefPointWith[orthogonal normed](A,B)
   \tkzGetPoint{C}
   \tkzDrawPoints[color=red](A,B,C)
   \tkzLabelPoints[above right=3pt](A,B,C)
\end{tikzpicture} 
\end{tkzexample}

\subsubsection{\tkzcname{tkzDefPointWith} et  \tkzname{orthogonal normed}} 
$K=2$ donc AC=2.

\begin{tkzexample}[latex=6cm]
\begin{tikzpicture}[scale=1.2]
   \tkzInit[ymin=1,xmax=5,ymax=5] \tkzGrid
   \tkzDefPoint(2,3){A}   \tkzDefPoint(4,2){B}
   \tkzDefPointWith[orthogonal normed,K=2](A,B)
   \tkzGetPoint{C}
   \tkzDrawPoints[color=red](A,B,C)
   \tkzLabelPoints[above right=3pt](A,B,C)
\end{tikzpicture} 
\end{tkzexample}

\subsubsection{\tkzcname{tkzDefPointWith} et \tkzname{colinear}} 
$K=2$ donc AC=2.

\begin{tkzexample}[latex=6cm]
\begin{tikzpicture}[scale=1.2]
   \tkzInit[xmax=5,ymax=4] \tkzGrid
   \tkzDefPoint(2,3){A}   \tkzDefPoint(4,2){B}
   \tkzDefPoint(0,1){C}
   \tkzDefPointWith[colinear=at C](A,B)
   \tkzGetPoint{D}
   \tkzDrawPoints[color=red](A,B,C,D)
   \tkzLabelPoints[above right=3pt](A,B,C,D)
\end{tikzpicture} 
\end{tkzexample}

\subsubsection{\tkzcname{tkzDefPointWith}  \tkzname{linear} } 
 Ici $K=0.5$
Cela revient à appliquer une homothétie ou bien encore une multiplication d'un vecteur par un réel. C est ici le milieu de $[AB]$.

\begin{tkzexample}[latex=6cm]
\begin{tikzpicture}[scale=1.2]
   \tkzInit[ymin=1,xmax=5,ymax=4] \tkzGrid
   \tkzDefPoint(1,3){A}   \tkzDefPoint(4,2){B}
   \tkzDefPointWith[linear,K=0.5](A,B)
   \tkzGetPoint{C}
   \tkzDrawPoints[color=red](A,B,C)
   \tkzLabelPoints[above right=3pt](A,B,C)
\end{tikzpicture} 
\end{tkzexample}

\subsubsection{\tkzcname{tkzDefPointWith}  \tkzname{linear normed}}
Dans l'exemple suivant AC=1 et C appartient à $(AB)$.

\begin{tkzexample}[latex=6cm]
\begin{tikzpicture}[scale=1.2]
   \tkzInit[ymin=1,xmax=5,ymax=4] \tkzGrid
   \tkzDefPoint(1,3){A}   \tkzDefPoint(4,2){B}
   \tkzDefPointWith[linear normed](A,B)
   \tkzGetPoint{C}
   \tkzDrawPoints[color=red](A,B,C)
   \tkzLabelPoints[above right=3pt](A,B,C)
\end{tikzpicture} 
\end{tkzexample}
\endinput
  
\section{Definition of polygons}

\subsection{Defining the points of a square} \label{def_square}

We have seen the definitions of some triangles. Let us look at the definitions
of some quadrilaterals and regular polygons.

\begin{NewMacroBox}{tkzDefSquare}{\parg{pt1,pt2}}%
The square is defined in the forward direction. From two points, two more points
are obtained such that the four taken in order form a square. The square is
defined in the forward direction.    The results are in
\tkzname{tkzFirstPointResult} and \tkzname{tkzSecondPointResult}.\\
We can rename them with \tkzcname{tkzGetPoints}.

\medskip
\begin{tabular}{lll}%
\toprule
Arguments             & example & explication                         \\
\midrule
\TAline{\parg{pt1,pt2}}{\tkzcname{tkzDefSquare}\parg{A,B}}{The square is defined
in the direct direction.}
\end{tabular}
\end{NewMacroBox}

\subsubsection{Using \tkzcname{tkzDefSquare} with two points}

Note the inversion of the first two points and the result.

\begin{tkzexample}[latex=4cm,small]
\begin{tikzpicture}[scale=.5]
  \tkzDefPoint(0,0){A} \tkzDefPoint(3,0){B}
  \tkzDefSquare(A,B)
  \tkzDrawPolygon[color=red](A,B,tkzFirstPointResult,%
    tkzSecondPointResult)
  \tkzDefSquare(B,A)
  \tkzDrawPolygon[color=blue](B,A,tkzFirstPointResult,%
    tkzSecondPointResult)
\end{tikzpicture}
\end{tkzexample}

We may only need one point to draw an isosceles right-angled triangle so we use
\tkzcname{tkzGetFirstPoint} or \tkzcname{tkzGetSecondPoint}.

\subsubsection{Use of \tkzcname{tkzDefSquare} to obtain an isosceles right-angled triangle}

\begin{tkzexample}[latex=7cm,small]
\begin{tikzpicture}[scale=1]
  \tkzDefPoint(0,0){A}
  \tkzDefPoint(3,0){B}
  \tkzDefSquare(A,B) \tkzGetFirstPoint{C}
  \tkzDrawPolygon[color=blue,fill=blue!30](A,B,C)
\end{tikzpicture}
\end{tkzexample}

\newpage

\subsubsection{Pythagorean Theorem and \tkzcname{tkzDefSquare}}

\begin{tkzexample}[latex=8cm,small]
\begin{tikzpicture}[scale=.75]
  \tkzInit
  \tkzDefPoint(0,0){C}
  \tkzDefPoint(4,0){A}
  \tkzDefPoint(0,3){B}
  \tkzDefSquare(B,A)\tkzGetPoints{E}{F}
  \tkzDefSquare(A,C)\tkzGetPoints{G}{H}
  \tkzDefSquare(C,B)\tkzGetPoints{I}{J}
  \tkzFillPolygon[fill = red!50 ](A,C,G,H)
  \tkzFillPolygon[fill = blue!50 ](C,B,I,J)
  \tkzFillPolygon[fill = purple!50](B,A,E,F)
  \tkzFillPolygon[fill = orange,opacity=.5](A,B,C)
  \tkzDrawPolygon[line width = 1pt](A,B,C)
  \tkzDrawPolygon[line width = 1pt](A,C,G,H)
  \tkzDrawPolygon[line width = 1pt](C,B,I,J)
  \tkzDrawPolygon[line width = 1pt](B,A,E,F)
  \tkzLabelSegment[](A,C){$a$}
  \tkzLabelSegment[](C,B){$b$}
  \tkzLabelSegment[swap](A,B){$c$}
\end{tikzpicture}
\end{tkzexample}

\subsection{Definition of parallelogram}

\subsection{Defining the points of a parallelogram}

It is a matter of completing three points in order to obtain a parallelogram.
\begin{NewMacroBox}{tkzDefParallelogram}{\parg{pt1,pt2,pt3}}%
From three points, another point is obtained such that the four taken in order
form a parallelogram.  The result is in \tkzname{tkzPointResult}.\par
We can rename it with the name \tkzcname{tkzGetPoint}\dots

\begin{tabular}{lll}%
\toprule
arguments &  default & definition  \\
\midrule
\TAline{\parg{pt1,pt2,pt3}}{no default}{Three points are necessary}
\bottomrule
\end{tabular}
\end{NewMacroBox}

\subsubsection{Example of a parallelogram definition}

\begin{tkzexample}[latex=7 cm,small]
\begin{tikzpicture}[scale=1]
  \tkzDefPoints{0/0/A,3/0/B,4/2/C}
  \tkzDefParallelogram(A,B,C)
  \tkzGetPoint{D}
  \tkzDrawPolygon(A,B,C,D)
  \tkzLabelPoints(A,B)
  \tkzLabelPoints[above right](C,D)
  \tkzDrawPoints(A,...,D)
\end{tikzpicture}
\end{tkzexample}

\newpage

\subsubsection{Simple example}

Explanation of the definition of a parallelogram

\begin{tkzexample}[latex=7 cm,small]
\begin{tikzpicture}[scale=1]
  \tkzDefPoints{0/0/A,3/0/B,4/2/C}
  \tkzDefPointWith[colinear= at C](B,A)
  \tkzGetPoint{D}
  \tkzDrawPolygon(A,B,C,D)
  \tkzLabelPoints(A,B)
  \tkzLabelPoints[above right](C,D)
  \tkzDrawPoints(A,...,D)
\end{tikzpicture}
\end{tkzexample}

\subsubsection{Construction of the golden rectangle}

\begin{tkzexample}[latex=8cm,small]
\begin{tikzpicture}[scale=.5]
  \tkzInit[xmax=14,ymax=10]
  \tkzClip[space=1]
  \tkzDefPoint(0,0){A}
  \tkzDefPoint(8,0){B}
  \tkzDefMidPoint(A,B)\tkzGetPoint{I}
  \tkzDefSquare(A,B)\tkzGetPoints{C}{D}
  \tkzDrawSquare(A,B)
  \tkzInterLC(A,B)(I,C)\tkzGetPoints{G}{E}
  \tkzDrawArc[style=dashed,color=gray](I,E)(D)
  \tkzDefPointWith[colinear= at C](E,B)
  \tkzGetPoint{F}
  \tkzDrawPoints(C,D,E,F)
  \tkzLabelPoints(A,B,C,D,E,F)
  \tkzDrawSegments[style=dashed,color=gray]%
(E,F C,F B,E)
\end{tikzpicture}
\end{tkzexample}

\subsection{Drawing a square}

\begin{NewMacroBox}{tkzDrawSquare}{\oarg{local options}\parg{pt1,pt2}}%
The macro draws a square but not the vertices. It is possible to color the
inside. The order of the points is that of the direct direction of the
trigonometric circle.

\medskip
\begin{tabular}{lll}%
\toprule
arguments             & example & explication                         \\
\midrule
\TAline{\parg{pt1,pt2}}{|\tkzcname{tkzDrawSquare}|\parg{A,B}}{|\tkzcname{tkzGetPoints\{C\}\{D\}}|}
\bottomrule
\end{tabular}

\medskip
\begin{tabular}{lll}%
options             & example & explication                         \\
\midrule
\TOline{Options TikZ}{|red,line width=1pt|}{}
\end{tabular}
\end{NewMacroBox}

\newpage

\subsubsection{The idea is to inscribe two squares in a semi-circle}

\begin{tkzexample}[latex=7cm,small]
\begin{tikzpicture}[scale=.75]
  \tkzInit[ymax=8,xmax=8]
  \tkzClip[space=.25]   \tkzDefPoint(0,0){A}
  \tkzDefPoint(8,0){B}  \tkzDefPoint(4,0){I}
  \tkzDefSquare(A,B)    \tkzGetPoints{C}{D}
  \tkzInterLC(I,C)(I,B) \tkzGetPoints{E'}{E}
  \tkzInterLC(I,D)(I,B) \tkzGetPoints{F'}{F}
  \tkzDefPointsBy[projection=onto A--B](E,F){H,G}
  \tkzDefPointsBy[symmetry  = center H](I){J}
  \tkzDefSquare(H,J)    \tkzGetPoints{K}{L}
  \tkzDrawSector[fill=yellow](I,B)(A)
  \tkzFillPolygon[color=red!40](H,E,F,G)
  \tkzFillPolygon[color=blue!40](H,J,K,L)
  \tkzDrawPolySeg[color=red](H,E,F,G)
  \tkzDrawPolySeg[color=red](J,K,L)
  \tkzDrawPoints(E,G,H,F,J,K,L)
\end{tikzpicture}
\end{tkzexample}

\subsection{The golden rectangle}

\begin{NewMacroBox}{tkzDefGoldRectangle}{\parg{point,point}}%

The macro determines a rectangle whose size ratio is the number $\Phi$. The
created points are in \tkzname{tkzFirstPointResult} and \tkzname{tkzSecondPointResult}.
They can be obtained with the macro \tkzcname{tkzGetPoints}. The following
macro is used to draw the rectangle.

\begin{tabular}{lll}%
\toprule
arguments             & example & explication                         \\
\midrule
\TAline{\parg{pt1,pt2}}{\parg{A,B}}{If C and D are created then $AB/BC=\Phi$.}
 \end{tabular}
\end{NewMacroBox}

\begin{NewMacroBox}{tkzDrawGoldRectangle}{\oarg{local
options}\parg{point,point}}
\begin{tabular}{lll}%
arguments             & example & explication                         \\
\midrule
\TAline{\parg{pt1,pt2}}{\parg{A,B}}{Draws the golden rectangle based on the
segment $[AB]$}
\end{tabular}

\medskip
\begin{tabular}{lll}%
options     & example & explication     \\
\midrule
\TOline{Options TikZ}{|red,line width=1pt|}{}
\end{tabular}
\end{NewMacroBox}

\subsubsection{Golden Rectangles}
\begin{tkzexample}[latex=6 cm,small]
\begin{tikzpicture}[scale=.6]
  \tkzDefPoint(0,0){A}      \tkzDefPoint(8,0){B}
  \tkzDefGoldRectangle(A,B) \tkzGetPoints{C}{D}
  \tkzDefGoldRectangle(B,C) \tkzGetPoints{E}{F}
  \tkzDrawPolygon[color=red,fill=red!20](A,B,C,D)
  \tkzDrawPolygon[color=blue,fill=blue!20](B,C,E,F)
\end{tikzpicture}
\end{tkzexample}

\newpage

\subsection{Drawing a polygon}

\begin{NewMacroBox}{tkzDrawPolygon}{\oarg{local options}\parg{points list}}%
Just give a list of points and the macro plots the polygon using the \TIKZ\
options present. You can  replace $(A,B,C,D,E)$ by $(A,\dots,E)$ and
$(P_1,P_2,P_3,P_4,P_5)$ by $(P_1,P\dots,P_5)$

\begin{tabular}{lll}%
\toprule
arguments             & example & explication                         \\
\midrule
\TAline{\parg{pt1,pt2,pt3,\dots}}{|\BS
tkzDrawPolygon[gray,dashed](A,B,C)|}{Drawing a triangle}
\end{tabular}

\medskip
\begin{tabular}{lll}%
\toprule
options             & default & example                         \\
\midrule
\TOline{Options TikZ}{\dots}{|\BS tkzDrawPolygon[red,line width=2pt](A,B,C)|}
\end{tabular}
\end{NewMacroBox}

\subsubsection{\tkzcname{tkzDrawPolygon}}

\begin{tkzexample}[latex=7cm, small]
\begin{tikzpicture}[rotate=18,scale=1.5]
  \tkzDefPoint(0,0){A}
  \tkzDefPoint(2.25,0.2){B}
  \tkzDefPoint(2.5,2.75){C}
  \tkzDefPoint(-0.75,2){D}
  \tkzDrawPolygon[fill=black!50!blue!20!](A,B,C,D)
  \tkzDrawSegments[style=dashed](A,C B,D)
\end{tikzpicture}
\end{tkzexample}

\subsection{Drawing a polygonal chain}

\begin{NewMacroBox}{tkzDrawPolySeg}{\oarg{local options}\parg{points list}}%
Just give a list of points and the macro plots the polygonal chain using the
\TIKZ\ options present.

\begin{tabular}{lll}%
\toprule
arguments             & example & explication                         \\
\midrule
\TAline{\parg{pt1,pt2,pt3,\dots}}{|\BS
tkzDrawPolySeg[gray,dashed](A,B,C)|}{Drawing a triangle}
\end{tabular}

\medskip
\begin{tabular}{lll}%
\toprule
options             & default & example                         \\
\midrule
\TOline{Options TikZ}{\dots}{|\BS tkzDrawPolySeg[red,line width=2pt](A,B,C)|}
\end{tabular}
\end{NewMacroBox}

\newpage

\subsubsection{Polygonal chain}

\begin{tkzexample}[latex=7cm, small]
\begin{tikzpicture}
  \tkzDefPoints{0/0/A,6/0/B,3/4/C,2/2/D}
  \tkzDrawPolySeg(A,...,D)
  \tkzDrawPoints(A,...,D)
\end{tikzpicture}
\end{tkzexample}

\subsubsection{Polygonal chain: index notation}

\begin{tkzexample}[latex=7cm, small]
\begin{tikzpicture}
  \foreach \pt in {1,2,...,8} {%
    \tkzDefPoint(\pt*20:3){P_\pt}}
  \tkzDrawPolySeg(P_1,P_...,P_8)
  \tkzDrawPoints(P_1,P_...,P_8)
\end{tikzpicture}
\end{tkzexample}

\subsection{Clip a polygon}

\begin{NewMacroBox}{tkzClipPolygon}{\oarg{local options}\parg{points list}}%
This macro makes it possible to contain the different plots in the designated
polygon.

\medskip
\begin{tabular}{lll}%
\toprule
arguments       & example & explication     \\
\midrule
\TAline{\parg{pt1,pt2}}{\parg{A,B}}{}
%\bottomrule
\end{tabular}
\end{NewMacroBox}

\subsubsection{\tkzcname{tkzClipPolygon}}

\begin{tkzexample}[latex=7 cm,small]
\begin{tikzpicture}[scale=1.25]
  \tkzInit[xmin=0,xmax=4,ymin=0,ymax=3]
  \tkzClip[space=.5]
  \tkzDefPoint(0,0){A}
  \tkzDefPoint(4,0){B}
  \tkzDefPoint(1,3){C}
  \tkzDrawPolygon(A,B,C)
  \tkzDefPoint(0,2){D}
  \tkzDefPoint(2,0){E}
  \tkzDrawPoints(D,E)
  \tkzLabelPoints(D,E)
  \tkzClipPolygon(A,B,C)
  \tkzDrawLine[color=red](D,E)
\end{tikzpicture}
\end{tkzexample}

\newpage

\subsubsection{Example: use of \enquote{Clip} for Sangaku in a square}

\begin{tkzexample}[latex=7cm, small]
\begin{tikzpicture}[scale=.75]
  \tkzDefPoint(0,0){A} \tkzDefPoint(8,0){B}
  \tkzDefSquare(A,B) \tkzGetPoints{C}{D}
  \tkzDrawPolygon(B,C,D,A)
  \tkzClipPolygon(B,C,D,A)
  \tkzDefPoint(4,8){F}
  \tkzDefTriangle[equilateral](C,D)
  \tkzGetPoint{I}
  \tkzDrawPoint(I)
  \tkzDefPointBy[projection=onto B--C](I)
  \tkzGetPoint{J}
  \tkzInterLL(D,B)(I,J)  \tkzGetPoint{K}
  \tkzDefPointBy[symmetry=center K](B)
  \tkzGetPoint{M}
  \tkzDrawCircle(M,I)
  \tkzCalcLength(M,I)   \tkzGetLength{dMI}
  \tkzFillPolygon[color = orange](A,B,C,D)
  \tkzFillCircle[R,color = yellow](M,\dMI pt)
  \tkzFillCircle[R,color = blue!50!black](F,4 cm)%
\end{tikzpicture}
\end{tkzexample}

\subsection{Color a polygon}

\begin{NewMacroBox}{tkzFillPolygon}{\oarg{local options}\parg{points list}}%
You can color by drawing the polygon, but in this case you color the inside of
the polygon without drawing it.

\medskip
\begin{tabular}{lll}%
\toprule
arguments                & example & explication                         \\
\midrule
\TAline{\parg{pt1,pt2,\dots}}{\parg{A,B,\dots}}{}
%\bottomrule
\end{tabular}
\end{NewMacroBox}

\subsubsection{\tkzcname{tkzFillPolygon}}

\begin{tkzexample}[latex=7cm, small]
\begin{tikzpicture}[scale=0.7]
  \tkzInit[xmin=-3,xmax=6,ymin=-1,ymax=6]
  \tkzDrawX[noticks]
  \tkzDrawY[noticks]
  \tkzDefPoint(0,0){O}  \tkzDefPoint(4,2){A}
  \tkzDefPoint(-2,6){B}
  \tkzPointShowCoord[xlabel=$x$,ylabel=$y$](A)
  \tkzPointShowCoord[xlabel=$x'$,ylabel=$y'$,%
                     ystyle={right=2pt}](B)
  \tkzDrawSegments[->](O,A O,B)
  \tkzLabelSegment[above=3pt](O,A){$\vec{u}$}
  \tkzLabelSegment[above=3pt](O,B){$\vec{v}$}
  \tkzMarkAngle[fill= yellow,size=1.8cm,%
                opacity=.5](A,O,B)
  \tkzFillPolygon[red!30,opacity=0.25](A,B,O)
  \tkzLabelAngle[pos = 1.5](A,O,B){$\alpha$}
\end{tikzpicture}
\end{tkzexample}

\newpage

\subsection{Regular polygon}

\begin{NewMacroBox}{tkzDefRegPolygon}{\oarg{local options}\parg{pt1,pt2}}%
From the number of sides, depending on the options, this macro determines a
regular polygon according to its center or one side.

\begin{tabular}{lll}%
\toprule
arguments             & example & explication                         \\
\midrule
\TAline{\parg{pt1,pt2}}{\parg{O,A}}{with option \enquote{center}, $O$ is the center of
the polygon.}
\TAline{\parg{pt1,pt2}}{\parg{A,B}}{with option \enquote{side}, $[AB]$ is a side.}
\end{tabular}

\medskip
\begin{tabular}{lll}%
\toprule
options             & default & example                         \\
\midrule
\TOline{name}{P}{The vertices are named $P1$, $P2$, \dots}
\TOline{sides}{5}{number of sides.}
\TOline{center}{center}{The first point is the center.}
\TOline{side}{center}{The two points are vertices.}
\TOline{Options TikZ}{\dots}{}
\end{tabular}
\end{NewMacroBox}

\subsubsection{Option \tkzname{center}}

\begin{tkzexample}[latex=6cm, small]
\begin{tikzpicture}[scale=1.25]
  \tkzDefPoints{0/0/P0,0/0/Q0,2/0/P1}
  \tkzDefMidPoint(P0,P1) \tkzGetPoint{Q1}
  \tkzDefRegPolygon[center,sides=7](P0,P1)
  \tkzDefMidPoint(P1,P2) \tkzGetPoint{Q1}
  \tkzDefRegPolygon[center,sides=7,name=Q](P0,Q1)
  \tkzDrawPolygon(P1,P...,P7)
  \tkzFillPolygon[gray!20](Q0,Q1,P2,Q2)
  \foreach \j in {1,...,7} {
    \tkzDrawSegment[black](P0,Q\j)}
\end{tikzpicture}
\end{tkzexample}

\subsubsection{Option \tkzname{side}}

\begin{tkzexample}[latex=6cm, small]
\begin{tikzpicture}[scale=1]
  \tkzDefPoints{-4/0/A, -1/0/B}
  \tkzDefRegPolygon[side,sides=5,name=P](A,B)
  \tkzDrawPolygon[thick](P1,P...,P5)
\end{tikzpicture}
\end{tkzexample}

\endinput
  
\section{The Circles}

Among the following macros, one will allow you to draw a circle, which is not a
real feat. To do this, you will need to know the center of the circle and either
the radius of the circle or a point on the circumference. It seemed to me that
the most frequent use was to draw a circle with a given centre passing through a
given point. This will be the default method, otherwise you will have to use the
\tkzname{R} option. There are a large number of special circles, for example the
circle circumscribed by a triangle.

\begin{itemize}
\item  I have created a first macro \tkzcname{tkzDefCircle} which allows,
according to a particular circle, to retrieve its center and the measurement of
the radius in cm. This recovery is done with the macros \tkzcname{tkzGetPoint}
and \tkzcname{tkzGetLength};

\item then a macro \tkzcname{tkzDrawCircle};

\item then a macro that allows you to color in a disc, but without drawing the
circle \tkzcname{tkzFillCircle};

\item sometimes, it is necessary for a drawing to be contained in a disk, this
is the role assigned to \tkzcname{tkzClipCircle};

\item it finally remains to be able to give a label to designate a circle and
if several possibilities are offered, we will see here \tkzcname{tkzLabelCircle}.
\end{itemize}

\subsection{Characteristics of a circle: \tkzcname{tkzDefCircle}}

This macro allows you to retrieve the characteristics (center and radius) of
certain circles.

\begin{NewMacroBox}{tkzDefCircle}{\oarg{local options}\parg{A,B} or \parg{A,B,C}}%
\tkzHandBomb{}Attention the arguments are lists of two or three points. This
macro is either used in partnership with \tkzcname{tkzGetPoint} and/or
\tkzcname{tkzGetLength} to obtain the center and the radius of the circle, or by
using \tkzname{tkzPointResult} and \tkzname{tkzLengthResult} if it is not
necessary to keep the results.

\medskip
\begin{tabular}{lll}%
\toprule
arguments           & example & explication                         \\
\midrule
\TAline{\parg{pt1,pt2} or \parg{pt1,pt2,pt3}}{\parg{A,B}} {$[AB]$ is radius $A$
is the center}
\bottomrule
\end{tabular}

\medskip
\begin{tabular}{lll}%
\toprule
options             & default & definition                         \\
\midrule
\TOline{through}      {through}{circle characterized by two points defining a
radius}
\TOline{diameter}     {through}{circle characterized by two points defining a
diameter}
\TOline{circum}       {through}{circle circumscribed of a triangle}
\TOline{in}           {through}{incircle a triangle}
\TOline{ex}           {through}{excircle of a  triangle}
\TOline{euler or nine}{through}{Euler's Circle}
\TOline{spieker}      {through}{Spieker Circle}
\TOline{apollonius}   {through}{circle of Apollonius}
\TOline{orthogonal}   {through}{circle of given centre orthogonal to another
circle}
\TOline{orthogonal through}{through}{circle orthogonal circle passing through 2
points}
\TOline{K} {1}{coefficient used for a circle of Apollonius}
\bottomrule
\end{tabular}

{In the following examples, I draw the circles with a macro not yet presented,
but this is not necessary. In some cases you may only need the center or the
radius.}
\end{NewMacroBox}

\newpage

\subsubsection{Example with a random point and  option \tkzname{through}}

\begin{tkzexample}[latex=7 cm,small]
\begin{tikzpicture}[scale=1]
  \tkzDefPoint(0,4){A}
  \tkzDefPoint(2,2){B}
  \tkzDefMidPoint(A,B) \tkzGetPoint{I}
  \tkzDefRandPointOn[segment = I--B]
  \tkzGetPoint{C}
  \tkzDefCircle[through](A,C)
  \tkzGetLength{rACpt}
  \tkzpttocm(\rACpt){rACcm}
  \tkzDrawCircle(A,C)
  \tkzDrawPoints(A,B,C)
  \tkzLabelPoints(A,B,C)
  \tkzLabelCircle[draw,fill=orange, text width=3cm,
    text centered, font=\scriptsize](A,C)(-90)%
    {The radius measurement is:  \rACpt pt i.e. \rACcm cm}
\end{tikzpicture}
\end{tkzexample}

\subsubsection{Example with  option \tkzname{diameter}}

It is simpler here to search directly for the middle of $[AB]$.

\begin{tkzexample}[latex=7cm,small]
\begin{tikzpicture}[scale=1.25]
  \tkzDefPoint(0,0){A}
  \tkzDefPoint(2,2){B}
  \tkzDefCircle[diameter](A,B)
  \tkzGetPoint{O}
  \tkzDrawCircle[blue,fill=blue!20](O,B)
  \tkzDrawSegment(A,B)
  \tkzDrawPoints(A,B,O)
  \tkzLabelPoints(A,B,O)
\end{tikzpicture}
\end{tkzexample}

\subsubsection{Circles inscribed and circumscribed for a given triangle}

You can also obtain the center of the inscribed circle and its projection on
one side of the triangle with \tkzcname{tkzGetFirstPoint{I}} and
\tkzcname{tkzGetSecondPoint{Ib}}.

\begin{tkzexample}[latex=7cm,small]
\begin{tikzpicture}[scale=1]
  \tkzDefPoint(2,2){A}
  \tkzDefPoint(5,-2){B}
  \tkzDefPoint(1,-2){C}
  \tkzDefCircle[in](A,B,C)
  \tkzGetPoint{I} \tkzGetLength{rIN}
  \tkzDefCircle[circum](A,B,C)
  \tkzGetPoint{K} \tkzGetLength{rCI}
  \tkzDrawPoints(A,B,C,I,K)
  \tkzDrawCircle[R,blue](I,\rIN pt)
  \tkzDrawCircle[R,red](K,\rCI pt)
  \tkzLabelPoints[below](B,C)
  \tkzLabelPoints[above left](A,I,K)
  \tkzDrawPolygon(A,B,C)
\end{tikzpicture}
\end{tkzexample}

\newpage

\subsubsection{Example with option \tkzname{ex}}

We want to define an excircle of a  triangle relatively to point $C$

\begin{tkzexample}[latex=8cm,small]
\begin{tikzpicture}[scale=.65]
  \tkzDefPoints{ 0/0/A,4/0/B,0.8/4/C}
  \tkzDefCircle[ex](B,C,A)
  \tkzGetPoint{J_c} \tkzGetLength{rc}
  \tkzDefPointBy[projection=onto A--C ](J_c)
  \tkzGetPoint{X_c}
  \tkzDefPointBy[projection=onto A--B ](J_c)
  \tkzGetPoint{Y_c}
  \tkzGetPoint{I}
  \tkzDrawPolygon[color=blue](A,B,C)
  \tkzDrawCircle[R,color=lightgray](J_c,\rc pt)
  % possible  \tkzDrawCircle[ex](A,B,C)
  \tkzDrawCircle[in,color=red](A,B,C)    \tkzGetPoint{I}
  \tkzDefPointBy[projection=onto A--C ](I)
  \tkzGetPoint{F}
  \tkzDefPointBy[projection=onto A--B ](I)
  \tkzGetPoint{D}
  \tkzDrawLines[add=0 and 2.2,dashed](C,A C,B)
  \tkzDrawSegments[dashed](J_c,X_c I,D  I,F J_c,Y_c)
  \tkzMarkRightAngles(A,F,I B,D,I J_c,X_c,A J_c,Y_c,B)
  \tkzDrawPoints(B,C,A,I,D,F,X_c,J_c,Y_c)
  \tkzLabelPoints(B,A,J_c,I,D,X_c,Y_c)
  \tkzLabelPoints[above left](C)
  \tkzLabelPoints[left](F)
\end{tikzpicture}
\end{tkzexample}

\subsubsection{Euler's circle for a given triangle with option \tkzname{euler}}

We verify that this circle passes through the middle of each side.

\begin{tkzexample}[vbox,small]
\begin{tikzpicture}[scale=.95]
  \tkzDefPoint(5,3.5){A}
  \tkzDefPoint(0,0){B}
  \tkzDefPoint(7,0){C}
  \tkzDefCircle[euler](A,B,C)
  \tkzGetPoint{E}
  \tkzGetLength{rEuler}
  \tkzDefSpcTriangle[medial](A,B,C){M_a,M_b,M_c}
  \tkzDrawPoints(A,B,C,E,M_a,M_b,M_c)
  \tkzDrawCircle[R,blue](E,\rEuler pt)
  \tkzDrawPolygon(A,B,C)
  \tkzLabelPoints[below](B,C)
  \tkzLabelPoints[left](A,E)
\end{tikzpicture}
\end{tkzexample}

\newpage

\subsubsection{Apollonius circles for a given segment option \tkzname{apollonius}}

\begin{tkzexample}[latex=8cm,small]
\begin{tikzpicture}[scale=0.75]
  \tkzDefPoint(0,0){A}
  \tkzDefPoint(4,0){B}
  \tkzDefCircle[apollonius,K=2](A,B)
  \tkzGetPoint{K1}
  \tkzGetLength{rAp}
  \tkzDrawCircle[R,color = blue!50!black,
    fill=blue!20,opacity=.4](K1,\rAp pt)
  \tkzDefCircle[apollonius,K=3](A,B)
  \tkzGetPoint{K2}   \tkzGetLength{rAp}
  \tkzDrawCircle[R,color=red!50!black,
    fill=red!20,opacity=.4](K2,\rAp pt)
  \tkzLabelPoints[below](A,B,K1,K2)
  \tkzDrawPoints(A,B,K1,K2)
  \tkzDrawLine[add=.2 and 1](A,B)
\end{tikzpicture}
\end{tkzexample}

\subsubsection{Circles exinscribed to a given triangle option \tkzname{ex}}

You can also get the center and the projection of it on one side of the
triangle.

with \tkzcname{tkzGetFirstPoint\{Jb\}} and \tkzcname{tkzGetSecondPoint\{Tb\}}.

\begin{tkzexample}[latex=8cm,small]
\begin{tikzpicture}[scale=.6]
  \tkzDefPoint(0,0){A}
  \tkzDefPoint(3,0){B}
  \tkzDefPoint(1,2.5){C}
  \tkzDefCircle[ex](A,B,C) \tkzGetPoint{I}
    \tkzGetLength{rI}
  \tkzDefCircle[ex](C,A,B) \tkzGetPoint{J}
    \tkzGetLength{rJ}
  \tkzDefCircle[ex](B,C,A) \tkzGetPoint{K}
    \tkzGetLength{rK}
   \tkzDefCircle[in](B,C,A) \tkzGetPoint{O}
     \tkzGetLength{rO}
  \tkzDrawLines[add=1.5 and 1.5](A,B A,C B,C)
  \tkzDrawPoints(I,J,K)
  \tkzDrawPolygon(A,B,C)
  \tkzDrawPolygon[dashed](I,J,K)
  \tkzDrawCircle[R,blue!50!black](O,\rO)
  \tkzDrawSegments[dashed](A,K B,J C,I)
  \tkzDrawPoints(A,B,C)
  \tkzDrawCircles[R](J,{\rJ} I,{\rI} K,{\rK})
  \tkzLabelPoints(A,B,C,I,J,K)
\end{tikzpicture}
\end{tkzexample}

\newpage

\subsubsection{Spieker circle with option \tkzname{spieker}}

The incircle of the medial triangle $M_aM_bM_c$ is the Spieker circle:

\begin{tkzexample}[latex=6cm, small]
\begin{tikzpicture}[scale=1]
  \tkzDefPoints{ 0/0/A,4/0/B,0.8/4/C}
  \tkzDefSpcTriangle[medial](A,B,C){M_a,M_b,M_c}
  \tkzDefTriangleCenter[spieker](A,B,C)
  \tkzGetPoint{S_p}
  \tkzDrawPolygon[blue](A,B,C)
  \tkzDrawPolygon[red](M_a,M_b,M_c)
  \tkzDrawPoints[blue](B,C,A)
  \tkzDrawPoints[red](M_a,M_b,M_c,S_p)
  \tkzDrawCircle[in,red](M_a,M_b,M_c)
  \tkzAutoLabelPoints[center=S_p,dist=.3](M_a,M_b,M_c)
  \tkzLabelPoints[blue,right](S_p)
  \tkzAutoLabelPoints[center=S_p](A,B,C)
\end{tikzpicture}
\end{tkzexample}

\subsubsection{Orthogonal circle passing through two given points, option \tkzname{orthogonal through}}

\begin{tkzexample}[latex=7cm,small]
\begin{tikzpicture}[scale=1]
  \tkzDefPoint(0,0){O}
  \tkzDefPoint(1,0){A}
  \tkzDrawCircle(O,A)
  \tkzDefPoint(-1.5,-1.5){z1}
  \tkzDefPoint(1.5,-1.25){z2}
  \tkzDefCircle[orthogonal through=z1 and z2](O,A)
   \tkzGetPoint{c}
  \tkzDrawCircle[thick,color=red](tkzPointResult,z1)
  \tkzDrawPoints[fill=red,color=black,
  size=4](O,A,z1,z2,c)
  \tkzLabelPoints(O,A,z1,z2,c)
\end{tikzpicture}
\end{tkzexample}

\subsubsection{Orthogonal circle of given center}

\begin{tkzexample}[latex=7cm,small]
\begin{tikzpicture}[scale=.75]
  \tkzDefPoints{0/0/O,1/0/A}
  \tkzDefPoints{1.5/1.25/B,-2/-3/C}
  \tkzDefCircle[orthogonal from=B](O,A)
  \tkzGetPoints{z1}{z2}
  \tkzDefCircle[orthogonal from=C](O,A)
  \tkzGetPoints{t1}{t2}
  \tkzDrawCircle(O,A)
  \tkzDrawCircle[thick,color=red](B,z1)
  \tkzDrawCircle[thick,color=red](C,t1)
  \tkzDrawPoints(t1,t2,C)
  \tkzDrawPoints(z1,z2,O,A,B)
  \tkzLabelPoints(O,A,B,C)
\end{tikzpicture}
\end{tkzexample}

%<---------------------------------------------------------------------------->
\newpage

\section{Draw, Label the Circles}

\begin{itemize}
\item I created a first macro  \tkzcname{tkzDrawCircle},

\item then a macro that allows you to color a disc, but without drawing the
circle. \tkzcname{tkzFillCircle},

\item sometimes, it is necessary for a drawing to be contained in a disc,this
is the role assigned to \tkzcname{tkzClipCircle},

\item It finally remains to be able to give a label to designate a circle and
if several possibilities are offered, we will see here
\tkzcname{tkzLabelCircle}.
\end{itemize}

\subsection{Draw a circle}

\begin{NewMacroBox}{tkzDrawCircle}{\oarg{local options}\parg{A,B}}%
\tkzHandBomb{}Attention you need only two points to define a radius or a
diameter.  An additional option \tkzname{R} is available  to give a measure
directly.

\medskip
\begin{tabular}{lll}%
\toprule
arguments           & example & explication                         \\
\midrule
\TAline{\parg{pt1,pt2}}{\parg{A,B}} {two points to define a radius or a
diameter}
\bottomrule
\end{tabular}

\medskip
\begin{tabular}{lll}%
\toprule
options             & default & definition                         \\
\midrule
\TOline{through}{through}{circle with two points defining a radius}
\TOline{diameter}{through}{circle with two points defining a diameter}
\TOline{R} {through}{circle characterized by a point and the measurement of a
radius}
\bottomrule
\end{tabular}

\medskip
Of course, you have to add all the styles of \TIKZ\ for the tracings\dots
\end{NewMacroBox}

\subsubsection{Circles and styles, draw a circle and color the disc}

We'll see that it's possible to colour in a disc while tracing the circle.

\begin{tkzexample}[latex=7cm,small]
\begin{tikzpicture}
  \tkzDefPoint(0,0){O}
  \tkzDefPoint(3,0){A}
  % circle with centre O and passing through A
  \tkzDrawCircle[color=blue](O,A)
  % diameter circle $[OA]$
  \tkzDrawCircle[diameter,color=red,%
                 line width=2pt,fill=red!40,%
                 opacity=.5](O,A)
  % circle with centre O and radius = exp(1) cm
  \edef\rayon{\fpeval{0.25*exp(1)}}
  \tkzDrawCircle[R,color=orange](O,\rayon cm)
\end{tikzpicture}
\end{tkzexample}

\newpage

\subsection{Drawing circles}

\begin{NewMacroBox}{tkzDrawCircles}{\oarg{local options}\parg{A,B C,D}}%
\tkzHandBomb{}Attention, the arguments are lists of two points. The circles that
can be drawn are the same as in the previous macro. An additional option
\tkzname{R} is available to give  a measure directly.

\medskip
\begin{tabular}{lll}%
\toprule
arguments           & example & explication                         \\
\midrule
\TAline{\parg{pt1,pt2 pt3,pt4,\dots}}{\parg{A,B C,D}} {List of two points}
\bottomrule
\end{tabular}

\medskip
\begin{tabular}{lll}%
\toprule
options             & default & definition                         \\
\midrule
\TOline{through}{through}{circle with two points defining a radius}
\TOline{diameter}{through}{circle with two points defining a diameter}
\TOline{R} {through}{circle characterized by a point and the measurement of a
radius}
\bottomrule
\end{tabular}

\medskip
Of course, you have to add all the styles of \TIKZ\ for the tracings\dots
\end{NewMacroBox}

\subsubsection{Circles defined by a triangle}

\begin{tkzexample}[vbox,small]
\begin{tikzpicture}[scale=1.0]
  \tkzDefPoint(0,0){A}
  \tkzDefPoint(2,0){B}
  \tkzDefPoint(3,2){C}
  \tkzDrawPolygon(A,B,C)
  \tkzDrawCircles(A,B B,C C,A)
  \tkzDrawPoints(A,B,C)
  \tkzLabelPoints(A,B,C)
\end{tikzpicture}
\end{tkzexample}

\newpage

\subsubsection{Concentric circles}

\begin{tkzexample}[latex=7cm,small]
\begin{tikzpicture}[scale=0.75]
  \tkzDefPoint(0,0){A}
  \tkzDrawCircles[R](A,1cm A,2cm A,3cm)
  \tkzDrawPoint(A)
  \tkzLabelPoints(A)
\end{tikzpicture}
\end{tkzexample}

\subsubsection{Exinscribed circles}

\begin{tkzexample}[latex=7cm,small]
\begin{tikzpicture}[scale=0.65]
  \tkzDefPoints{0/0/A,4/0/B,1/2.5/C}
  \tkzDrawPolygon(A,B,C)
  \tkzDefCircle[ex](B,C,A)
  \tkzGetPoint{J_c} \tkzGetSecondPoint{T_c}
  \tkzGetLength{rJc}
  \tkzDrawCircle[R](J_c,{\rJc pt})
  \tkzDrawLines[add=0 and 1](C,A C,B)
  \tkzDrawSegment(J_c,T_c)
  \tkzMarkRightAngle(J_c,T_c,B)
  \tkzDrawPoints(A,B,C,J_c,T_c)
\end{tikzpicture}
\end{tkzexample}

\subsubsection{Cardioid}

Based on an idea by O. Reboux made with pst-eucl (Pstricks module) by D.
Rodriguez.

Its name comes from the Greek \textit{kardia (heart)}, in reference to its
shape, and was given to it by Johan Castillon (Wikipedia).

\begin{tkzexample}[latex=7cm,small]
\begin{tikzpicture}[scale=.5]
  \tkzDefPoint(0,0){O}
  \tkzDefPoint(2,0){A}
  \foreach \ang in {5,10,...,360}{%
     \tkzDefPoint(\ang:2){M}
     \tkzDrawCircle(M,A)
   }
\end{tikzpicture}
\end{tkzexample}

\newpage

\subsection{Draw a semicircle}

\begin{NewMacroBox}{tkzDrawSemiCircle}{\oarg{local options}\parg{A,B}}%

\medskip
\begin{tabular}{lll}%
\toprule
arguments           & example & explication                         \\
\midrule
\TAline{\parg{pt1,pt2}}{\parg{O,A} or\parg{A,B}} {radius or diameter}
\bottomrule
\end{tabular}

\medskip
\begin{tabular}{lll}%
\toprule
options             & default & definition                         \\
\midrule
\TOline{through}  {through}{circle characterized by two points defining a
radius}
\TOline{diameter} {through}{circle characterized by two points defining a
diameter}
\end{tabular}
\end{NewMacroBox}

\subsubsection{Use of \tkzcname{tkzDrawSemiCircle}}

\begin{tkzexample}[latex=6.5cm,small]
\begin{tikzpicture}
  \tkzDefPoint(0,0){A}
  \tkzDefPoint(6,0){B}
  \tkzDefSquare(A,B) \tkzGetPoints{C}{D}
  \tkzDrawPolygon(B,C,D,A)
  \tkzDefPoint(3,6){F}
  \tkzDefTriangle[equilateral](C,D) \tkzGetPoint{I}
  \tkzDefPointBy[projection=onto B--C](I) \tkzGetPoint{J}
  \tkzInterLL(D,B)(I,J)  \tkzGetPoint{K}
  \tkzDefPointBy[symmetry=center K](B) \tkzGetPoint{M}
  \tkzDrawCircle(M,I)
  \tkzCalcLength(M,I)  \tkzGetLength{dMI}
  \tkzFillPolygon[color = red!50](A,B,C,D)
  \tkzFillCircle[R,color = yellow](M,\dMI pt)
  \tkzDrawSemiCircle[fill = blue!50!black](F,D)%
\end{tikzpicture}
\end{tkzexample}

\subsection{Colouring a disc}

This was possible with the previous macro, but disk tracing was mandatory, this
is no longer the case.

\begin{NewMacroBox}{tkzFillCircle}{\oarg{local options}\parg{A,B}}%
\begin{tabular}{lll}%
options             & default & definition                         \\
\midrule
\TOline{radius}  {radius}{two points define a radius}
\TOline{R} {radius}{a point and the measurement of a radius }
\bottomrule
\end{tabular}

\medskip
You don't need to put \tkzname{radius} because that's the default option. Of
course, you have to add all the styles of \TIKZ\ for the plots.
\end{NewMacroBox}

\newpage

\subsubsection{Example from a sangaku}

\begin{tkzexample}[latex=7cm,small]
\begin{tikzpicture}
  \tkzInit[xmin=0,xmax = 6,ymin=0,ymax=6]
  \tkzDefPoint(0,0){B}  \tkzDefPoint(6,0){C}%
  \tkzDefSquare(B,C)    \tkzGetPoints{D}{A}
  \tkzClipPolygon(B,C,D,A)
  \tkzDefMidPoint(A,D)  \tkzGetPoint{F}
  \tkzDefMidPoint(B,C)  \tkzGetPoint{E}
  \tkzDefMidPoint(B,D)  \tkzGetPoint{Q}
  \tkzDefTangent[from = B](F,A) \tkzGetPoints{G}{H}
  \tkzInterLL(F,G)(C,D) \tkzGetPoint{J}
  \tkzInterLL(A,J)(F,E) \tkzGetPoint{K}
  \tkzDefPointBy[projection=onto B--A](K)
  \tkzGetPoint{M}
  \tkzFillPolygon[color = green](A,B,C,D)
  \tkzFillCircle[color = orange](B,A)
  \tkzFillCircle[color = blue!50!black](M,A)
  \tkzFillCircle[color = purple](E,B)
  \tkzFillCircle[color = yellow](K,Q)
\end{tikzpicture}
\end{tkzexample}

\subsection{Clipping a disc}

\begin{NewMacroBox}{tkzClipCircle}{\oarg{local options}\parg{A,B} or
\parg{A,r}}%
\begin{tabular}{lll}%
\toprule
arguments           & example & explication                         \\
\midrule
\TAline{\parg{A,B} or \parg{A,r}}{\parg{A,B} or \parg{A,2cm}} {AB radius or
diameter }
\bottomrule
\end{tabular}

\medskip
\begin{tabular}{lll}%
options             & default & definition                         \\
\midrule
\TOline{radius} {radius}{circle characterized by two points defining a radius}
\TOline{R} {radius}{circle characterized by a point and the measurement of a
radius }
\bottomrule
\end{tabular}

\medskip
It is not necessary to put \tkzname{radius} because that is the default option.
\end{NewMacroBox}

\subsubsection{Example}

\begin{tkzexample}[latex=7cm,small]
\begin{tikzpicture}
  \tkzInit[xmax=5,ymax=5]
  \tkzGrid
  \tkzClip
  \tkzDefPoint(0,0){A}
  \tkzDefPoint(2,2){O}
  \tkzDefPoint(4,4){B}
  \tkzDefPoint(6,6){C}
  \tkzDrawPoints(O,A,B,C)
  \tkzLabelPoints(O,A,B,C)
  \tkzDrawCircle(O,A)
  \tkzClipCircle(O,A)
  \tkzDrawLine(A,C)
  \tkzDrawCircle[fill=red!20,opacity=.5](C,O)
\end{tikzpicture}
\end{tkzexample}

\newpage

\subsection{Giving a label to a circle}

\begin{NewMacroBox}{tkzLabelCircle}{\oarg{local options}\parg{A,B}\parg{angle}\marg{label}}%
\begin{tabular}{lll}%
options             & default & definition                         \\
\midrule
\TOline{radius}  {radius}{circle characterized by two points defining a radius}
\TOline{R} {radius}{circle characterized by a point and the measurement of a
radius }
\bottomrule
\end{tabular}

\medskip
You don't need to put \tkzname{radius} because that's the default option. We can
use the styles from \TIKZ. The label is created and therefore \enquote{passed} between
braces.
\end{NewMacroBox}

\subsubsection{Example}
\begin{tkzexample}[latex=7cm,small]
\begin{tikzpicture}[scale=1.25]
  \tkzDefPoint(0,0){O}
  \tkzDefPoint(2,0){N}
  \tkzDefPointBy[rotation=center O angle 50](N)
    \tkzGetPoint{M}
  \tkzDefPointBy[rotation=center O angle -20](N)
    \tkzGetPoint{P}
  \tkzDefPointBy[rotation=center O angle 125](N)
    \tkzGetPoint{P'}
  \tkzLabelCircle[above=4pt](O,N)(120){$\mathcal{C}$}
  \tkzDrawCircle(O,M)
  \tkzFillCircle[color=blue!20,opacity=.4](O,M)
  \tkzLabelCircle[R,draw,fill=orange, text width=2cm,
     text centered](O,3 cm)(-60)%
     {The circle\\ $\mathcal{C}$}
  \tkzDrawPoints(M,P)
  \tkzLabelPoints[right](M,P)
\end{tikzpicture}
\end{tkzexample}

\endinput
 
\section{Using the compass}

\subsection{Main macro \tkzcname{tkzCompass}}

\begin{NewMacroBox}{tkzCompass}{\oarg{local options}\parg{A,B}}%
This macro allows you to leave a compass trace, i.e. an arc at a designated
point. The center must be indicated. Several specific options will modify the
appearance of the arc as well as TikZ options such as style, color, line
thickness etc.

You can define the length of the arc with the option |length| or the option
|delta|.

\medskip
\begin{tabular}{lll}%
\toprule
options             & default & definition                        \\
\midrule
\TOline{delta} {0 (deg)}{Modifies the angle of the arc by increasing it
symmetrically (in degrees)}
\TOline{length}{1 (cm)}{Changes the length (in cm)}
\end{tabular}
\end{NewMacroBox}

\subsubsection{Option \tkzname{length}}

\begin{tkzexample}[latex=7cm,small]
\begin{tikzpicture}
  \tkzDefPoint(1,1){A}
  \tkzDefPoint(6,1){B}
  \tkzInterCC[R](A,4cm)(B,3cm)
  \tkzGetPoints{C}{D}
  \tkzDrawPoint(C)
  \tkzCompass[color=red,length=1.5](A,C)
  \tkzCompass[color=red](B,C)
  \tkzDrawSegments(A,B A,C B,C)
\end{tikzpicture}
\end{tkzexample}

\subsubsection{Option \tkzname{delta}}

\begin{tkzexample}[latex=7cm,small]
\begin{tikzpicture}
  \tkzDefPoint(0,0){A}
  \tkzDefPoint(5,0){B}
  \tkzInterCC[R](A,4cm)(B,3cm)
  \tkzGetPoints{C}{D}
  \tkzDrawPoints(A,B,C)
  \tkzCompass[color=red,delta=20](A,C)
  \tkzCompass[color=red,delta=20](B,C)
  \tkzDrawPolygon(A,B,C)
  \tkzMarkAngle(A,C,B)
\end{tikzpicture}
\end{tkzexample}

\subsection{Multiple constructions \tkzcname{tkzCompasss}}

\begin{NewMacroBox}{tkzCompasss}{\oarg{local options}\parg{pt1,pt2, pt3,pt4,\dots}}%
\tkzHandBomb{}Attention the arguments are lists of two points. This saves a few
lines of code.

\medskip
\begin{tabular}{lll}%
\toprule
options             & default & definition                        \\
\midrule
\TOline{delta} {0}{Modifies the angle of the arc by increasing it symmetrically}
\TOline{length}{1}{Changes the length}
\end{tabular}
\end{NewMacroBox}

\begin{tkzexample}[latex=8cm,small]
\begin{tikzpicture}[scale=.75]
  \tkzDefPoint(2,2){A}
  \tkzDefPoint(5,-2){B}
  \tkzDefPoint(3,4){C}
  \tkzDrawPoints(A,B)
  \tkzDrawPoint[color=red,shape=cross out](C)
  \tkzCompasss[color=orange](A,B A,C B,C C,B)
  \tkzShowLine[mediator,color=red,
                      dashed,length = 2](A,B)
  \tkzShowLine[parallel = through C,
                    color=blue,length=2](A,B)
  \tkzDefLine[mediator](A,B)
  \tkzGetPoints{i}{j}
  \tkzDefLine[parallel=through C](A,B)
  \tkzGetPoint{D}
  \tkzDrawLines[add=.6 and .6](C,D A,C B,D)
  \tkzDrawLines(i,j) \tkzDrawPoints(A,B,C,i,j,D)
  \tkzLabelPoints(A,B,C,i,j,D)
\end{tikzpicture}
\end{tkzexample}

\subsection{Configuration macro \tkzcname{tkzSetUpCompass}}

\begin{NewMacroBox}{tkzSetUpCompass}{\oarg{local options}}%
\begin{tabular}{lll}%
options             & default & definition                        \\
\midrule
\TOline{line width}  {0.4pt}{line thickness}
\TOline{color}  {black!50}{line colour}
\TOline{style}  {solid}{solid line style, dashed,dotted, \dots}
\end{tabular}
\end{NewMacroBox}

\subsubsection{Use of \tkzcname{tkzSetUpCompass}}

\begin{tkzexample}[latex=7cm,small]
\begin{tikzpicture}[showbi/.style={bisector,
                    size=2,gap=3}, scale=.75]
  \tkzSetUpCompass[color=blue,line width=.3 pt]
  \tkzDefPoints{0/1/A, 8/3/B, 3/6/C}
  \tkzDrawPolygon(A,B,C)
  \tkzDefLine[bisector](B,A,C) \tkzGetPoint{a}
  \tkzDefLine[bisector](C,B,A) \tkzGetPoint{b}
  \tkzShowLine[showbi](B,A,C)
  \tkzShowLine[showbi](C,B,A)
  \tkzInterLL(A,a)(B,b) \tkzGetPoint{I}
  \tkzDefPointBy[projection= onto A--B](I)
  \tkzGetPoint{H}
  \tkzDrawCircle[radius,color=gray](I,H)
  \tkzDrawSegments[color=gray!50](I,H)
  \tkzDrawLines[add=0 and -.2,color=blue!50](A,a B,b)
  \tkzShowBB
\end{tikzpicture}
\end{tkzexample}

\endinput

\section{Sectors}
\subsection{\tkzcname{tkzDrawSector}}
\tkzHandBomb\  Attention the arguments vary according to the options.
\begin{NewMacroBox}{tkzDrawSector}{\oarg{local options}\parg{O,\dots}\parg{\dots}}%
\begin{tabular}{lll}%
options             & default & definition                         \\
\midrule
\TOline{towards}{towards}{$O$ is the center and the arc from $A$ to $(OB)$}
\TOline{rotate} {towards}{the arc starts from $A$ and the angle determines its length }
\TOline{R}{towards}{We give the radius and two angles}
\TOline{R with nodes}{towards}{We give the radius and two points}
\bottomrule
\end{tabular}

\medskip
You have to add, of course, all the styles of \TIKZ\ for tracings...

\medskip
\begin{tabular}{lll}%
\toprule
options             & arguments & example                         \\
\midrule
\TOline{towards}{\parg{pt,pt}\parg{pt}}{\tkzcname{tkzDrawSector(O,A)(B)}}
\TOline{rotate} {\parg{pt,pt}\parg{an}}{\tkzcname{tkzDrawSector[rotate,color=red](O,A)(90)}}
\TOline{R}{\parg{pt,$r$}\parg{an,an}}{\tkzcname{tkzDrawSector[R,color=blue](O,2 cm)(30,90)}}
\TOline{R with nodes}{\parg{pt,$r$}\parg{pt,pt}}{\tkzcname{tkzDrawSector[R with nodes](O,2 cm)(A,B)}}
\bottomrule
\end{tabular}
\end{NewMacroBox}

Here are a few examples:

\subsubsection{\tkzcname{tkzDrawSector} and \tkzname{towards}}
There's no need to put \tkzname{towards}. You can use \tkzname{fill} as an option.

\begin{tkzexample}[latex=7cm,small]
\begin{tikzpicture}[scale=1]
  \tkzDefPoint(0,0){O}
  \tkzDefPoint(-30:3){A}
  \tkzDefPointBy[rotation = center O angle -60](A)
  \tkzDrawSector[fill=red!50](O,A)(tkzPointResult)
 \begin{scope}[shift={(-60:1cm)}]
  \tkzDefPoint(0,0){O}
  \tkzDefPoint(-30:3){A}
  \tkzDefPointBy[rotation = center O angle -60](A)
  \tkzDrawSector[fill=blue!50](O,tkzPointResult)(A)
  \end{scope}
\end{tikzpicture}
\end{tkzexample}

\subsubsection{\tkzcname{tkzDrawSector} and \tkzname{rotate}}
\begin{tkzexample}[latex=7cm,small]
\begin{tikzpicture}[scale=2]
 \tkzDefPoint(0,0){O}
 \tkzDefPoint(2,2){A}
 \tkzDrawSector[rotate,draw=red!50!black,%
 fill=red!20](O,A)(30)
 \tkzDrawSector[rotate,draw=blue!50!black,%
 fill=blue!20](O,A)(-30)
\end{tikzpicture}
\end{tkzexample}

\subsubsection{\tkzcname{tkzDrawSector} and \tkzname{R}}
\begin{tkzexample}[latex=7cm,small]
\begin{tikzpicture}[scale=1.25]
 \tkzDefPoint(0,0){O}
 \tkzDefPoint(2,-1){A}
 \tkzDrawSector[R,draw=white,%
 fill=red!50](O,2cm)(30,90)
 \tkzDrawSector[R,draw=white,%
 fill=red!60](O,2cm)(90,180)
 \tkzDrawSector[R,draw=white,%
 fill=red!70](O,2cm)(180,270)
 \tkzDrawSector[R,draw=white,%
 fill=red!90](O,2cm)(270,360)
\end{tikzpicture}
\end{tkzexample}

\subsubsection{\tkzcname{tkzDrawSector} and \tkzname{R}}
\begin{tkzexample}[latex=7cm,small]
\begin{tikzpicture}[scale=1.25]
 \tkzDefPoint(0,0){O}
 \tkzDefPoint(4,-2){A}
 \tkzDefPoint(4,1){B}
 \tkzDefPoint(3,3){C}
 \tkzDrawSector[R with nodes,%
                fill=blue!20](O,1 cm)(B,C)
 \tkzDrawSector[R with nodes,%
                fill=red!20](O,1.25 cm)(A,B)
\tkzDrawSegments(O,A O,B O,C)
\tkzDrawPoints(O,A,B,C)
\tkzLabelPoints(A,B,C)
\tkzLabelPoints[left](O)
\end{tikzpicture}
\end{tkzexample}

\subsubsection{\tkzcname{tkzDrawSector} and \tkzname{R with nodes}}
\begin{tkzexample}[latex=7cm,small]
\begin{tikzpicture} [scale=.5]
 \tkzDefPoint(-1,-2){A}
 \tkzDefPoint(1,3){B}
 \tkzDefRegPolygon[side,sides=6](A,B)
 \tkzGetPoint{O}
 \tkzDrawPolygon[fill=black!10,
                 draw=blue](P1,P...,P6)
 \tkzLabelRegPolygon[sep=1.05](O){A,...,F}
 \tkzDrawCircle[dashed](O,A)
 \tkzLabelSegment[above,sloped,
                  midway](A,B){\(A B = 16m\)}
 \foreach \i  [count=\xi from 1]  in {2,...,6,1}
   {%
    \tkzDefMidPoint(P\xi,P\i)
    \path (O) to [pos=1.1] node {\xi} (tkzPointResult) ;
    }
  \tkzDefRandPointOn[segment = P3--P5]
  \tkzGetPoint{S}
  \tkzDrawSegments[thick,dashed,red](A,S S,B)
  \tkzDrawPoints(P1,P...,P6,S)
  \tkzLabelPoint[left,above](S){$S$}
  \tkzDrawSector[R with nodes,fill=red!20](S,2 cm)(A,B)
  \tkzLabelAngle[pos=1.5](A,S,B){$\alpha$}
\end{tikzpicture}
\end{tkzexample}

\subsection{\tkzcname{tkzFillSector}}
\tkzHandBomb\ Attention the arguments vary according to the options.
\begin{NewMacroBox}{tkzFillSector}{\oarg{local options}\parg{O,\dots}\parg{\dots}}%
\begin{tabular}{lll}%
options          & default & definition      \\
\midrule
\TOline{towards}{towards}{$O$ is the center and the arc from $A$ to $(OB)$}
\TOline{rotate} {towards}{the arc starts from A and the angle determines its length }
\TOline{R}{towards}{We give the radius and two angles}
\TOline{R with nodes}{towards}{We give the radius and two points}
\bottomrule
\end{tabular}

\medskip
Of course, you have to add all the styles of \TIKZ\ for the tracings...

\medskip
\begin{tabular}{lll}%
\toprule
options             & arguments & example                         \\
\midrule
\TOline{towards}{\parg{pt,pt}\parg{pt}}{\tkzcname{tkzFillSector(O,A)(B)}}
\TOline{rotate} {\parg{pt,pt}\parg{an}}{\tkzcname{tkzFillSector[rotate,color=red](O,A)(90)}}
\TOline{R}{\parg{pt,$r$}\parg{an,an}}{\tkzcname{tkzFillSector[R,color=blue](O,2 cm)(30,90)}}
\TOline{R with nodes}{\parg{pt,$r$}\parg{pt,pt}}{\tkzcname{tkzFillSector[R with nodes](O,2 cm)(A,B)}}
\end{tabular}
\end{NewMacroBox}

\subsubsection{\tkzcname{tkzFillSector} and \tkzname{towards}}
It is useless to put \tkzname{towards} and you will notice that the contours are not drawn, only the surface is colored.
\begin{tkzexample}[latex=5.75cm,small]
\begin{tikzpicture}[scale=.6]
  \tkzDefPoint(0,0){O}
  \tkzDefPoint(-30:3){A}
  \tkzDefPointBy[rotation = center O angle -60](A)
  \tkzFillSector[fill=red!50](O,A)(tkzPointResult)
  \begin{scope}[shift={(-60:1cm)}]
   \tkzDefPoint(0,0){O}
   \tkzDefPoint(-30:3){A}
   \tkzDefPointBy[rotation = center O angle -60](A)
   \tkzFillSector[color=blue!50](O,tkzPointResult)(A)
  \end{scope}
\end{tikzpicture}
\end{tkzexample}


\subsubsection{\tkzcname{tkzFillSector} and \tkzname{rotate}}
\begin{tkzexample}[latex=5.75cm,small]
\begin{tikzpicture}[scale=1.5]
 \tkzDefPoint(0,0){O} \tkzDefPoint(2,2){A}
 \tkzFillSector[rotate,color=red!20](O,A)(30)
 \tkzFillSector[rotate,color=blue!20](O,A)(-30)
\end{tikzpicture}
\end{tkzexample}

\subsection{\tkzcname{tkzClipSector}}
\tkzHandBomb\  Attention the arguments vary according to the options.
\begin{NewMacroBox}{tkzClipSector}{\oarg{local options}\parg{O,\dots}\parg{\dots}}%
\begin{tabular}{lll}%
options             & default & definition                         \\
\midrule
\TOline{towards}{towards}{$O$ is the centre and the sector starts from $A$ to $(OB)$}
\TOline{rotate} {towards}{The sector starts from $A$ and the angle determines its amplitude. }
\TOline{R}{towards}{We give the radius and two angles}
\bottomrule
\end{tabular}

\medskip
You have to add, of course, all the styles of \TIKZ\ for tracings...

\medskip
\begin{tabular}{lll}%
\toprule
options             & arguments & example                         \\
\midrule
\TOline{towards}{\parg{pt,pt}\parg{pt}}{\tkzcname{tkzClipSector(O,A)(B)}}
\TOline{rotate} {\parg{pt,pt}\parg{angle}}{\tkzcname{tkzClipSector[rotate](O,A)(90)}}
\TOline{R}{\parg{pt,$r$}\parg{angle 1,angle 2}}{\tkzcname{tkzClipSector[R](O,2 cm)(30,90)}}
\end{tabular}
\end{NewMacroBox}

\subsubsection{\tkzcname{tkzClipSector}}
\begin{tkzexample}[latex=7cm,small]
\begin{tikzpicture}[scale=1.5]
  \tkzDefPoint(0,0){O}
  \tkzDefPoint(2,-1){A}
  \tkzDefPoint(1,1){B}
  \tkzDrawSector[color=blue,dashed](O,A)(B)
  \tkzDrawSector[color=blue](O,B)(A)
  \tkzClipBB
  \begin{scope}
    \tkzClipSector(O,B)(A)
    \draw[fill=gray!20] (-1,0) rectangle (3,3);
  \end{scope}
  \tkzDrawPoints(A,B,O)
\end{tikzpicture}
\end{tkzexample}

\endinput


\section{The arcs}
\begin{NewMacroBox}{tkzDrawArc}{\oarg{local options}\parg{O,\dots}\parg{\dots}}%

This macro traces the arc of center $O$. Depending on the options, the arguments differ.   It is a question of determining a starting point and an end point. Either the starting point is given, which is the simplest, or the radius of the arc is given. In the latter case, it is necessary to have two angles. Either the angles can be given directly, or nodes associated with the center can be given to determine them. The angles are in degrees.

\medskip

\begin{tabular}{lll}%
\toprule
options             & default & definition                        \\
\midrule
\TOline{towards}{towards}{$O$ is the center and the arc from $A$ to $(OB)$}
\TOline{rotate} {towards}{the arc starts from $A$ and the angle determines its length}
\TOline{R}{towards}{We give the radius and two angles}
\TOline{R with nodes}{towards}{We give the radius and two points}
\TOline{angles}{towards}{We give the radius and two points}
\TOline{delta}{0}{angle added on each side }
\bottomrule
\end{tabular}

\medskip
Of course, you have to add all the styles of \TIKZ\ for the tracings...

\medskip

\begin{tabular}{lll}%
\toprule
options             & arguments & example                         \\
\midrule
\TOline{towards}{\parg{pt,pt}\parg{pt}}{\tkzcname{tkzDrawArc[delta=10](O,A)(B)}}
\TOline{rotate} {\parg{pt,pt}\parg{an}}{\tkzcname{tkzDrawArc[rotate,color=red](O,A)(90)}}
\TOline{R}{\parg{pt,$r$}\parg{an,an}}{\tkzcname{tkzDrawArc[R](O,2 cm)(30,90)}}
\TOline{R with nodes}{\parg{pt,$r$}\parg{pt,pt}}{\tkzcname{tkzDrawArc[R with nodes](O,2 cm)(A,B)}}
\TOline{angles}{\parg{pt,pt}\parg{an,an}}{\tkzcname{tkzDrawArc[angles](O,A)(0,90)}}
\end{tabular}
\end{NewMacroBox}

Here are a few examples:

\subsection{Option \tkzname{towards}}
It's useless to put \tkzname{towards}. In this first example the arc starts from $A$ and goes to $B$. The arc going from $B$ to $A$ is different. The salient is obtained by going in the direct direction of the trigonometric circle.
\begin{tkzexample}[latex=6cm,small]
\begin{tikzpicture}
  \tkzDefPoint(0,0){O}
  \tkzDefPoint(2,-1){A}
  \tkzDefPointBy[rotation= center O angle 90](A)
  \tkzGetPoint{B}
  \tkzDrawArc[color=blue,<->](O,A)(B)
  \tkzDrawArc(O,B)(A)
  \tkzDrawLines[add = 0 and .5](O,A O,B)
  \tkzDrawPoints(O,A,B)
  \tkzLabelPoints[below](O,A,B)
\end{tikzpicture}
\end{tkzexample}


\subsection{Option \tkzname{towards}}
In this one, the arc starts from A but stops on the right (OB).

\begin{tkzexample}[latex=6cm,small]
\begin{tikzpicture}[scale=1.5]
  \tkzDefPoint(0,0){O}
  \tkzDefPoint(2,-1){A}
  \tkzDefPoint(1,1){B}
  \tkzDrawArc[color=blue,->](O,A)(B)
  \tkzDrawArc[color=gray](O,B)(A)
  \tkzDrawArc(O,B)(A)
  \tkzDrawLines[add = 0 and .5](O,A O,B)
  \tkzDrawPoints(O,A,B)
  \tkzLabelPoints[below](O,A,B)
\end{tikzpicture}
\end{tkzexample}

\subsection{Option \tkzname{rotate}}
\begin{tkzexample}[latex=5cm,small]
\begin{tikzpicture}
  \tkzDefPoint(0,0){O}
  \tkzDefPoint(2,-2){A}
  \tkzDefPoint(60:2){B}
  \tkzDrawLines[add = 0 and .5](O,A O,B)
  \tkzDrawArc[rotate,color=red](O,A)(180)
  \tkzDrawPoints(O,A,B)
  \tkzLabelPoints[below](O,A,B)
\end{tikzpicture}
\end{tkzexample}


\subsection{Option \tkzname{R}}
\begin{tkzexample}[latex=5cm,small]
\begin{tikzpicture}
  \tkzDefPoints{0/0/O}
  \tikzset{compass style/.append style={<->}}
  \tkzDrawArc[R,color=orange,double](O,3cm)(270,360)
  \tkzDrawArc[R,color=blue,double](O,2cm)(0,270)
  \tkzDrawPoint(O)
  \tkzLabelPoint[below](O){$O$}
\end{tikzpicture}
\end{tkzexample}

\subsection{Option \tkzname{R with nodes}}
\begin{tkzexample}[latex=5cm,small]
\begin{tikzpicture}
  \tkzDefPoint(0,0){O}
  \tkzDefPoint(2,-1){A}
  \tkzDefPoint(1,1){B}
  \tkzCalcLength(B,A)\tkzGetLength{radius}
  \tkzDrawArc[R with nodes](B,\radius pt)(A,O)
\end{tikzpicture}
\end{tkzexample}

\subsection{Option \tkzname{delta}}
This option allows a bit like \tkzcname{tkzCompass} to place an arc and overflow on either side. delta is a measure in degrees.

\begin{tkzexample}[latex=7cm,small]
\begin{tikzpicture}
 \tkzDefPoint(0,0){A}
 \tkzDefPoint(5,0){B}
 \tkzDefPointBy[rotation= center A angle 60](B)
 \tkzGetPoint{C}
 \tkzSetUpLine[color=gray]
 \tkzDefPointBy[symmetry= center C](A)
 \tkzGetPoint{D}
 \tkzDrawSegments(A,B A,D)
 \tkzDrawLine(B,D)
 \tkzSetUpCompass[color=orange]
 \tkzDrawArc[orange,delta=10](A,B)(C)
 \tkzDrawArc[orange,delta=10](B,C)(A)
 \tkzDrawArc[orange,delta=10](C,D)(D)
 \tkzDrawPoints(A,B,C,D)
 \tkzLabelPoints(A,B,C,D)
 \tkzMarkRightAngle(D,B,A)
\end{tikzpicture}
\end{tkzexample}

\subsection{Option \tkzname{angles}: example 1}

\begin{tkzexample}[latex=7cm,small]
\begin{tikzpicture}[scale=.75]
  \tkzDefPoint(0,0){A}
  \tkzDefPoint(5,0){B}
  \tkzDefPoint(2.5,0){O}
  \tkzDefPointBy[rotation=center O angle 60](B)
  \tkzGetPoint{D}
  \tkzDefPointBy[symmetry=center D](O)
  \tkzGetPoint{E}
  \tkzSetUpLine[color=Maroon]
  \tkzDrawArc[angles](O,B)(0,180)
  \tkzDrawArc[angles,](B,O)(100,180)
  \tkzCompass[delta=20](D,E)
  \tkzDrawLines(A,B O,E B,E)
  \tkzDrawPoints(A,B,O,D,E)
  \tkzLabelPoints(A,B,O,D,E)
  \tkzMarkRightAngle(O,B,E)
\end{tikzpicture}
\end{tkzexample}

\subsection{Option \tkzname{angles}: example 2}


\begin{tkzexample}[latex=7cm,small]
  \begin{tikzpicture}
   \tkzDefPoint(0,0){O}
   \tkzDefPoint(5,0){I}
   \tkzDefPoint(0,5){J}
   \tkzInterCC(O,I)(I,O)\tkzGetPoints{B}{C}
   \tkzInterCC(O,I)(J,O)\tkzGetPoints{D}{A}
   \tkzInterCC(I,O)(J,O)\tkzGetPoints{L}{K}
   \tkzDrawArc[angles](O,I)(0,90)
   \tkzDrawArc[angles,color=gray,style=dashed](I,O)(90,180)
   \tkzDrawArc[angles,color=gray,style=dashed](J,O)(-90,0)
   \tkzDrawPoints(A,B,K)
   \foreach \point in {I,A,B,J,K}{\tkzDrawSegment(O,\point)}
  \end{tikzpicture}
\end{tkzexample}


 \endinput


%!TEX root = /Users/ego/Boulot/TKZ/tkz-euclide/doc_fr/TKZdoc-euclide-main.tex


\section{Rapporteurs} % (fold)
\label{sec:rapporteurs}

D'après une idée de Yves Combe., la  macro suivante permet de dessiner un rapporteur. J'ai ajouté mon propre rapporteur qui est obtenu avec l'option \tkzname{full} (par défaut), celui de Yves est obtenu avec \tkzname{half}.


\begin{NewMacroBox}{tkzProtractor}{\oarg{local options}\parg{$O,A$}}
 
\medskip
\begin{tabular}{lll}
\toprule
options            & défaut  & définition                         \\ 
\midrule
\TOline{with}     {full}    { full ou bien half}
\TOline{lw}  {0.4 pt} { épaisseur des lignes}
\TOline{scale}   {1} { ratio : permet d'ajuster la taille du rapporteur} \TOline{return} {false} { sens indirect du cercle trigonométrique}
\bottomrule
\end{tabular}

\medskip
\emph{Le principe de fonctionnement est encore plus simple. Il suffit de nommer une demi-droite. Le rapporteur sera placé sur l'origine $O$ la direction de la demi-droites est donnée par $A$. L'angle est mesuré dans le sens direct du cercle trigonométrique} 
\end{NewMacroBox}


\subsection{Le rapporteur circulaire} 

Mesure dans le sens direct

\begin{tkzltxexample}[] 
\begin{tikzpicture}[scale=.75]
\tkzDefPoint(2,3){A}
\tkzDefPoint[shift={(2,3)}](31:8){B}
\tkzDefPoint[shift={(2,3)}](158:8){C}
\tkzDrawSegments[color = red,
           line width = 1pt](A,B A,C)
\tkzProtractor[with  = full,
               scale = 1.25](A,B)  
\end{tikzpicture}  
\end{tkzltxexample}
 
\vspace*{6cm}\hspace*{6cm}   
\begin{tikzpicture}[scale=.75,overlay]
\tkzDefPoint(2,3){A}
\tkzDefPoint[shift={(2,3)}](31:8){B}
\tkzDefPoint[shift={(2,3)}](158:8){C}
\tkzDrawSegments[color = red,
           line width = 1pt](A,B A,C)
\tkzProtractor[with  = full,
               scale = 1.25](A,B)  
\end{tikzpicture}  

\newpage
\subsection{Le rapporteur circulaire, transparent et retourné}
Mesure dans le sens indirect, on retourne le rapporteur.

\begin{center}
  \begin{tkzexample}[vbox] 
\begin{tikzpicture}
  \tkzInit[xmin=-4,xmax=9,ymin=-3,ymax=9]
  \tkzClip
  \tkzDefPoint(2,3){A}
  \tkzDefPoint[shift={(2,3)}](31:8){B}  
  \tkzDefPoint[shift={(2,3)}](158:8){C}   
  \tkzDrawSegments[color=red,line width=1pt](A,B A,C)
  \tkzProtractor[scale=1.25,with=full,return](A,C) 
\end{tikzpicture}
\end{tkzexample}
\end{center}    
 
\newpage
\subsection{Le rapporteur original semi-circulaire (Yves Combes)}

Mesure dans le sens direct avec un rapporteur semi-circulaire
\begin{center} 
\begin{tkzexample}[vbox]
\begin{tikzpicture}
  \tkzInit[xmin=-5,xmax=9,ymin=-3,ymax=10]
  \tkzClip     
  \tkzDefPoint(2,3){A}
  \tkzDefPoint[shift={(2,3)}](31:8){B}  
  \tkzDefPoint[shift={(2,3)}](158:8){C} 
  \tkzDrawSegments[color=red,line width=1pt](A,B A,C)
  \tkzProtractor[scale=1.25,with=half](A,B) 
\end{tikzpicture}
\end{tkzexample}  
\end{center}
\subsection{Le rapporteur semi-circulaire dans le sens indirect}

\begin{center} 
\begin{tkzexample}[vbox]
\begin{tikzpicture}
  \tkzInit[xmin=-5,xmax=9,ymin=-3,ymax=10]
  \tkzClip      
  \tkzDefPoint(2,3){A}
  \tkzDefPoint[shift={(2,3)}](31:8){B}  
  \tkzDefPoint[shift={(2,3)}](158:8){C}  
  \tkzDrawSegments[color=red,line width=1pt](A,B A,C)
  \tkzProtractor[scale=1.25,with=half,return](A,C) 
\end{tikzpicture}
\end{tkzexample}
\end{center}

le cas échéant vous pouvez utiliser la macro originale de Yves

\begin{NewMacroBox}{tkzOriProtractor}{\oarg{local options}}
 
\medskip
\begin{tabular}{lll}
\toprule
options            & défaut  & définition                         \\ 
\midrule
\TOline{with}  {full} {full ou bien half}   
\TOline{lw}  {0.4 pt} {épaisseur des lignes} 
\TOline{shift} {(x;y)}{permet de faire glisser le rapporteur} 
\TOline{rotate}  {0}  {permet de faire pivoter le rapporteur}
\TOline{scale}   {1}  {ratio : permet d'ajuster la taille du rapporteur} \TOline{return}{false}{sens indirect du cercle trigonométrique} 
\bottomrule
\end{tabular}

\medskip
\emph{Le principe de fonctionnement est encore plus simple. Il suffit de nommer une demi-droite. Le rapporteur sera placé sur l'origine.} 
\end{NewMacroBox}

\subsection{Le rapporteur semi-circulaire avec la macro originale} 
\begin{center} 
  \begin{tkzexample}[vbox] 
\begin{tikzpicture}
  \tkzInit[xmin=-5,xmax=9,ymin=-3,ymax=10]
  \tkzClip  
  \tkzDefPoint(2,3){A} 
  \tkzDefPoint[shift={(2,3)}](158:8){B}
  \tkzDefPoint[shift={(2,3)}](31:8){C}  
  \tkzDrawSegments[color=red,line width=1pt](A,B A,C)
  \tkzOriProtractor[shift = {(2,3)},scale=1.25, rotate = +31,with=half]
\end{tikzpicture}
\end{tkzexample}
\end{center} 

\subsection{Le rapporteur semi-circulaire avec la macro originale dans le sens indirect} 
\begin{center} 
  \begin{tkzexample}[vbox] 
\begin{tikzpicture}
  \tkzInit[xmin=-5,xmax=9,ymin=-3,ymax=10]
  \tkzClip  
  \tkzDefPoint(2,3){A} 
  \tkzDefPoint[shift={(2,3)}](158:8){B}
  \tkzDefPoint[shift={(2,3)}](31:8){C}  
  \tkzDrawSegments[color=red,line width=1pt](A,B A,C)
  \tkzOriProtractor[shift = {(2,3)},scale=1.25, rotate = -22,with=half]
\end{tikzpicture}
   \end{tkzexample}
\end{center}    
\endinput
\section{Miscellaneous tools}
\subsection{Duplicate a segment} 
This involves constructing a segment on a given half-line of the same length as a given segment.

\begin{NewMacroBox}{tkzDuplicateSegment}{\parg{pt1,pt2}\parg{pt3,pt4}\marg{pt5}}%
This involves creating a segment on a given half-line of the same length as a given segment . It is in fact the definition of a point.
\tkzcname{tkzDuplicateSegment} is the new name of \tkzcname{tkzDuplicateLen}.
\medskip  
\begin{tabular}{lll}%
\toprule
arguments             & example & explication                         \\ 

\midrule
\TAline{(pt1,pt2)(pt3,pt4)\{pt5\}} {\tkzcname{tkzDuplicateSegment}(A,B)(E,F)\{C\}}{AC=EF and $C \in [AB)$} \\  
\bottomrule
\end{tabular}

\medskip
The macro \tkzcname{tkzDuplicateLength} is identical to this one. 
\end{NewMacroBox}

\begin{tkzexample}[latex=6cm,small]
   \begin{tikzpicture}
   \tkzDefPoint(0,0){A}
   \tkzDefPoint(2,-3){B}
   \tkzDefPoint(2,5){C} 
   \tkzDrawSegments[red](A,B A,C)
   \tkzDuplicateSegment(A,B)(A,C)  
   \tkzGetPoint{D}
   \tkzDrawSegment[green](A,D)
   \tkzDrawPoints[color=red](A,B,C,D) 
   \tkzLabelPoints[above right=3pt](A,B,C,D)
 \end{tikzpicture} 
\end{tkzexample} 

\subsubsection{Proportion of gold with \tkzcname{tkzDuplicateSegment}} 
\begin{tkzexample}[latex=7cm,small]
\begin{tikzpicture}[rotate=-90,scale=.75]
 \tkzDefPoint(0,0){A}
 \tkzDefPoint(10,0){B}
 \tkzDefMidPoint(A,B)   
 \tkzGetPoint{I}
 \tkzDefPointWith[orthogonal,K=-.75](B,A)
 \tkzGetPoint{C}
 \tkzInterLC(B,C)(B,I)  \tkzGetSecondPoint{D}
 \tkzDuplicateSegment(B,D)(D,A) \tkzGetPoint{E}
 \tkzInterLC(A,B)(A,E)   \tkzGetPoints{N}{M}
 \tkzDrawArc[orange,delta=10](D,E)(B)
 \tkzDrawArc[orange,delta=10](A,M)(E)
 \tkzDrawLines(A,B B,C A,D)
 \tkzDrawArc[orange,delta=10](B,D)(I)
 \tkzDrawPoints(A,B,D,C,M,I,N)
 \tkzLabelPoints(A,B,D,C,M,I,N)
\end{tikzpicture}
\end{tkzexample}

\subsection{Segment length \tkzcname{tkzCalcLength}}
There's an option in \TIKZ\  named \tkzname{veclen}. This option
 is used to calculate AB if A and B are two points.

The only problem for me is that the version of \TIKZ\ is not accurate enough in some cases. My version uses the \tkzNamePack{xfp} package and is slower, but more accurate.

\begin{NewMacroBox}{tkzCalcLength}{\oarg{local options}\parg{pt1,pt2}\marg{name of macro}}%
The result is stored in a macro.

\medskip
\begin{tabular}{lll}%
\toprule
arguments    & example & explication       \\
\midrule
\TAline{(pt1,pt2)\{name of macro\}} {\tkzcname{tkzCalcLength}(A,B)\{dAB\}}{\tkzcname{dAB} gives $AB$ in pt}
\bottomrule
\end{tabular}

\medskip
Only one option

\begin{tabular}{lll}%
   
\toprule
 options    & default & example       \\
\midrule
\TOline{cm}  {false}{\tkzcname{tkzCalcLength}[cm](A,B)\{dAB\} \tkzcname{dAB} gives $AB$ in cm}
\end{tabular}
\end{NewMacroBox}

\subsubsection{Compass square construction}

\begin{tkzexample}[latex=7cm,small]
\begin{tikzpicture}[scale=1]
  \tkzDefPoint(0,0){A} \tkzDefPoint(4,0){B}
  \tkzDrawLine[add= .6 and .2](A,B)
  \tkzCalcLength[cm](A,B)\tkzGetLength{dAB}
  \tkzDefLine[perpendicular=through A](A,B)
  \tkzDrawLine(A,tkzPointResult) \tkzGetPoint{D}
  \tkzShowLine[orthogonal=through A,gap=2](A,B)
  \tkzMarkRightAngle(B,A,D)
  \tkzVecKOrth[-1](B,A)\tkzGetPoint{C}
  \tkzCompasss(A,D D,C)
  \tkzDrawArc[R](B,\dAB)(80,110)
  \tkzDrawPoints(A,B,C,D)
  \tkzDrawSegments[color=gray,style=dashed](B,C C,D)
  \tkzLabelPoints(A,B,C,D)
\end{tikzpicture}
\end{tkzexample}


\subsection{Transformation from pt to cm}
Not sure if this is necessary and it is only a division by 28.45274 and a multiplication by the same number. The macros are:

\begin{NewMacroBox}{tkzpttocm}{\parg{nombre}\marg{name of macro}}%
\begin{tabular}{lll}%
arguments    & example & explication     \\
\midrule
\TAline{(number){name of macro}} {\tkzcname{tkzpttocm}(120)\{len\}}{\tkzcname{len} gives a number of \tkzname{cm}}
\bottomrule
\end{tabular}

\medskip
You'll have to use \tkzcname{len} along with \tkzname{cm}. The result is stored in a macro.
\end{NewMacroBox}

\subsection{Transformation from cm to pt}
\begin{NewMacroBox}{tkzcmtopt}{\parg{nombre}\marg{name of macro}}%
\begin{tabular}{lll}%
arguments             & example & explication                         \\
\midrule
\TAline{(nombre)\{name of macro\}}{\tkzcname{tkzcmtopt}(5)\{len\}}{\tkzcname{len} length in \tkzname{pt}}
\bottomrule
\end{tabular}

\medskip
The result is stored in a macro. The result can be used with \tkzcname{len} \tkzname{pt}. 
\end{NewMacroBox}

\subsubsection{Example}
The macro \tkzcname{tkzDefCircle[radius](A,B)} defines the radius that we retrieve with \tkzcname{tkzGetLength}, but this result is in \tkzname{pt}.

\begin{tkzexample}[latex=6cm,small]
\begin{tikzpicture}[scale=.5]
 \tkzDefPoint(0,0){A}
 \tkzDefPoint(3,-4){B}
 \tkzDefCircle[through](A,B)
 \tkzGetLength{rABpt}
 \tkzpttocm(\rABpt){rABcm}
 \tkzDrawCircle(A,B)
 \tkzDrawPoints(A,B)
 \tkzLabelPoints(A,B)
 \tkzDrawSegment[dashed](A,B)
 \tkzLabelSegment(A,B){$\pgfmathprintnumber{\rABcm}$}
\end{tikzpicture}
\end{tkzexample}

\subsection{Get point coordinates}
%<--------------------------------------------------------------------------–>
%                    Coordonnées d'un point 
%    result in #2x and #2y    #1 is the point and we get its coordinates
% use either $A$ one point \tkzGetPointCoord(A){V} then \Vx = xA and \Vy = yA
% in cm 
% tkzGetPointCoord with [#1] cm or  pt ?? todo
%<--------------------------------------------------------------------------–>
\begin{NewMacroBox}{tkzGetPointCoord}{\parg{$A$}\marg{name of macro}}%
\begin{tabular}{lll}%
arguments             & example & explication                         \\
\midrule
\TAline{(point)\{name of macro\}} {\tkzcname{tkzGetPointCoord}(A)\{A\}}{\tkzcname{Ax} and \tkzcname{Ay} give coordinates for $A$}
\end{tabular}

\medskip
Stores in two macros the coordinates of a point. If the name of the macro is \tkzname{p}, then \tkzcname{px} and \tkzcname{py} give the coordinates of the chosen point with the cm as unit.
\end{NewMacroBox}

\subsubsection{Coordinate transfer with \tkzcname{tkzGetPointCoord}}

\begin{tkzexample}[width=8cm,small]
\begin{tikzpicture}
 \tkzInit[xmax=5,ymax=3]
 \tkzGrid[sub,orange]
 \tkzAxeXY
 \tkzDefPoint(1,0){A}
 \tkzDefPoint(4,2){B}
 \tkzGetPointCoord(A){a}
 \tkzGetPointCoord(B){b}
 \tkzDefPoint(\ax,\ay){C}
 \tkzDefPoint(\bx,\by){D}
 \tkzDrawPoints[color=red](C,D)
\end{tikzpicture}
\end{tkzexample}

\subsubsection{Sum of vectors with \tkzcname{tkzGetPointCoord}}
\begin{tkzexample}[width=6cm,small]
\begin{tikzpicture}[>=latex]
  \tkzDefPoint(1,4){a}
  \tkzDefPoint(3,2){b}
  \tkzDefPoint(1,1){c}
  \tkzDrawSegment[->,red](a,b)
  \tkzGetPointCoord(c){c}
  \draw[color=blue,->](a) -- ([shift=(b)]\cx,\cy) ;
  \draw[color=purple,->](b) -- ([shift=(b)]\cx,\cy) ;
  \tkzDrawSegment[->,blue](a,c)
  \tkzDrawSegment[->,purple](b,c)
\end{tikzpicture}
\end{tkzexample}

\endinput  
\section{Customization}

\subsection{Use of \tkzcname{tkzSetUpLine}} \label{tkzsetupline}
It is a macro that allows you to define the style of all the lines.

\begin{NewMacroBox}{tkzSetUpLine}{\oarg{local options}}%
\begin{tabular}{lll}%
options &  default & definition                 \\
\midrule
\TOline{color}{black}{colour of the construction lines}
\TOline{line width}{0.4pt}{thickness of the construction lines}
\TOline{style}{solid}{style of construction lines}
\TOline{add}{.2 and .2}{changing the length of a line segment}
\end{tabular}
\end{NewMacroBox}

\subsubsection{Example 1: change line width}
\begin{tkzexample}[latex=8cm,small]
\begin{tikzpicture}
   \tkzSetUpLine[color=blue,line width=1pt]
\begin{scope}[rotate=-90]
    \tkzDefPoint(10,6){C}
    \tkzDefPoint( 0,6){A}
    \tkzDefPoint(10,0){B}
    \tkzDefPointBy[projection = onto B--A](C)
    \tkzGetPoint{H}
    \tkzDrawPolygon(A,B,C)
    \tkzMarkRightAngle[size=.4,fill=blue!20](B,C,A)
    \tkzMarkRightAngle[size=.4,fill=red!20](B,H,C)
    \tkzDrawSegment[color=red](C,H)
\end{scope}
 \tkzLabelSegment[below](C,B){$a$}
 \tkzLabelSegment[right](A,C){$b$}
 \tkzLabelSegment[left](A,B){$c$}
 \tkzLabelSegment[color=red](C,H){$h$}
 \tkzDrawPoints(A,B,C)
 \tkzLabelPoints[above left](H)
 \tkzLabelPoints(B,C)
 \tkzLabelPoints[above](A)
\end{tikzpicture}
\end{tkzexample}




\subsubsection{Example 2: change style of line}

\begin{tkzexample}[latex=7cm,small]
\begin{tikzpicture}[scale=.6]
 \tkzDefPoint(1,0){A} \tkzDefPoint(4,0){B}
 \tkzDefPoint(1,1){C} \tkzDefPoint(5,1){D}
 \tkzDefPoint(1,2){E} \tkzDefPoint(6,2){F}
 \tkzDefPoint(0,4){A'}\tkzDefPoint(3,4){B'}
 \tkzCalcLength[cm](C,D)  \tkzGetLength{rCD}
 \tkzCalcLength[cm](E,F)  \tkzGetLength{rEF}
 \tkzInterCC[R](A',\rCD cm)(B',\rEF cm)
 \tkzGetPoints{I}{J}
 \tkzSetUpLine[style=dashed,color=gray]
 \tkzDrawLine(A',B')
 \tkzCompass(A',B')
 \tkzDrawSegments(A,B C,D E,F)
 \tkzDrawCircle[R](A',\rCD cm)
 \tkzDrawCircle[R](B',\rEF cm)
 \tkzSetUpLine[color=red]
 \tkzDrawSegments(A',I B',I)
 \tkzDrawPoints(A,B,C,D,E,F,A',B',I,J)
 \tkzLabelPoints(A,B,C,D,E,F,A',B',I,J)
\end{tikzpicture}
\end{tkzexample}


\subsubsection{Example 3: extend lines}
\begin{tkzexample}[latex=7cm,small]
  \begin{tikzpicture}
  \tkzSetUpLine[add=.5 and .5]
  \tkzDefPoints{0/0/A,4/0/B,1/3/C}
  \tkzDrawLines(A,B B,C A,C)
  \end{tikzpicture}
\end{tkzexample}


\subsection{Points style}
\begin{NewMacroBox}{tkzSetUpPoint}{\oarg{local options}}%
\begin{tabular}{lll}%
options &  default & definition                 \\
\midrule
\TOline{color}{black}{point color}
\TOline{size}{3pt}{point size}
\TOline{fill}{black!50}{inside point color}
\TOline{shape}{circle}{point shape circle or cross}
\end{tabular}
\end{NewMacroBox}

\subsubsection{Use of \tkzcname{tkzSetUpPoint}}
\begin{tkzexample}[latex=8cm,small]
\begin{tikzpicture}
  \tkzSetUpPoint[shape = cross out,color=blue]
  \tkzInit[xmax=100,xstep=20,ymax=.5]
  \tkzDefPoint(20,1){A}
  \tkzDefPoint(80,0){B}
  \tkzDrawLine(A,B)
  \tkzDrawPoints(A,B)
\end{tikzpicture}
\end{tkzexample}

\subsubsection{Use of \tkzcname{tkzSetUpPoint} inside a group}
\begin{tkzexample}[latex=8cm,small]
  \begin{tikzpicture}
    \tkzInit[ymin=-0.5,ymax=3,xmin=-0.5,xmax=7]
    \tkzDefPoint(0,0){A}
    \tkzDefPoint(02.25,04.25){B}
    \tkzDefPoint(4,0){C}
    \tkzDefPoint(3,2){D}
    \tkzDrawSegments(A,B A,C A,D)
  {\tkzSetUpPoint[shape=cross out,
              fill= teal!50,
              size=4,color=teal]
    \tkzDrawPoints(A,B)}
    \tkzSetUpPoint[fill= teal!50,size=4,
                 color=teal]
     \tkzDrawPoints(C,D)
    \tkzLabelPoints(A,B,C,D)
  \end{tikzpicture}
\end{tkzexample}



\subsection{Use of \tkzcname{tkzSetUpCompass}}

\begin{NewMacroBox}{tkzSetUpCompass}{\oarg{local options}}%
\begin{tabular}{lll}%
options &  default & definition                 \\
\midrule
\TOline{color}{black}{color of construction arcs}
\TOline{line width}{0.4pt}{thickness of construction arcs}
\TOline{style}{solid}{style of the building arcs}
\end{tabular}
\end{NewMacroBox}

\subsubsection{Use of \tkzcname{tkzSetUpCompass} with bisector}
\begin{tkzexample}[latex=7cm,small]
  \begin{tikzpicture}[scale=0.75]
    \tkzDefPoints{0/1/A, 8/3/B, 3/6/C}
    \tkzDrawPolygon(A,B,C)
    \tkzSetUpCompass[color=red,line width=.2 pt]
    \tkzDefLine[bisector](A,C,B) \tkzGetPoint{c}
    \tkzDefLine[bisector](B,A,C) \tkzGetPoint{a}
    \tkzDefLine[bisector](C,B,A) \tkzGetPoint{b}
    \tkzShowLine[bisector,size=2,gap=3](A,C,B)
    \tkzShowLine[bisector,size=2,gap=3](B,A,C)
    \tkzShowLine[bisector,size=1,gap=2](C,B,A)
    \tkzDrawLines[add=0 and 0 ](B,b)
    \tkzDrawLines[add=0 and -.4 ](A,a  C,c)
    \tkzLabelPoints(A,B) \tkzLabelPoints[above](C)
  \end{tikzpicture}
  \end{tkzexample}

\subsubsection{Another example of of\tkzcname{tkzSetUpCompass}}
\begin{tkzexample}[latex=7cm,small]
  \begin{tikzpicture}[scale=1,rotate=90]
    \tkzDefPoints{0/1/A, 8/3/B, 3/6/C}
    \tkzDrawPolygon(A,B,C)
    \tkzSetUpCompass[color=brown,
            line width=.3 pt,style=tkzdotted]
    \tkzDefLine[bisector](B,A,C)  \tkzGetPoint{a}
    \tkzDefLine[bisector](C,B,A)  \tkzGetPoint{b}
    \tkzInterLL(A,a)(B,b) \tkzGetPoint{I}
    \tkzDefPointBy[projection= onto A--B](I)
    \tkzGetPoint{H}
    \tkzMarkRightAngle(I,H,A)
    \tkzDrawCircle[radius,color=red](I,H)
    \tkzDrawSegments[color=red](I,H)
    \tkzDrawLines[add=0 and -.5,,color=red](A,a)
    \tkzDrawLines[add=0 and 0,color=red](B,b)
    \tkzShowLine[bisector,size=2,gap=3](B,A,C)
    \tkzShowLine[bisector,size=1,gap=3](C,B,A)
    \tkzLabelPoints(A,B)\tkzLabelPoints[left](C)
  \end{tikzpicture}
\end{tkzexample}

\subsection{Own style}
You can set the normal style with |tkzSetUpPoint| and your own style

\begin{tkzexample}[latex=2cm,small]
\tkzSetUpPoint[color=blue!50!white, fill=gray!20!red!50!white]
\tikzset{/tikz/mystyle/.style={color=blue!20!black,fill=blue!20}}
  \begin{tikzpicture}
    \tkzDefPoint(0,0){O}
    \tkzDefPoint(0,1){A}
    \tkzDrawPoints(O) % general style
    \tkzDrawPoints[mystyle,size=4](A) % my style
    \tkzLabelPoints(O,A)
  \end{tikzpicture}
\end{tkzexample}

\endinput
\section{Some examples}
\subsection{Some interesting examples}

\subsubsection{Similar isosceles triangles}

The following is from the excellent site \textbf{Descartes et les Mathématiques}. I did not modify the text and I am only the author of the programming of the figures.

\url{http://debart.pagesperso-orange.fr/seconde/triangle.html}

Bibliography:

\begin{itemize}

\item   Géométrie au Bac - Tangente, special issue no. 8 - Exercise 11, page 11


\item   Elisabeth Busser and Gilles Cohen: 200 nouveaux problèmes du "Monde" - POLE 2007 (200 new problems of "Le Monde")


\item   Affaire de logique n° 364 - Le Monde February 17, 2004
\end{itemize}


Two statements were proposed, one by the magazine \textit{Tangente} and the other by \textit{Le Monde}.

\vspace*{2cm}
\emph{Editor of the magazine "Tangente"}: \textcolor{orange}{Two similar isosceles triangles $AXB$ and $BYC$ are constructed with main vertices $X$ and $Y$, such that $A$, $B$ and $C$ are aligned and that these triangles are "indirect". Let $\alpha$ be the angle at vertex $\widehat{AXB}$ = $\widehat{BYC}$. We then construct a third isosceles triangle $XZY$ similar to the first two, with main vertex $Z$ and "indirect".
We ask to demonstrate that point $Z$ belongs to the straight line $(AC)$.}

\vspace*{2cm}
\emph{Editor of  "Le Monde"}: \textcolor{orange}{We construct two similar isosceles triangles $AXB$ and $BYC$ with principal vertices $X$ and $Y$, such that $A$, $B$ and $C$ are aligned and that these triangles are "indirect". Let $\alpha$ be the angle at vertex $\widehat{AXB}$ = $\widehat{BYC}$. The point Z of the line segment $[AC]$ is equidistant from the two vertices $X$ and $Y$.\\
At what angle does he see these two vertices?}

\vspace*{2cm} The constructions and their associated codes are on the next two pages, but you can search before looking. The programming respects (it seems to me ...) my reasoning in both cases.

 \subsubsection{Revised version of "Tangente"}
\begin{tkzexample}[]
\begin{tikzpicture}[scale=.8,rotate=60]
  \tkzDefPoint(6,0){X}   \tkzDefPoint(3,3){Y}
  \tkzDefShiftPoint[X](-110:6){A}    \tkzDefShiftPoint[X](-70:6){B}
  \tkzDefShiftPoint[Y](-110:4.2){A'} \tkzDefShiftPoint[Y](-70:4.2){B'}
  \tkzDefPointBy[translation= from A' to B ](Y) \tkzGetPoint{Y}
  \tkzDefPointBy[translation= from A' to B ](B') \tkzGetPoint{C}
  \tkzInterLL(A,B)(X,Y) \tkzGetPoint{O}
  \tkzDefMidPoint(X,Y) \tkzGetPoint{I}
  \tkzDefPointWith[orthogonal](I,Y)
  \tkzInterLL(I,tkzPointResult)(A,B) \tkzGetPoint{Z}
  \tkzDefCircle[circum](X,Y,B) \tkzGetPoint{O}
  \tkzDrawCircle(O,X)
  \tkzDrawLines[add = 0 and 1.5](A,C) \tkzDrawLines[add = 0 and 3](X,Y)
  \tkzDrawSegments(A,X B,X B,Y C,Y)   \tkzDrawSegments[color=red](X,Z Y,Z)
  \tkzDrawPoints(A,B,C,X,Y,O,Z)
  \tkzLabelPoints(A,B,C,Z)   \tkzLabelPoints[above right](X,Y,O)
\end{tikzpicture}
\end{tkzexample}

\subsubsection{"Le Monde" version}

\begin{tkzexample}[]
\begin{tikzpicture}[scale=1.25]
  \tkzDefPoint(0,0){A}
  \tkzDefPoint(3,0){B}
  \tkzDefPoint(9,0){C}
  \tkzDefPoint(1.5,2){X}
  \tkzDefPoint(6,4){Y}
  \tkzDefCircle[circum](X,Y,B) \tkzGetPoint{O}
  \tkzDefMidPoint(X,Y)               \tkzGetPoint{I}
  \tkzDefPointWith[orthogonal](I,Y)  \tkzGetPoint{i}
  \tkzDrawLines[add = 2 and 1,color=orange](I,i)
  \tkzInterLL(I,i)(A,B)              \tkzGetPoint{Z}
  \tkzInterLC(I,i)(O,B)              \tkzGetSecondPoint{M}
  \tkzDefPointWith[orthogonal](B,Z)  \tkzGetPoint{b}
  \tkzDrawCircle(O,B)
  \tkzDrawLines[add = 0 and 2,color=orange](B,b)
  \tkzDrawSegments(A,X B,X B,Y C,Y A,C X,Y)
  \tkzDrawSegments[color=red](X,Z Y,Z)
  \tkzDrawPoints(A,B,C,X,Y,Z,M,I)
  \tkzLabelPoints(A,B,C,Z)
  \tkzLabelPoints[above right](X,Y,M,I)
\end{tikzpicture}
\end{tkzexample}

\subsubsection{Triangle altitudes}

The following is again from the excellent site \textbf{Descartes et les Mathématiques} (Descartes and the Mathematics).

\url{http://debart.pagesperso-orange.fr/geoplan/geometrie_triangle.html}

The three altitudes of a triangle intersect at the same H-point.

\begin{tkzexample}[latex=7cm]
\begin{tikzpicture}[scale=.8]
   \tkzDefPoint(0,0){C}
   \tkzDefPoint(7,0){B}
   \tkzDefPoint(5,6){A}
   \tkzDrawPolygon(A,B,C)
   \tkzDefMidPoint(C,B)
   \tkzGetPoint{I}
   \tkzDrawArc(I,B)(C)
   \tkzInterLC(A,C)(I,B)
   \tkzGetSecondPoint{B'}
   \tkzInterLC(A,B)(I,B)
   \tkzGetFirstPoint{C'}
   \tkzInterLL(B,B')(C,C')
   \tkzGetPoint{H}
   \tkzInterLL(A,H)(C,B)
   \tkzGetPoint{A'}
     \tkzDefCircle[circum](A,B',C')
    \tkzGetPoint{O}
   \tkzDrawCircle[color=red](O,A)
   \tkzDrawSegments[color=orange](B,B' C,C' A,A')
   \tkzMarkRightAngles(C,B',B B,C',C C,A',A)
   \tkzDrawPoints(A,B,C,A',B',C',H)
   \tkzLabelPoints(A,B,C,A',B',C',H)
\end{tikzpicture}
\end{tkzexample}

\subsubsection{Altitudes - other construction}

\begin{tkzexample}[latex=7cm]
\begin{tikzpicture}[scale=.75]
  \tkzDefPoint(0,0){A}
  \tkzDefPoint(8,0){B}
  \tkzDefPoint(3.5,10){C}
  \tkzDefMidPoint(A,B)
  \tkzGetPoint{O}
  \tkzDefPointBy[projection=onto A--B](C)
  \tkzGetPoint{P}
  \tkzInterLC(C,A)(O,A)
  \tkzGetSecondPoint{M}
  \tkzInterLC(C,B)(O,A)
  \tkzGetFirstPoint{N}
  \tkzInterLL(B,M)(A,N)
  \tkzGetPoint{I}
  \tkzDrawCircle[diameter](A,B)
  \tkzDrawSegments(C,A C,B A,B B,M A,N)
  \tkzMarkRightAngles[fill=brown!20](A,M,B A,N,B A,P,C)
  \tkzDrawSegment[style=dashed,color=orange](C,P)
  \tkzLabelPoints(O,A,B,P)
  \tkzLabelPoint[left](M){$M$}
  \tkzLabelPoint[right](N){$N$}
  \tkzLabelPoint[above](C){$C$}
  \tkzLabelPoint[above right](I){$I$}
  \tkzDrawPoints[color=red](M,N,P,I)
  \tkzDrawPoints[color=brown](O,A,B,C)
\end{tikzpicture}
\end{tkzexample}

\subsection{Different authors}

\subsubsection{ Square root of the integers}
How to get $1$, $\sqrt{2}$, $\sqrt{3}$ with a rule and a compass.

\begin{tkzexample}[latex=7cm,small]
\begin{tikzpicture}[scale=1.5]
  \tkzDefPoint(0,0){O}
  \tkzDefPoint(1,0){a0}
   \tkzDrawSegment[blue](O,a0)
  \foreach \i [count=\j] in {0,...,10}{%
    \tkzDefPointWith[orthogonal normed](a\i,O)
    \tkzGetPoint{a\j}
    \tkzDrawPolySeg[color=blue](a\i,a\j,O)}
 \end{tikzpicture}
\end{tkzexample}


\subsubsection{About right triangle}

We have a segment $[AB]$ and we want to determine a point $C$ such that $AC=8$~cm    and $ABC$ is a right triangle in $B$.

\begin{tkzexample}[latex=7cm]
\begin{tikzpicture}[scale=.5]
  \tkzDefPoint["$A$" left](2,1){A}
  \tkzDefPoint(6,4){B}
  \tkzDrawSegment(A,B)
  \tkzDrawPoint[color=red](A)
  \tkzDrawPoint[color=red](B)
  \tkzDefPointWith[orthogonal,K=-1](B,A)
  \tkzDrawLine[add = .5 and .5](B,tkzPointResult)
  \tkzInterLC[R](B,tkzPointResult)(A,8 cm)
  \tkzGetPoints{C}{J}
  \tkzDrawPoint[color=red](C)
  \tkzCompass(A,C)
  \tkzMarkRightAngle(A,B,C)
  \tkzDrawLine[color=gray,style=dashed](A,C)
\end{tikzpicture}
\end{tkzexample}


\subsubsection{Archimedes}

This is an ancient problem   proved by the great Greek mathematician Archimedes .
The figure below shows a semicircle, with diameter $AB$. A tangent line is drawn and  touches the semicircle at $B$.   An other tangent line at a point, $C$, on the semicircle is drawn. We project the point $C$ on the line segment $[AB]$  on a point $D$. The two tangent lines intersect at the point $T$.

Prove that the line $(AT)$ bisects $(CD)$

\begin{tkzexample}[]
\begin{tikzpicture}[scale=1.25]
  \tkzDefPoint(0,0){A}\tkzDefPoint(6,0){D}
  \tkzDefPoint(8,0){B}\tkzDefPoint(4,0){I}
  \tkzDefLine[orthogonal=through D](A,D)
  \tkzInterLC[R](D,tkzPointResult)(I,4 cm) \tkzGetFirstPoint{C}
  \tkzDefLine[orthogonal=through C](I,C)    \tkzGetPoint{c}
  \tkzDefLine[orthogonal=through B](A,B)    \tkzGetPoint{b}
  \tkzInterLL(C,c)(B,b) \tkzGetPoint{T}
  \tkzInterLL(A,T)(C,D) \tkzGetPoint{P}
  \tkzDrawArc(I,B)(A)
  \tkzDrawSegments(A,B A,T C,D I,C) \tkzDrawSegment[color=orange](I,C)
  \tkzDrawLine[add = 1 and 0](C,T)   \tkzDrawLine[add = 0 and 1](B,T)
  \tkzMarkRightAngle(I,C,T)
  \tkzDrawPoints(A,B,I,D,C,T)
  \tkzLabelPoints(A,B,I,D)  \tkzLabelPoints[above right](C,T)
  \tkzMarkSegment[pos=.25,mark=s|](C,D) \tkzMarkSegment[pos=.75,mark=s|](C,D)
\end{tikzpicture}
\end{tkzexample}

\subsubsection{Example: Dimitris Kapeta}

You need in this example to use \tkzname{mkpos=.2} with \tkzcname{tkzMarkAngle} because the measure of $ \widehat{CAM}$ is too small.
Another possiblity is to use \tkzcname{tkzFillAngle}.


\begin{tkzexample}[]
\begin{tikzpicture}[scale=1.25]
  \tkzDefPoint(0,0){O}
  \tkzDefPoint(2.5,0){N}
  \tkzDefPoint(-4.2,0.5){M}
  \tkzDefPointBy[rotation=center O angle 30](N)
  \tkzGetPoint{B}
  \tkzDefPointBy[rotation=center O angle -50](N)
  \tkzGetPoint{A}
  \tkzInterLC(M,B)(O,N) \tkzGetFirstPoint{C}
  \tkzInterLC(M,A)(O,N) \tkzGetSecondPoint{A'}
  \tkzMarkAngle[mkpos=.2, size=0.5](A,C,B)
  \tkzMarkAngle[mkpos=.2, size=0.5](A,M,C)
  \tkzDrawSegments(A,C M,A M,B)
  \tkzDrawCircle(O,N)
  \tkzLabelCircle[above left](O,N)(120){$\mathcal{C}$}
  \tkzMarkAngle[mkpos=.2, size=1.2](C,A,M)
  \tkzDrawPoints(O, A, B, M, B, C)
  \tkzLabelPoints[right](O,A,B)
  \tkzLabelPoints[above left](M,C)
  \tkzLabelPoint[below left](A'){$A'$}
\end{tikzpicture}
\end{tkzexample}


\subsubsection{Example 1: John Kitzmiller }

Prove that $\bigtriangleup LKJ$ is equilateral.


\begin{tkzexample}[vbox,small]
\begin{tikzpicture}[scale=2]
  \tkzDefPoint[label=below left:A](0,0){A}
  \tkzDefPoint[label=below right:B](6,0){B}
  \tkzDefTriangle[equilateral](A,B) \tkzGetPoint{C}
  \tkzMarkSegments[mark=|](A,B A,C B,C)
  \tkzDefBarycentricPoint(A=1,B=2) \tkzGetPoint{C'}
  \tkzDefBarycentricPoint(A=2,C=1) \tkzGetPoint{B'}
  \tkzDefBarycentricPoint(C=2,B=1) \tkzGetPoint{A'}
  \tkzInterLL(A,A')(C,C') \tkzGetPoint{J}
  \tkzInterLL(C,C')(B,B') \tkzGetPoint{K}
  \tkzInterLL(B,B')(A,A') \tkzGetPoint{L}
  \tkzLabelPoint[above](C){C}
  \tkzDrawPolygon(A,B,C) \tkzDrawSegments(A,J B,L C,K)
  \tkzMarkAngles[size=1 cm](J,A,C K,C,B L,B,A)
  \tkzMarkAngles[thick,size=1 cm](A,C,J C,B,K B,A,L)
  \tkzMarkAngles[opacity=.5](A,C,J C,B,K B,A,L)
  \tkzFillAngles[fill= orange,size=1 cm,opacity=.3](J,A,C K,C,B L,B,A)
  \tkzFillAngles[fill=orange, opacity=.3,thick,size=1,](A,C,J C,B,K B,A,L)
  \tkzFillAngles[fill=green, size=1, opacity=.5](A,C,J C,B,K B,A,L)
  \tkzFillPolygon[color=yellow, opacity=.2](J,A,C)
  \tkzFillPolygon[color=yellow, opacity=.2](K,B,C)
  \tkzFillPolygon[color=yellow, opacity=.2](L,A,B)
  \tkzDrawSegments[line width=3pt,color=cyan,opacity=0.4](A,J C,K B,L)
  \tkzDrawSegments[line width=3pt,color=red,opacity=0.4](A,L B,K C,J)
  \tkzMarkSegments[mark=o](J,K K,L L,J)
  \tkzLabelPoint[right](J){J}
  \tkzLabelPoint[below](K){K}
  \tkzLabelPoint[above left](L){L}
\end{tikzpicture}
\end{tkzexample}

\subsubsection{Example 2:  John Kitzmiller }
Prove that $\dfrac{AC}{CE}=\dfrac{BD}{DF}$.

Another interesting example from John, you can see how to use some extra options like \tkzname{decoration} and \tkzname{postaction}  from \TIKZ\ with \tkzname{tkz-euclide}.

\begin{tkzexample}[vbox,small]
\begin{tikzpicture}[scale=2,decoration={markings,
  mark=at position 3cm with {\arrow[scale=2]{>}}}]
  \tkzDefPoints{0/0/E, 6/0/F, 0/1.8/P, 6/1.8/Q, 0/3/R, 6/3/S}
  \tkzDrawLines[postaction={decorate}](E,F P,Q R,S)
  \tkzDefPoints{3.5/3/A, 5/3/B}
  \tkzDrawSegments(E,A F,B)
  \tkzInterLL(E,A)(P,Q) \tkzGetPoint{C}
  \tkzInterLL(B,F)(P,Q) \tkzGetPoint{D}
  \tkzLabelPoints[above right](A,B)
  \tkzLabelPoints[below](E,F)
  \tkzLabelPoints[above left](C)
  \tkzDrawSegments[style=dashed](A,F)
  \tkzInterLL(A,F)(P,Q) \tkzGetPoint{G}
  \tkzLabelPoints[above right](D,G)
  \tkzDrawSegments[color=teal, line width=3pt, opacity=0.4](A,C A,G)
  \tkzDrawSegments[color=magenta, line width=3pt, opacity=0.4](C,E G,F)
  \tkzDrawSegments[color=teal, line width=3pt, opacity=0.4](B,D)
  \tkzDrawSegments[color=magenta, line width=3pt, opacity=0.4](D,F)
\end{tikzpicture}
\end{tkzexample}

\subsubsection{Example 3:  John Kitzmiller }
Prove that $\dfrac{BC}{CD}=\dfrac{AB}{AD} \qquad$ (Angle Bisector).

\begin{tkzexample}[vbox,small]
\begin{tikzpicture}[scale=2]
  \tkzDefPoints{0/0/B, 5/0/D}       \tkzDefPoint(70:3){A}
  \tkzDrawPolygon(B,D,A)
  \tkzDefLine[bisector](B,A,D)      \tkzGetPoint{a}
  \tkzInterLL(A,a)(B,D)           \tkzGetPoint{C}
  \tkzDefLine[parallel=through B](A,C) \tkzGetPoint{b}
  \tkzInterLL(A,D)(B,b)           \tkzGetPoint{P}
  \begin{scope}[decoration={markings,
   mark=at position .5 with {\arrow[scale=2]{>}}}]
   \tkzDrawSegments[postaction={decorate},dashed](C,A P,B)
  \end{scope}
  \tkzDrawSegment(A,C) \tkzDrawSegment[style=dashed](A,P)
  \tkzLabelPoints[below](B,C,D) \tkzLabelPoints[above](A,P)
  \tkzDrawSegments[color=magenta, line width=3pt, opacity=0.4](B,C P,A)
  \tkzDrawSegments[color=teal,    line width=3pt, opacity=0.4](C,D A,D)
  \tkzDrawSegments[color=magenta, line width=3pt, opacity=0.4](A,B)
  \tkzMarkAngles[size=3mm](B,A,C C,A,D)
  \tkzMarkAngles[size=3mm](B,A,C A,B,P)
  \tkzMarkAngles[size=3mm](B,P,A C,A,D)
  \tkzMarkAngles[size=3mm](B,A,C A,B,P B,P,A C,A,D)
  \tkzFillAngles[fill=green,  opacity=0.5](B,A,C A,B,P)
  \tkzFillAngles[fill=yellow, opacity=0.3](B,P,A C,A,D)
  \tkzFillAngles[fill=green,  opacity=0.6](B,A,C A,B,P B,P,A C,A,D)
  \tkzLabelAngle[pos=1](B,A,C){1}   \tkzLabelAngle[pos=1](C,A,D){2}
  \tkzLabelAngle[pos=1](A,B,P){3}    \tkzLabelAngle[pos=1](B,P,A){4}
  \tkzMarkSegments[mark=|](A,B A,P)
\end{tikzpicture}
\end{tkzexample}


\subsubsection{Example 4: author John Kitzmiller }
Prove that $\overline{AG}\cong\overline{EF} \qquad$ (Detour).

\begin{tkzexample}[vbox,small]
\begin{tikzpicture}[scale=2]
  \tkzDefPoint(0,3){A}    \tkzDefPoint(6,3){E}  \tkzDefPoint(1.35,3){B}
  \tkzDefPoint(4.65,3){D} \tkzDefPoint(1,1){G}  \tkzDefPoint(5,5){F}
  \tkzDefMidPoint(A,E)    \tkzGetPoint{C}
  \tkzFillPolygon[yellow, opacity=0.4](B,G,C)
  \tkzFillPolygon[yellow, opacity=0.4](D,F,C)
  \tkzFillPolygon[blue, opacity=0.3](A,B,G)
  \tkzFillPolygon[blue, opacity=0.3](E,D,F)
  \tkzMarkAngles[size=0.5 cm](B,G,A D,F,E)
  \tkzMarkAngles[size=0.5 cm](B,C,G D,C,F)
  \tkzMarkAngles[size=0.5 cm](G,B,C F,D,C)
  \tkzMarkAngles[size=0.5 cm](A,B,G E,D,F)
  \tkzFillAngles[size=0.5 cm,fill=green](B,G,A D,F,E)
  \tkzFillAngles[size=0.5 cm,fill=orange](B,C,G D,C,F)
  \tkzFillAngles[size=0.5 cm,fill=yellow](G,B,C F,D,C)
  \tkzFillAngles[size=0.5 cm,fill=red](A,B,G E,D,F)
  \tkzMarkSegments[mark=|](B,C D,C)  \tkzMarkSegments[mark=s||](G,C F,C)
  \tkzMarkSegments[mark=o](A,G E,F)  \tkzMarkSegments[mark=s](B,G D,F)
  \tkzDrawSegment[color=red](A,E)
  \tkzDrawSegment[color=blue](F,G)
  \tkzDrawSegments(A,G G,B E,F F,D)
  \tkzLabelPoints[below](C,D,E,G)  \tkzLabelPoints[above](A,B,F)
\end{tikzpicture}
\end{tkzexample}

\subsubsection{Example 1: from Indonesia}

\begin{tkzexample}[vbox,small]
\begin{tikzpicture}[scale=3]
   \tkzDefPoints{0/0/A,2/0/B}
   \tkzDefSquare(A,B) \tkzGetPoints{C}{D}
   \tkzDefPointBy[rotation=center D angle 45](C)\tkzGetPoint{G}
   \tkzDefSquare(G,D)\tkzGetPoints{E}{F}
   \tkzInterLL(B,C)(E,F)\tkzGetPoint{H}
   \tkzFillPolygon[gray!10](D,E,H,C,D)
   \tkzDrawPolygon(A,...,D)\tkzDrawPolygon(D,...,G)
   \tkzDrawSegment(B,E)
   \tkzMarkSegments[mark=|,size=3pt,color=gray](A,B B,C C,D D,A E,F F,G G,D D,E)
   \tkzMarkSegments[mark=||,size=3pt,color=gray](B,E E,H)
   \tkzLabelPoints[left](A,D)
   \tkzLabelPoints[right](B,C,F,H)
   \tkzLabelPoints[above](G)\tkzLabelPoints[below](E)
   \tkzMarkRightAngles(D,A,B D,G,F)
\end{tikzpicture}
\end{tkzexample}

\subsubsection{Example 2: from Indonesia}
\begin{tkzexample}[vbox,small]
  \begin{tikzpicture}[pol/.style={fill=brown!40,opacity=.5},
                     seg/.style={tkzdotted,color=gray},
                     hidden pt/.style={fill=gray!40},
                     mra/.style={color=gray!70,tkzdotted,/tkzrightangle/size=.2},
                     scale=3]
  \tkzSetUpPoint[size=2]                
  \tkzDefPoints{0/0/A,2.5/0/B,1.33/0.75/D,0/2.5/E,2.5/2.5/F}
  \tkzDefLine[parallel=through D](A,B) \tkzGetPoint{I1}
  \tkzDefLine[parallel=through B](A,D) \tkzGetPoint{I2}
  \tkzInterLL(D,I1)(B,I2) \tkzGetPoint{C}
  \tkzDefLine[parallel=through E](A,D) \tkzGetPoint{I3}
  \tkzDefLine[parallel=through D](A,E) \tkzGetPoint{I4}
  \tkzInterLL(E,I3)(D,I4) \tkzGetPoint{H}
  \tkzDefLine[parallel=through F](E,H) \tkzGetPoint{I5}
  \tkzDefLine[parallel=through H](E,F) \tkzGetPoint{I6}
  \tkzInterLL(F,I5)(H,I6) \tkzGetPoint{G}
  \tkzDefMidPoint(G,H) \tkzGetPoint{P}
  \tkzDefMidPoint(G,C) \tkzGetPoint{Q}
  \tkzDefMidPoint(B,C) \tkzGetPoint{R}
  \tkzDefMidPoint(A,B) \tkzGetPoint{S}
  \tkzDefMidPoint(A,E) \tkzGetPoint{T}
  \tkzDefMidPoint(E,H) \tkzGetPoint{U}
  \tkzDefMidPoint(A,D) \tkzGetPoint{M}
  \tkzDefMidPoint(D,C) \tkzGetPoint{N}
  \tkzInterLL(B,D)(S,R) \tkzGetPoint{L}
  \tkzInterLL(H,F)(U,P) \tkzGetPoint{K}
  \tkzDefLine[parallel=through K](D,H) \tkzGetPoint{I7}
  \tkzInterLL(K,I7)(B,D) \tkzGetPoint{O}
  
  \tkzFillPolygon[pol](P,Q,R,S,T,U)
  \tkzDrawSegments[seg](K,O K,L P,Q R,S T,U 
                    C,D H,D A,D M,N B,D)
  \tkzDrawSegments(E,H B,C G,F G,H G,C Q,R S,T U,P H,F)
  \tkzDrawPolygon(A,B,F,E)
  \tkzDrawPoints(A,B,C,E,F,G,H,P,Q,R,S,T,U,K)
  \tkzDrawPoints[hidden pt](M,N,O,D)
  \tkzMarkRightAngle[mra](L,O,K)
  \tkzMarkSegments[mark=|,size=1pt,thick,color=gray](A,S B,S B,R C,R 
                    Q,C Q,G G,P H,P 
                    E,U H,U E,T A,T)
  
  \tkzLabelAngle[pos=.3](K,L,O){$\alpha$}
  \tkzLabelPoints[below](O,A,S,B)
  \tkzLabelPoints[above](H,P,G)
  \tkzLabelPoints[left](T,E)
  \tkzLabelPoints[right](C,Q)
  \tkzLabelPoints[above left](U,D,M)
  \tkzLabelPoints[above right](L,N)
  \tkzLabelPoints[below right](F,R)
  \tkzLabelPoints[below left](K)
  \end{tikzpicture}
\end{tkzexample}


\subsubsection{Three circles}

\begin{tkzexample}[vbox,small]
\begin{tikzpicture}[scale=1.5]
  \tkzDefPoints{0/0/A,8/0/B,0/4/a,8/4/b,8/8/c}
  \tkzDefTriangle[equilateral](A,B) \tkzGetPoint{C}
  \tkzDrawPolygon(A,B,C)
  \tkzDefSquare(A,B) \tkzGetPoints{D}{E}
  \tkzClipBB
  \tkzDefMidPoint(A,B) \tkzGetPoint{M}
  \tkzDefMidPoint(B,C) \tkzGetPoint{N}
  \tkzDefMidPoint(A,C) \tkzGetPoint{P}
  \tkzDrawSemiCircle[gray,dashed](M,B)
  \tkzDrawSemiCircle[gray,dashed](A,M)
  \tkzDrawSemiCircle[gray,dashed](A,B)
  \tkzDrawCircle[gray,dashed](B,A)
  \tkzInterLL(A,N)(M,a) \tkzGetPoint{Ia}
  \tkzDefPointBy[projection = onto A--B](Ia)
  \tkzGetPoint{ha}
  \tkzDrawCircle[gray](Ia,ha)
  \tkzInterLL(B,P)(M,b) \tkzGetPoint{Ib}
  \tkzDefPointBy[projection = onto A--B](Ib)
  \tkzGetPoint{hb}
  \tkzDrawCircle[gray](Ib,hb)
  \tkzInterLL(A,c)(M,C) \tkzGetPoint{Ic}
  \tkzDefPointBy[projection = onto A--C](Ic)
  \tkzGetPoint{hc}
  \tkzDrawCircle[gray](Ic,hc)
  \tkzInterLL(A,Ia)(B,Ib) \tkzGetPoint{G}
  \tkzDrawCircle[gray,dashed](G,Ia)
  \tkzDrawPolySeg(A,E,D,B)
  \tkzDrawPoints(A,B,C)
  \tkzDrawPoints(G,Ia,Ib,Ic)
  \tkzDrawSegments[gray,dashed](C,M A,N B,P M,a M,b A,a a,b b,B A,D Ia,ha)
\end{tikzpicture}
\end{tkzexample}

\subsubsection{"The" Circle of APOLLONIUS}

\begin{tkzexample}[vbox,small]
  \begin{tikzpicture}[scale=.5]
  \tkzDefPoints{0/0/A,6/0/B,0.8/4/C}
  \tkzDefTriangleCenter[euler](A,B,C)        \tkzGetPoint{N} 
  \tkzDefTriangleCenter[circum](A,B,C)       \tkzGetPoint{O} 
  \tkzDefTriangleCenter[lemoine](A,B,C)      \tkzGetPoint{K} 
  \tkzDefTriangleCenter[spieker](A,B,C)      \tkzGetPoint{Sp}
  \tkzDefExCircle(A,B,C)     \tkzGetPoint{Jb}
  \tkzDefExCircle(C,A,B)     \tkzGetPoint{Ja}
  \tkzDefExCircle(B,C,A)     \tkzGetPoint{Jc}
  \tkzDefPointBy[projection=onto B--C ](Jc)   \tkzGetPoint{Xc}
  \tkzDefPointBy[projection=onto B--C ](Jb)   \tkzGetPoint{Xb}
  \tkzDefPointBy[projection=onto A--B ](Ja)   \tkzGetPoint{Za}
  \tkzDefPointBy[projection=onto A--B ](Jb)   \tkzGetPoint{Zb}
  \tkzDefLine[parallel=through Xc](A,C)       \tkzGetPoint{X'c}
  \tkzDefLine[parallel=through Xb](A,B)       \tkzGetPoint{X'b}
  \tkzDefLine[parallel=through Za](C,A)       \tkzGetPoint{Z'a}
  \tkzDefLine[parallel=through Zb](C,B)       \tkzGetPoint{Z'b}
  \tkzInterLL(Xc,X'c)(A,B)                    \tkzGetPoint{B'}
  \tkzInterLL(Xb,X'b)(A,C)                    \tkzGetPoint{C'}
  \tkzInterLL(Za,Z'a)(C,B)                    \tkzGetPoint{A''}
  \tkzInterLL(Zb,Z'b)(C,A)                    \tkzGetPoint{B''}
  \tkzDefPointBy[reflection= over Jc--Jb](B') \tkzGetPoint{Ca}
  \tkzDefPointBy[reflection= over Jc--Jb](C') \tkzGetPoint{Ba}
  \tkzDefPointBy[reflection= over Ja--Jb](A'')\tkzGetPoint{Bc}
  \tkzDefPointBy[reflection= over Ja--Jb](B'')\tkzGetPoint{Ac}
  \tkzDefCircle[circum](Ac,Ca,Ba)             \tkzGetPoint{Q}
  \tkzDrawCircle[circum](Ac,Ca,Ba)
  \tkzDefPointWith[linear,K=1.1](Q,Ac)        \tkzGetPoint{nAc}
  \tkzClipCircle[through](Q,nAc)
  \tkzDrawLines[add=1.5 and 1.5,dashed](A,B B,C A,C)
  \tkzDrawPolygon[color=blue](A,B,C)
  \tkzDrawPolygon[dashed,color=blue](Ja,Jb,Jc)
  \tkzDrawCircles[ex](A,B,C B,C,A C,A,B) 
  \tkzDrawLines[add=0 and 0,dashed](Ca,Bc B,Za A,Ba B',C')
  \tkzDrawLine[add=1 and 1,dashed](Xb,Xc)
  \tkzDrawLine[add=7 and 3,blue](O,K)
  \tkzDrawLine[add=8 and 15,red](N,Sp)
  \tkzDrawLines[add=10 and 10](K,O N,Sp)
  \tkzDrawSegments(Ba,Ca Bc,Ac)
  \tkzDrawPoints(A,B,C,N,Ja,Jb,Jc,Xb,Xc,B',C',Za,Zb,Ba,Ca,Bc,Ac,Q,Sp,K,O)
  \tkzLabelPoints(A,B,C,N,Ja,Jb,Jc,Xb,Xc,B',C',Za,Zb,Ba,Ca,Bc,Ac,Q,Sp)
  \tkzLabelPoints[above](K,O)
  \end{tikzpicture}
\end{tkzexample}


 
\endinput

%!TEX root = /Users/ego/Boulot/TKZ/tkz-euclide/doc_fr/TKZdoc-euclide-main.tex

\section{Gallery  : Some examples}

Some examples with explanations in english.
%–––––––––––––––––––––––––––––––––––––––––––––––––––––––––––––––––––––––––––>

\subsection{White on Black}
This example shows how to get a segment with a length equal at $\sqrt{a}$ from a segment of length $a$, only with a rule and a compass.


\begin{center}
\begin{tkzexample}[]
  \tikzset{background rectangle/.style={fill=black}} 
\begin{tikzpicture}[show background rectangle]
   \tkzInit[ymin=-1.5,ymax=7,xmin=-1,xmax=+11]
   \tkzClip 
   \tkzDefPoint(0,0){O}
   \tkzDefPoint(1,0){I}
   \tkzDefPoint(10,0){A}
   \tkzDefPointWith[orthogonal](I,A) \tkzGetPoint{H}
   \tkzDefMidPoint(O,A) \tkzGetPoint{M}
   \tkzInterLC(I,H)(M,A)\tkzGetPoints{C}{B}
   \tkzDrawSegments[color=white,line width=1pt](I,H O,A)
   \tkzDrawPoints[color=white](O,I,A,B,M) 
   \tkzMarkRightAngle[color=white,line width=1pt](A,I,B) 
   \tkzDrawArc[color=white,line width=1pt,style=dashed](M,A)(O) 
  \tkzLabelSegment[white,right=1ex,pos=.5](I,B){$\sqrt{a}$} 
  \tkzLabelSegment[white,below=1ex,pos=.5](O,I){$1$}   
  \tkzLabelSegment[pos=.6,white,below=1ex](I,A){$a$} 
\end{tikzpicture}
\end{tkzexample}
\end{center}

\vfill\newpage
%<–––––––––––––––––––––––––––––––––––––––––––––––––––––––––––––––––––––––––––>

\subsection{ Square root of the integers }       
How to get $1$, $\sqrt{2}$, $\sqrt{3}$ with a rule and a compass.
\begin{center}
\begin{tkzexample}[]
\begin{tikzpicture}[scale=1.75]
   \tkzInit[xmin=-3,xmax=4,ymin=-2,ymax=4]
   \tkzGrid
   \tkzDefPoint(0,0){O}
   \tkzDefPoint(1,0){a0}
   \newcounter{tkzcounter}
   \setcounter{tkzcounter}{0}
   \newcounter{density}
   \setcounter{density}{20}
   \foreach \i in {0,...,15}{%
      \pgfmathsetcounter{density}{\thedensity+2}
      \setcounter{density}{\thedensity}    
      \stepcounter{tkzcounter}
      \tkzDefPointWith[orthogonal normed](a\i,O)
      \tkzGetPoint{a\thetkzcounter}
      \tkzDrawPolySeg[color=Maroon!\thedensity,%
         fill=Maroon!\thedensity,opacity=.5](a\i,a\thetkzcounter,O)}
 \end{tikzpicture}  
\end{tkzexample}
\end{center}

%<–––––––––––––––––––––––––––––––––––––––––––––––––––––––––––––––––––––––––––>
 \vfill\newpage
%<–––––––––––––––––––––––––––––––––––––––––––––––––––––––––––––––––––––––––––>
% 
\subsection{How to construct the tangent lines from a point to a circle with a rule and a compass.}
\begin{center}
\begin{tkzexample}[] 
  \begin{tikzpicture}
    \tkzPoint(0,0){O}
    \tkzPoint(9,2){P}
    \tkzDefMidPoint(O,P) \tkzGetPoint{I}
    \tkzDrawCircle[R](O,4cm)
    \tkzDrawCircle[diameter](O,P)
    \tkzCalcLength(I,P)  \tkzGetLength{dIP}
    \tkzInterCC[R](O,4cm)(I,\dIP pt)\tkzGetPoints{Q1}{Q2}
    \tkzDrawPoint[color=red](Q1)
    \tkzDrawPoint[color=red](Q2)
    \tkzDrawLine(P,Q1) 
    \tkzDrawLine(P,Q2) 
    \tkzDrawSegments(O,Q1 O,Q2)
    \tkzDrawLine(P,O)
\end{tikzpicture}
\end{tkzexample}
\end{center} 
% 
% %<–––––––––––––––––––––––––––––––––––––––––––––––––––––––––––––––––––––––––––>
 \vfill\newpage
%<–––––––––––––––––––––––––––––––––––––––––––––––––––––––––––––––––––––––––––>

\subsection{Circle and tangent}
We have a point A $(8,2)$, a circle with center A and radius=3cm and a line
  $\delta$ $y=4$. The line intercepts the circle at B. We want to draw the tangent at the circle in B.
   
\begin{center}
\begin{tkzexample}[]
\begin{tikzpicture}
  \tkzInit[xmax=14,ymin=-2,ymax=6]
  \tkzDrawX[noticks,label=$(d)$]
  \tkzPoint[pos=above right](8,2){A};
  \tkzPoint[color=red,pos=above right](0,0){O};
  \tkzDrawCircle[R,color=blue,line width=.8pt](A,3 cm)
  \tkzHLine[color=red,style=dashed]{4} 
  \tkzText[above](12,4){$\delta$}
  \FPeval\alphaR{arcsin(2/3)}% on a les bonnes valeurs
  \FPeval\xB{8-3*cos(\alphaR)}
  \tkzPoint[pos=above left](\xB,4){B};
  \tkzDrawSegment[line width=1pt](A,B)
  \tkzDefLine[orthogonal=through B](A,B) \tkzGetPoint{b}
  \tkzDefPoint(1,0){i}
  \tkzInterLL(B,b)(O,i) \tkzGetPoint{B'}
  \tkzDrawPoint(B')
  \tkzDrawLine(B,B')
 \end{tikzpicture}
\end{tkzexample}
\end{center}

 \vfill\newpage
%<–––––––––––––––––––––––––––––––––––––––––––––––––––––––––––––––––––––––––––>

\subsection{About right triangle}

We have a segment $[AB]$ and we want to determine a point $C$ such as $AC=8 cm$ and $ABC$ is a right triangle in $B$.

\begin{center}
\begin{tkzexample}[]
\begin{tikzpicture}
  \tkzInit
  \tkzClip
  \tkzPoint[pos=left](2,1){A}
  \tkzPoint(6,4){B} 
  \tkzDrawSegment(A,B)
  \tkzDrawPoint[color=red](A)
  \tkzDrawPoint[color=red](B)
  \tkzDefPointWith[orthogonal,K=-1](B,A)    
  \tkzDrawLine[add = .5 and .5](B,tkzPointResult)
  \tkzInterLC[R](B,tkzPointResult)(A,8 cm) \tkzGetPoints{C}{J}
  \tkzDrawPoint[color=red](C)
  \tkzCompass(A,C)
  \tkzMarkRightAngle(A,B,C)
  \tkzDrawLine[color=gray,style=dashed](A,C)
\end{tikzpicture} 
\end{tkzexample}
\end{center}

 %<–––––––––––––––––––––––––––––––––––––––––––––––––––––––––––––––––––––––––––>
 \vfill\newpage %<–––––––––––––––––––––––––––––––––––––––––––––––––––––––––––––––––––––––––––>

\subsection{Archimedes}

This is an ancient problem  proved by the great Greek mathematician Archimedes .
The figure below shows a semicircle, with diameter $AB$. A tangent line is drawn and  touches the semicircle at $B$.  An other tangent line at a point, $C$, on the semicircle is drawn. We project the point $C$ on the segment$[AB]$  on a point $D$ . The two tangent lines intersect at the point $T$.

Prove that the line $(AT)$ bisects $(CD)$

\begin{center}
\begin{tkzexample}[]  
\begin{tikzpicture}[scale=1.25] 
   \tkzInit[ymin=-1,ymax=7]
   \tkzClip
   \tkzDefPoint(0,0){A}\tkzDefPoint(6,0){D} 
   \tkzDefPoint(8,0){B}\tkzDefPoint(4,0){I}
   \tkzDefLine[orthogonal=through D](A,D)
   \tkzInterLC[R](D,tkzPointResult)(I,4 cm) \tkzGetFirstPoint{C}
   \tkzDefLine[orthogonal=through C](I,C)   \tkzGetPoint{c}
   \tkzDefLine[orthogonal=through B](A,B)   \tkzGetPoint{b}
   \tkzInterLL(C,c)(B,b) \tkzGetPoint{T} 
   \tkzInterLL(A,T)(C,D) \tkzGetPoint{P}
   \tkzDrawArc(I,B)(A) 
   \tkzDrawSegments(A,B A,T C,D I,C) \tkzDrawSegment[color=orange](I,C)
   \tkzDrawLine[add = 1 and 0](C,T)  \tkzDrawLine[add = 0 and 1](B,T)
   \tkzMarkRightAngle(I,C,T)
   \tkzDrawPoints(A,B,I,D,C,T)  
   \tkzLabelPoints(A,B,I,D)  \tkzLabelPoints[above right](C,T)
   \tkzMarkSegment[pos=.25,mark=s|](C,D) \tkzMarkSegment[pos=.75,mark=s|](C,D)
\end{tikzpicture}  
\end{tkzexample}
\end{center}  

\subsection{Example from Dimitris Kapeta}

You need in this example to use \tkzname{mkpos=.2} with \tkzcname{tkzMarkAngle} because the measure of $ \widehat{CAM}$ is too small.
Another possiblity is to use \tkzcname{tkzFillAngle}.

\begin{center}
\begin{tkzexample}[]
\begin{tikzpicture}[scale=1.25]
  \tkzInit[xmin=-5.2,xmax=3.2,ymin=-3.2,ymax=3.3]
  \tkzClip 
  \tkzDefPoint(0,0){O}
  \tkzDefPoint(2.5,0){N}
  \tkzDefPoint(-4.2,0.5){M}
  \tkzDefPointBy[rotation=center O angle 30](N)
  \tkzGetPoint{B}
  \tkzDefPointBy[rotation=center O angle -50](N)
  \tkzGetPoint{A}
  \tkzInterLC(M,B)(O,N) \tkzGetFirstPoint{C}
  \tkzInterLC(M,A)(O,N) \tkzGetSecondPoint{A'} 
  \tkzMarkAngle[fill=blue!25,mkpos=.2, size=0.5](A,C,B) 
  \tkzMarkAngle[fill=green!25,mkpos=.2, size=0.5](A,M,C)
  \tkzDrawSegments(A,C M,A M,B)
  \tkzDrawCircle(O,N)
  \tkzLabelCircle[above left](O,N)(120){$\mathcal{C}$}
  \tkzMarkAngle[fill=red!25,mkpos=.2, size=0.5cm](C,A,M)
  \tkzDrawPoints(O, A, B, M, B, C)
  \tkzLabelPoints[right](O,A,B)
  \tkzLabelPoints[above left](M,C)
  \tkzLabelPoint[below left](A'){$A'$}
\end{tikzpicture}
\end{tkzexample}
\end{center}

\newpage
\subsection{Example 1 from John Kitzmiller }
This figure is the last of beamer document. You can find the document on  my site 

Prove $\bigtriangleup LKJ$ is equilateral
  
\begin{center}
\begin{tkzexample}[vbox]
\begin{tikzpicture}[scale=1.5]
  \tkzDefPoint[label=below left:A](0,0){A}
  \tkzDefPoint[label=below right:B](6,0){B}
  \tkzDefTriangle[equilateral](A,B) \tkzGetPoint{C}
  \tkzMarkSegments[mark=|](A,B A,C B,C)
  \tkzDefBarycentricPoint(A=1,B=2) \tkzGetPoint{C'}
  \tkzDefBarycentricPoint(A=2,C=1) \tkzGetPoint{B'}
  \tkzDefBarycentricPoint(C=2,B=1) \tkzGetPoint{A'}
  \tkzInterLL(A,A')(C,C') \tkzGetPoint{J}
  \tkzInterLL(C,C')(B,B') \tkzGetPoint{K}
  \tkzInterLL(B,B')(A,A') \tkzGetPoint{L}
  \tkzLabelPoint[above](C){C}
  \tkzDrawPolygon(A,B,C) \tkzDrawSegments(A,J B,L C,K)
  \tkzMarkAngles[fill= orange,size=1cm,opacity=.3](J,A,C K,C,B L,B,A)
  \tkzLabelPoint[right](J){J}
  \tkzLabelPoint[below](K){K}
  \tkzLabelPoint[above left](L){L}
  \tkzMarkAngles[fill=orange, opacity=.3,thick,size=1,](A,C,J C,B,K B,A,L)
  \tkzMarkAngles[fill=green, size=1, opacity=.5](A,C,J C,B,K B,A,L)
  \tkzFillPolygon[color=yellow, opacity=.2](J,A,C)
  \tkzFillPolygon[color=yellow, opacity=.2](K,B,C)
  \tkzFillPolygon[color=yellow, opacity=.2](L,A,B)
  \tkzDrawSegments[line width=3pt,color=cyan,opacity=0.4](A,J C,K B,L)
  \tkzDrawSegments[line width=3pt,color=red,opacity=0.4](A,L B,K C,J)
  \tkzMarkSegments[mark=o](J,K K,L L,J)
\end{tikzpicture}  
\end{tkzexample}  

\end{center} 

\newpage
\subsection{Example 2 from John Kitzmiller }    
Prove $\dfrac{AC}{CE}=\dfrac{BD}{DF} \qquad$

Another interesting example from John, you can see how to use some extra options like \tkzname{decoration} and \tkzname{postaction}  from \TIKZ\ with \tkzname{tkz-euclide}.

\begin{center}
\begin{tkzexample}[vbox]
\begin{tikzpicture}[scale=1.5,decoration={markings,
  mark=at position 3cm with {\arrow[scale=2]{>}};}]
  \tkzInit[xmin=-0.25,xmax=6.25, ymin=-0.5,ymax=4]
  \tkzClip
  \tkzDefPoints{0/0/E, 6/0/F, 0/1.8/P, 6/1.8/Q, 0/3/R, 6/3/S}
  \tkzDrawLines[postaction={decorate}](E,F P,Q R,S)
  \tkzDefPoints{3.5/3/A, 5/3/B}
  \tkzDrawSegments(E,A F,B)
  \tkzInterLL(E,A)(P,Q) \tkzGetPoint{C}
  \tkzInterLL(B,F)(P,Q) \tkzGetPoint{D}
  \tkzLabelPoints[above right](A,B)
  \tkzLabelPoints[below](E,F)
  \tkzLabelPoints[above left](C)
  \tkzDrawSegments[style=dashed](A,F)
  \tkzInterLL(A,F)(P,Q) \tkzGetPoint{G}
  \tkzLabelPoints[above right](D,G) 
  \tkzDrawSegments[color=teal, line width=3pt, opacity=0.4](A,C A,G)
  \tkzDrawSegments[color=magenta, line width=3pt, opacity=0.4](C,E G,F)
  \tkzDrawSegments[color=teal, line width=3pt, opacity=0.4](B,D)
  \tkzDrawSegments[color=magenta, line width=3pt, opacity=0.4](D,F)
\end{tikzpicture} 
\end{tkzexample}
\end{center}

\newpage
\subsection{Example 3 from John Kitzmiller }    
Prove $\dfrac{BC}{CD}=\dfrac{AB}{AD} \qquad$ (Angle Bisector)


\begin{center}
\begin{tkzexample}[vbox]
\begin{tikzpicture}[scale=1.5] 
  \tkzInit[xmin=-4,xmax=5,ymax=4.5]   \tkzClip[space=.5]
  \tkzDefPoints{0/0/B, 5/0/D}         \tkzDefPoint(70:3){A}
  \tkzDrawPolygon(B,D,A)
  \tkzDefLine[bisector](B,A,D)         \tkzGetPoint{a}
  \tkzInterLL(A,a)(B,D)                \tkzGetPoint{C}
  \tkzDefLine[parallel=through B](A,C) \tkzGetPoint{b}
  \tkzInterLL(A,D)(B,b)                \tkzGetPoint{P}
  \begin{scope}[decoration={markings,
    mark=at position .5 with {\arrow[scale=2]{>}};}]
    \tkzDrawSegments[postaction={decorate},dashed](C,A P,B) 
  \end{scope}
  \tkzDrawSegment(A,C) \tkzDrawSegment[style=dashed](A,P)  
  \tkzLabelPoints[below](B,C,D) \tkzLabelPoints[above](A,P)
  \tkzDrawSegments[color=magenta, line width=3pt, opacity=0.4](B,C P,A)
  \tkzDrawSegments[color=teal,     line width=3pt, opacity=0.4](C,D A,D)
  \tkzDrawSegments[color=magenta, line width=3pt, opacity=0.4](A,B)
  \tkzMarkAngles[size=0.7](B,A,C C,A,D)
  \tkzMarkAngles[size=0.7, fill=green,  opacity=0.5](B,A,C A,B,P)
  \tkzMarkAngles[size=0.7, fill=yellow, opacity=0.3](B,P,A C,A,D)
  \tkzMarkAngles[size=0.7, fill=green,  opacity=0.6](B,A,C A,B,P B,P,A C,A,D)
  \tkzLabelAngle[pos=1](B,A,C){1}     \tkzLabelAngle[pos=1](C,A,D){2}
  \tkzLabelAngle[pos=1](A,B,P){3})    \tkzLabelAngle[pos=1](B,P,A){4}
  \tkzMarkSegments[mark=|](A,B A,P) 
\end{tikzpicture}   
\end{tkzexample}
\end{center} 

\newpage
\subsection{Example 4 from John Kitzmiller }    
Prove $\overline{AG}\cong\overline{EF} \qquad$ (Detour)

\begin{center}
\begin{tkzexample}[vbox]
\begin{tikzpicture}[scale=2]
  \tkzInit[xmax=5, ymax=5]
  \tkzDefPoint(0,3){A}    \tkzDefPoint(6,3){E}  \tkzDefPoint(1.35,3){B}
  \tkzDefPoint(4.65,3){D} \tkzDefPoint(1,1){G}  \tkzDefPoint(5,5){F} 
  \tkzDefMidPoint(A,E)    \tkzGetPoint{C}       
  \tkzFillPolygon[yellow, opacity=0.4](B,G,C)
  \tkzFillPolygon[yellow, opacity=0.4](D,F,C)
  \tkzFillPolygon[blue, opacity=0.3](A,B,G)
  \tkzFillPolygon[blue, opacity=0.3](E,D,F)
  \tkzMarkAngles[size=0.6,fill=green](B,G,A D,F,E)
  \tkzMarkAngles[size=0.6,fill=orange](B,C,G D,C,F)
  \tkzMarkAngles[size=0.6,fill=yellow](G,B,C F,D,C)
  \tkzMarkAngles[size=0.6,fill=red](A,B,G E,D,F)
  \tkzMarkSegments[mark=|](B,C D,C)  \tkzMarkSegments[mark=s||](G,C F,C)
  \tkzMarkSegments[mark=o](A,G E,F)  \tkzMarkSegments[mark=s](B,G D,F)
  \tkzDrawSegment[color=red](A,E)
  \tkzDrawSegment[color=blue](F,G)
  \tkzDrawSegments(A,G G,B E,F F,D) 
  \tkzLabelPoints[below](C,D,E,G)    \tkzLabelPoints[above](A,B,F)  
\end{tikzpicture}
\end{tkzexample}
\end{center}   
\endinput
%!TEX root = /Users/ego/Boulot/TKZ/tkz-euclide/doc_fr/TKZdoc-euclide-main.tex

\section{FAQ} 
\subsection{Erreurs les plus fréquentes}
 Je me base pour le moment sur les miennes, car ayant changé plusieurs fois de syntaxes, j'ai commis un certain nombre d'erreurs. Cette section est amenée à se développer.
 
 \begin{itemize}\setlength{\itemsep}{10pt}
  \item \tkzcname{tkzDrawPoint(A,B)} alors qu'il faut  \tkzcname{tkzDrawPoints}
  \item  \tkzcname{tkzGetPoint(A)} Quand on définit un objet, il faut utiliser des accolades et non des parenthèses, il faut donc écrire~: \tkzcname{tkzGetPoint\{A\}}
  
    \item \tkzcname{tkzGetPoint\{A\}} à la place de \tkzcname{tkzGetFirstPoint\{A\}}. Quant une macro donne deux points comme résultats, soit on récupère ces points  à l'aide de \tkzcname{tkzGetPoints\{A\}\{B\}}, soit on ne récupère que l'un des deux points, à l'aide  \tkzcname{tkzGetFirstPoint\{A\}} ou bien de \tkzcname{tkzGetSecondPoint\{A\}}. Ces deux points peuvent être utilisés avec comme référence \tkzname{tkzFirstPointResult} ou  \tkzname{tkzSecondPointResult}. Il est possible qu'un troisième point soit donné sous la référence \tkzname{tkzPointResult}  
     
  \item \tkzcname{tkzDrawSegment(A,B A,C)} alors qu'il faut  \tkzcname{tkzDrawSegments}. Il est possible de n'utiliser que les versions avec un « s » mais c'est moins efficace!
  \item Mélange option et arguments; toutes les macros  qui utilisent un cercle ont besoin de connaître le rayon de celui-ci. Si le rayon est donné par une mesure alors l'option comprend un \tkzname{R}.

\item  \tkzcname{tkzDrawSegments[color = gray,style=dashed]\{B,B' C,C'\}} est une erreur. Seules, les macros qui définissent un objet utilisent des accolades.   
  \item Les angles sont donnés en degrés 
  
  \item Si une erreur survient dans un calcul lors d'un passage de paramètres, alors il est préférable de faire ces calculs avant d'appeler la macro.
  \item Ne pas mélanger la syntaxe de \tkzNamePack{pgfmath} et celle de \tkzNamePack{fp.sty}. J'ai choisi souvent \tkzname{fp.sty} mais si vous préférez  pgfmath alors effectuez vos calculs avant le passage de paramètres.

%\tkzDrawLines[add=0 and 8]( A,a B,b) au lieu de \tkzDrawLines[add=0 and 8](A,a B,b)

%\tkzActivOff 
%\tkzDrawSegment[color=Maroon!50](I,H)

\item usage de \tkzcname{tkzClip} : Afin d'avoir des résultats  précis, j'ai évité de passer par des vecteurs normalisés. L'avantage de la normalisation est de contrôler la dimension des objets manipulés, le désavantage est qu'avec TeX, cela implique des erreurs. Ces erreurs sont souvent minimes, de l'ordre du millième, mais entraînent des catastrophes si le dessin est agrandi. Ne pas normaliser implique que certains points se trouvent bien loin de la zone de travail et seul \tkzcname{tkzClip} permet de réduire la taille du dessin. 


\item  une erreur se produit si vous utilisez la macro \tkzcname{tkzDrawAngle}
 avec un angle trop petit. L'erreur est produite par la librairie  \NameLib{decoration} quand on veut placer une marque sur un arc. Même si la marque est absente, l'erreur, elle, reste présente. Il est possible de contourner cette difficulté avec l'option \tkzname{mkpos=.2} par exemple, qui placera la marque avant l'arc. Une autre possibilité est d'utiliser la macro \tkzcname{tkzFillAngle}
\item Somme de deux vecteurs

Comment obtenir le point D tel que $\overrightarrow{AD} = \overrightarrow{AB} + \overrightarrow{AC}$?

\begin{tkzexample}[latex=5 cm,small]
  \begin{tikzpicture}[scale=.5]
  \tkzDefPoint(1,1){A}
  \tkzDefPoint(8,0){B}
  \tkzDefPoint(3,4){C} 
  \tkzDefVector[colinear= at C](A,B){D}
  \tkzDrawVectors[color=blue](A,B A,C) 
  \tkzDrawVector[color=red](A,D) 
  \tkzLabelPoints(A,B,C,D)  
\end{tikzpicture}
\end{tkzexample}

  \end{itemize}    
\endinput         
\clearpage\newpage
\printindex

\end{document}
