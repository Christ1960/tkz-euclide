\section{Special points relating to a triangle}

\subsection{Triangle center: \tkzcname{tkzDefTriangleCenter}}

This macro allows you to define the center of a triangle.


\begin{NewMacroBox}{tkzDefTriangleCenter}{\oarg{local options}\parg{A,B,C}}%
\tkzHandBomb\ Be careful, the arguments are lists of three points. This macro is used in conjunction with \tkzcname{tkzGetPoint} to get the center you are looking for. You can use \tkzname{tkzPointResult} if it is not necessary to keep the results.

\medskip
\begin{tabular}{lll}%
\toprule
arguments & default & definition \\

\midrule
\TAline{(pt1,pt2,pt3)}{no default}{three points}
\midrule
options             & default & definition                         \\
\midrule
\TOline{ortho}  {circum}{intersection of the altitudes of a triangle}
\TOline{centroid} {circum}{centre of gravity. Intersection of the medians }
\TOline{circum}{circum}{circle center circumscribed}
\TOline{in}    {circum}{center of the circle inscribed in a triangle }
\TOline{ex}    {circum}{center of a circle exinscribed to a triangle }
\TOline{euler}{circum}{center of Euler's circle }
\TOline{symmedian} {circum}{Lemoine's point or symmedian centre or Grebe's point }
\TOline{spieker} {circum}{Spieker Circle Center}
\TOline{nagel}{circum}{Nagel Center}
\TOline{mittenpunkt} {circum}{also called the middlespoint}
\TOline{feuerbach}{circum}{Feuerbach Point}

\end{tabular}
\end{NewMacroBox}

\subsubsection{Option \tkzname{ortho} or \tkzname{orthic}}
 The intersection $H$ of the three altitudes  of a triangle is called the orthocenter.

\begin{tkzexample}[latex=5cm,small]
\begin{tikzpicture}
  \tkzDefPoint(0,0){A}
  \tkzDefPoint(5,1){B}
  \tkzDefPoint(1,4){C}
  \tkzClipPolygon(A,B,C)
  \tkzDefTriangleCenter[ortho](B,C,A)
    \tkzGetPoint{H}
  \tkzDefSpcTriangle[orthic,name=H](A,B,C){a,b,c}
  \tkzDrawPolygon[color=blue](A,B,C)
  \tkzDrawPoints(A,B,C,H)
  \tkzDrawLines[add=0 and 1](A,Ha B,Hb C,Hc)
  \tkzLabelPoint(H){$H$}
  \tkzAutoLabelPoints[center=H](A,B,C)
  \tkzMarkRightAngles(A,Ha,B B,Hb,C C,Hc,A)
\end{tikzpicture}
\end{tkzexample}

\subsubsection{Option \tkzname{centroid}}
\begin{tkzexample}[latex=5cm,small]
\begin{tikzpicture}[scale=.75]
  \tkzDefPoints{-1/1/A,5/1/B}
  \tkzDefEquilateral(A,B)
  \tkzGetPoint{C}
  \tkzDefTriangleCenter[centroid](A,B,C)
      \tkzGetPoint{G}
  \tkzDrawPolygon[color=brown](A,B,C)
  \tkzDrawPoints(A,B,C,G)
  \tkzDrawLines[add = 0 and 2/3](A,G B,G C,G)
\end{tikzpicture}
\end{tkzexample}

\subsubsection{Option \tkzname{circum}}
\begin{tkzexample}[latex=6cm,small]
 \begin{tikzpicture}
  \tkzDefPoints{0/1/A,3/2/B,1/4/C}
  \tkzDefTriangleCenter[circum](A,B,C)
  \tkzGetPoint{G}
  \tkzDrawPolygon[color=brown](A,B,C)
  \tkzDrawCircle(G,A)
  \tkzDrawPoints(A,B,C,G)
 \end{tikzpicture}
\end{tkzexample}

\subsubsection{Option \tkzname{in}}
In geometry, the incircle or inscribed circle of a triangle is the largest circle contained in the triangle; it touches (is tangent to) the three sides. The center of the incircle is a triangle center called the triangle's incenter.
The center of the incircle, called the incenter, can be found as the intersection of the three internal angle bisectors. The center of an excircle is the intersection of the internal bisector of one angle (at vertex $A$, for example) and the external bisectors of the other two. The center of this excircle is called the excenter relative to the vertex $A$, or the excenter of $A$. Because the internal bisector of an angle is perpendicular to its external bisector, it follows that the center of the incircle together with the three excircle centers form an orthocentric system.(\url{https://en.wikipedia.org/wiki/Incircle_and_excircles_of_a_triangle})
 
 \medskip
 We get the centre of the inscribed circle of the triangle. The result is of course in \tkzname{tkzPointResult}. We can retrieve it with \tkzcname{tkzGetPoint}.

\begin{tkzexample}[latex=6cm,small]
\begin{tikzpicture}
  \tkzDefPoints{0/1/A,3/2/B,1/4/C}
  \tkzDefTriangleCenter[in](A,B,C)\tkzGetPoint{I}
  \tkzDefPointBy[projection=onto A--C](I)
  \tkzGetPoint{Ib}
  \tkzDrawPolygon[color=blue](A,B,C)
  \tkzDrawPoints(A,B,C,I)
  \tkzDrawLines[add = 0 and 2/3](A,I B,I C,I)
  \tkzDrawCircle(I,Ib)
\end{tikzpicture}
\end{tkzexample}

\subsubsection{Option \tkzname{ex}}
An excircle or escribed circle of the triangle is a circle lying outside the triangle, tangent to one of its sides and tangent to the extensions of the other two. Every triangle has three distinct excircles, each tangent to one of the triangle's sides.
(\url{https://en.wikipedia.org/wiki/Incircle_and_excircles_of_a_triangle})


 We get the centre of an inscribed circle of the triangle. The result is of course in \tkzname{tkzPointResult}. We can retrieve it with \tkzcname{tkzGetPoint}.

\begin{tkzexample}[latex=8cm,small]
  \begin{tikzpicture}[scale=.5]
    \tkzDefPoints{0/1/A,3/2/B,1/4/C}
    \tkzDefTriangleCenter[ex](B,C,A)
    \tkzGetPoint{J_c}
    \tkzDefPointBy[projection=onto A--B](J_c)
    \tkzGetPoint{Tc}
    %or
    % \tkzDefCircle[ex](B,C,A)
    % \tkzGetFirstPoint{J_c}
    % \tkzGetSecondPoint{Tc}
    \tkzDrawPolygon[color=blue](A,B,C)
    \tkzDrawPoints(A,B,C,J_c)
    \tkzDrawCircle[red](J_c,Tc)
    \tkzDrawLines[add=1.5 and 0](A,C B,C)
    \tkzLabelPoints(J_c)
  \end{tikzpicture}
\end{tkzexample}

\subsubsection{Option \tkzname{euler}}
This macro allows to obtain the center of the circle of the nine points or euler's circle or Feuerbach's circle.
The nine-point circle, also called Euler's circle or the Feuerbach circle, is the circle that passes through the perpendicular feet $H_A$, $H_B$, and $H_C$ dropped from the vertices of any reference triangle $ABC$ on the sides opposite them. Euler showed in 1765 that it also passes through the midpoints $M_A$, $M_B$, $M_C$ of the sides of $ABC$. By Feuerbach's theorem, the nine-point circle also passes through the midpoints $E_A$, $E_B$, and $E_C$ of the segments that join the vertices and the orthocenter $H$. These points are commonly referred to as the Euler points. (\url{http://mathworld.wolfram.com/Nine-PointCircle.html})

\begin{tkzexample}[latex=7cm,small]
\begin{tikzpicture}[scale=1]
 \tkzDefPoints{0/0/A,6/0/B,0.8/4/C}
 \tkzDefSpcTriangle[medial,
     name=M](A,B,C){_A,_B,_C}
 \tkzDefTriangleCenter[euler](A,B,C)
    \tkzGetPoint{N} % I= N nine points
 \tkzDefTriangleCenter[ortho](A,B,C)
    \tkzGetPoint{H}
 \tkzDefMidPoint(A,H) \tkzGetPoint{E_A}
 \tkzDefMidPoint(C,H) \tkzGetPoint{E_C}
 \tkzDefMidPoint(B,H) \tkzGetPoint{E_B}
 \tkzDefSpcTriangle[ortho,name=H](A,B,C){_A,_B,_C}
 \tkzDrawPolygon[color=blue](A,B,C)
 \tkzDrawCircle(N,E_A)
 \tkzDrawSegments[blue](A,H_A B,H_B C,H_C)
 \tkzDrawPoints(A,B,C,N,H)
 \tkzDrawPoints[red](M_A,M_B,M_C)
 \tkzDrawPoints[blue]( H_A,H_B,H_C)
 \tkzDrawPoints[green](E_A,E_B,E_C)
 \tkzAutoLabelPoints[center=N,
  font=\scriptsize](A,B,C,%
   M_A,M_B,M_C,%
   H_A,H_B,H_C,%
   E_A,E_B,E_C)
 \tkzLabelPoints[font=\scriptsize](H,N)
 \tkzMarkSegments[mark=s|,size=3pt,
     color=blue,line width=1pt](B,E_B E_B,H)
\end{tikzpicture}
\end{tkzexample}


\subsubsection{Option \tkzname{symmedian}}

\begin{tkzexample}[latex=6cm,small]
\begin{tikzpicture}
  \tkzDefPoint(0,0){A}
  \tkzDefPoint(5,0){B}
  \tkzDefPoint(1,4){C}
  \tkzDefTriangleCenter[symmedian](A,B,C)\tkzGetPoint{K}
  \tkzDefTriangleCenter[median](A,B,C)\tkzGetPoint{G}
  \tkzDefTriangleCenter[in](A,B,C)\tkzGetPoint{I}
  \tkzDefSpcTriangle[centroid,name=M](A,B,C){a,b,c}
  \tkzDefSpcTriangle[incentral,name=I](A,B,C){a,b,c}
  \tkzDrawPolygon[color=blue](A,B,C)
  \tkzDrawLines[add = 0 and 2/3,blue](A,K B,K C,K)
  \tkzDrawSegments[red,dashed](A,Ma B,Mb C,Mc)
  \tkzDrawSegments[orange,dashed](A,Ia B,Ib C,Ic)
  \tkzDrawLine[add=2 and 2](G,I)
  \tkzDrawPoints(A,B,C,K,G,I)
\end{tikzpicture}
\end{tkzexample}


\subsubsection{Option \tkzname{nagel}}
Let $Ta$ be the point at which the excircle with center $Ja$ meets the side $BC$ of a triangle $ABC$, and define $Tb$ and $Tc$ similarly. Then the lines $ATa$, $BTb$, and $CTc$ concur in the Nagel point $Na$.
\href{http://mathworld.wolfram.com/NagelPoint.html}{Weisstein, Eric W. "Nagel point." From MathWorld--A Wolfram Web Resource. }


\begin{tkzexample}[latex=8cm,small]
  \begin{tikzpicture}[scale=.5]
  \tkzDefPoints{0/0/A,6/0/B,4/6/C}
  \tkzDefSpcTriangle[ex](A,B,C){Ja,Jb,Jc}
  \tkzDefSpcTriangle[extouch](A,B,C){Ta,Tb,Tc}
  \tkzDrawPoints(Ja,Jb,Jc,Ta,Tb,Tc)
  \tkzLabelPoints(Ja,Jb,Jc,Ta,Tb,Tc)
  \tkzDrawPolygon[blue](A,B,C)
  \tkzDefTriangleCenter[nagel](A,B,C) \tkzGetPoint{Na}
  \tkzDrawPoints[blue](B,C,A)
  \tkzDrawPoints[red](Na)
  \tkzLabelPoints[blue](B,C,A)
  \tkzLabelPoints[red](Na)
  \tkzDrawLines[add=0 and 1](A,Ta B,Tb C,Tc)
  \tkzShowBB\tkzClipBB
  \tkzDrawLines[add=1 and 1,dashed](A,B B,C C,A)
  \tkzDrawCircles[ex,gray](A,B,C C,A,B B,C,A)
  \tkzDrawSegments[dashed](Ja,Ta Jb,Tb Jc,Tc)
  \tkzMarkRightAngles[fill=gray!20](Ja,Ta,C
              Jb,Tb,A Jc,Tc,B)
  \end{tikzpicture}
\end{tkzexample}


\subsubsection{Option   \tkzname{mittenpunkt}} 
\begin{tkzexample}[latex=8cm,small]
\begin{tikzpicture}[scale=.5]
 \tkzDefPoints{0/0/A,6/0/B,4/6/C}
 \tkzDefSpcTriangle[centroid](A,B,C){Ma,Mb,Mc}
 \tkzDefSpcTriangle[ex](A,B,C){Ja,Jb,Jc}
 \tkzDefSpcTriangle[extouch](A,B,C){Ta,Tb,Tc}
 \tkzDefTriangleCenter[mittenpunkt](A,B,C) 
 \tkzGetPoint{Mi}
 \tkzDrawPoints(Ma,Mb,Mc,Ja,Jb,Jc)
 \tkzClipBB
 \tkzDrawPolygon[blue](A,B,C)
 \tkzDrawLines[add=0 and 1](Ja,Ma 
               Jb,Mb Jc,Mc)
 \tkzDrawLines[add=1 and 1](A,B A,C B,C)
 \tkzDrawCircles[gray](Ja,Ta Jb,Tb Jc,Tc)
 \tkzDrawPoints[blue](B,C,A)
 \tkzDrawPoints[red](Mi)
 \tkzLabelPoints[red](Mi)
 \tkzLabelPoints[left](Mb)
 \tkzLabelPoints(Ma,Mc,Jb,Jc)
 \tkzLabelPoints[above left](Ja,Jc)
 \tkzShowBB
\end{tikzpicture}
\end{tkzexample}
%<---------------------------------------------------------------------->
%<---------------------------------------------------------------------->
\section{Draw a point}
\subsubsection{Drawing points \tkzcname{tkzDrawPoint}} \hypertarget{tdrp}{}

\begin{NewMacroBox}{tkzDrawPoint}{\oarg{local options}\parg{name}}%
\begin{tabular}{lll}%
arguments &  default & definition                 \\
\midrule
\TAline{name of point} {no default}  {Only one point name is accepted}
\bottomrule
\end{tabular}

\medskip
The argument is required. The disc takes the color of the circle, but  lighter. It is possible to change everything. The point is a node and therefore it is invariant if the drawing is modified by scaling.

\medskip
\begin{tabular}{lll}%
\toprule
options             & default & definition \\
\midrule
\TOline{shape}  {circle}{Possible \tkzname{cross} or \tkzname{cross out}}
\TOline{size}  {6}{$6 \times$ \tkzcname{pgflinewidth}}
\TOline{color}  {black}{the default color can be changed }
\bottomrule
\end{tabular}

\medskip
{We can create other forms such as \tkzname{cross}}
\end{NewMacroBox}

\subsubsection{Example of point drawings}
Note that \tkzname{scale} does not affect the shape of the dots. Which is normal.  Most of the time, we are satisfied with a single point shape that we can define from the beginning, either with a macro or by modifying a configuration file.


\begin{tkzexample}[latex=5cm,small]
  \begin{tikzpicture}[scale=.5]
   \tkzDefPoint(1,3){A}
   \tkzDefPoint(4,1){B}
   \tkzDefPoint(0,0){O}
   \tkzDrawPoint[color=red](A)
   \tkzDrawPoint[fill=blue!20,draw=blue](B)
   \tkzDrawPoint[color=green](O)
  \end{tikzpicture}
\end{tkzexample}

It is possible to draw several points at once but this macro is a little slower than the previous one. Moreover, we have to make do with the same options for all the points.

\hypertarget{tdrps}{}
\begin{NewMacroBox}{tkzDrawPoints}{\oarg{local options}\parg{liste}}%
\begin{tabular}{lll}%
arguments &  default  & definition \\
\midrule
\TAline{points list}{no default}{example \tkzcname{tkzDrawPoints(A,B,C)}}
\bottomrule
\end{tabular}

\medskip
\begin{tabular}{lll}%
options             & default & definition \\
\midrule
\TOline{shape}  {circle}{Possible \tkzname{cross} or \tkzname{cross out}}
\TOline{size}  {6}{$6 \times$ \tkzcname{pgflinewidth}}
\TOline{color}  {black}{the default color can be changed }
\bottomrule
\end{tabular}

\medskip
\tkzHandBomb\ Beware of the final "s", an oversight leads to cascading errors if you try to draw multiple points. The options are the same as for the previous macro.
\end{NewMacroBox}

\subsubsection{First example}

\begin{tkzexample}[latex=7cm,small]
\begin{tikzpicture}
  \tkzDefPoint(1,3){A} 
  \tkzDefPoint(4,1){B} 
  \tkzDefPoint(0,0){C} 
  \tkzDrawPoints[size=6,color=red,
               fill=red!50](A,B,C)
\end{tikzpicture}
\end{tkzexample}

\subsubsection{Second example}

\begin{tkzexample}[latex=7cm,small]
\begin{tikzpicture}[scale=.5]
 \tkzDefPoint(2,3){A}  \tkzDefPoint(5,-1){B}
 \tkzDefPoint[label=below:$\mathcal{C}$,
               shift={(2,3)}](-30:5.5){E}
 \begin{scope}[shift=(A)]
    \tkzDefPoint(30:5){C}
 \end{scope}
 \tkzCalcLength[cm](A,B)\tkzGetLength{rAB}
 \tkzDrawCircle[R](A,\rAB cm)
 \tkzDrawSegment(A,B)
 \tkzDrawPoints(A,B,C)
 \tkzLabelPoints(B,C)
 \tkzLabelPoints[above](A)
\end{tikzpicture}
\end{tkzexample}

\section{Point on line or circle}
\subsection{Point on a line}

\begin{NewMacroBox}{tkzDefPointOnLine}{\oarg{local options}\parg{A,B}}%
\begin{tabular}{lll}%
arguments &  default & definition                 \\
\midrule
\TAline{pt1,pt2} {no default}  {Two points to define a line}
\bottomrule
\end{tabular}

\medskip
\begin{tabular}{lll}%
options       & default & definition \\
\midrule
\TOline{pos=nb}  {}{nb is a decimal  }
\end{tabular}
\end{NewMacroBox}

\subsubsection{Use of option \tkzname{pos}}
\begin{tkzexample}[latex=9cm,small]
  \begin{tikzpicture}
  \tkzDefPoints{0/0/A,4/0/B}
  \tkzDrawLine[red](A,B)
  \tkzDefPointOnLine[pos=1.2](A,B) 
  \tkzGetPoint{P}
  \tkzDefPointOnLine[pos=-0.2](A,B) 
  \tkzGetPoint{R}
  \tkzDefPointOnLine[pos=0.5](A,B) 
  \tkzGetPoint{S}
  \tkzDrawPoints(A,B,P)
  \tkzLabelPoints(A,B)
  \tkzLabelPoint[above](P){pos=$1.2$}
  \tkzLabelPoint[above](R){pos=$-.2$}
  \tkzLabelPoint[above](S){pos=$.5$}
  \tkzDrawPoints(A,B,P,R,S)
  \tkzLabelPoints(A,B)
  \end{tikzpicture}
\end{tkzexample}

\subsection{Point on a circle}

\begin{NewMacroBox}{tkzDefPointOnCircle}{\oarg{local options}}%
\begin{tabular}{lll}%
options   & default & definition \\
\midrule
\TOline{angle}  {0}{angle formed with the abscissa axis}
\TOline{center}  {|tkzPointResult|}{circle center required}
\TOline{radius}  {|\BS tkzLengthResult|}{radius circle}
\end{tabular}
\end{NewMacroBox}

\begin{tkzexample}[latex=7cm,small]
\begin{tikzpicture}
 \tkzDefPoints{0/0/A,4/0/B,0.8/3/C}  
 \tkzDefPointOnCircle[angle=90,center=B,radius=1 cm]
 \tkzGetPoint{I}    
 \tkzDefCircle[circum](A,B,C)
 \tkzGetPoint{G} \tkzGetLength{rG}
 \tkzDefPointOnCircle[angle=30,center=G,radius=\rG pt]
 \tkzGetPoint{J}
 \tkzDrawCircle[R,teal](B,1cm) 
 \tkzDrawPoint[teal](I)
 \tkzDrawPoints(A,B,C)
 \tkzDrawCircle(G,J)
 \tkzDrawPoints(G,J)
 \tkzDrawPoint[red](J)
 \tkzLabelPoints(G,J)
\end{tikzpicture}
\end{tkzexample}
\endinput