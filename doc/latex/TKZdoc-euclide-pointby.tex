\section{Definition of points by transformation; \tkzcname{tkzDefPointBy} }
These transformations are:

\begin{itemize}
   \item translation;
   \item homothety;
   \item orthogonal reflection or symmetry;
   \item central symmetry;
   \item orthogonal projection;
   \item rotation (degrees or radians);
   \item inversion with respect to a circle.
\end{itemize}

The choice of transformations is made through the options. There are two macros, one for the transformation of a single point \tkzcname{tkzDefPointBy} and the other for the transformation of a list of points \tkzcname{tkzDefPointsBy}. By default the image of $A$ is $A'$. For example, we'll write:
\begin{tkzltxexample}[]
\tkzDefPointBy[translation= from A to A'](B)
\end{tkzltxexample}
The result is in \tkzname{tkzPointResult}
\medskip
\begin{NewMacroBox}{tkzDefPointBy}{\oarg{local options}\parg{pt}}%
The argument is a simple existing point and its image is stored in \tkzname{tkzPointResult}. If you want to keep this point then the macro \tkzcname{tkzGetPoint\{M\}} allows you to assign the name \tkzname{M} to the point.

\begin{tabular}{lll}%
\toprule
arguments &  definition & examples               \\
\midrule
\TAline{pt}   {existing point name}   {$(A)$}
\bottomrule
\end{tabular}

\begin{tabular}{lll}%
options     &     & examples                         \\
\midrule
\TOline{translation}{= from \#1 to \#2}{[translation=from A to B](E)}
\TOline{homothety}  {= center \#1 ratio \#2}{[homothety=center A ratio .5](E)}
\TOline{reflection} {= over \#1--\#2}{[reflection=over A--B](E)}
\TOline{symmetry }  {= center \#1}{[symmetry=center A](E)}
\TOline{projection }{= onto \#1--\#2}{[projection=onto A--B](E)}
\TOline{rotation }  {= center \#1 angle \#2}{[rotation=center O angle 30](E)}
\TOline{rotation in rad}{= center \#1 angle \#2}{[rotation in rad=center O angle pi/3](E)}
\TOline{inversion}{= center \#1 through \#2}{[inversion =center O through A](E)}
\bottomrule
\end{tabular}

The image is only defined and not drawn.
\end{NewMacroBox}

\subsection{Examples of transformations}
\subsubsection{Example of translation}

\subsection{Example of translation}
\begin{tkzexample}[latex=7cm,small]
\begin{tikzpicture}[>=latex]
 \tkzDefPoint(0,0){A}  \tkzDefPoint(3,1){B}
 \tkzDefPoint(3,0){C}
 \tkzDefPointBy[translation= from B to A](C)
 \tkzGetPoint{D}
 \tkzDrawPoints[teal](A,B,C,D)
 \tkzLabelPoints[color=teal](A,B,C,D)
 \tkzDrawSegments[orange,->](A,B D,C)
\end{tikzpicture}
\end{tkzexample}

\subsubsection{Example of reflection (orthogonal symmetry)}

\begin{tkzexample}[vbox,small]
\begin{tikzpicture}[scale=1]
 \tkzDefPoints{1.5/-1.5/C,-4.5/2/D}
 \tkzDefPoint(-4,-2){O}
 \tkzDefPoint(-2,-2){A}
 \foreach \i in {0,1,...,4}{%
 \pgfmathparse{0+\i * 72}
 \tkzDefPointBy[rotation=%
 center O angle \pgfmathresult](A)
  \tkzGetPoint{A\i}
 \tkzDefPointBy[reflection = over C--D](A\i)
  \tkzGetPoint{A\i'}}
 \tkzDrawPolygon(A0, A2, A4, A1, A3)
 \tkzDrawPolygon(A0', A2', A4', A1', A3')
 \tkzDrawLine[add= .5 and .5](C,D)
\end{tikzpicture}
\end{tkzexample}


\subsubsection{Example of \tkzname{homothety} and \tkzname{projection}}

\begin{tkzexample}[vbox,small]
\begin{tikzpicture}[scale=1.2]
  \tkzDefPoint(0,1){A}   \tkzDefPoint(5,3){B}   \tkzDefPoint(3,4){C}
  \tkzDefLine[bisector](B,A,C)                     \tkzGetPoint{a}
  \tkzDrawLine[add=0 and 0,color=magenta!50 ](A,a)
  \tkzDefPointBy[homothety=center A ratio .5](a)   \tkzGetPoint{a'}
  \tkzDefPointBy[projection = onto A--B](a')       \tkzGetPoint{k'}
  \tkzDefPointBy[projection = onto A--B](a)       \tkzGetPoint{k}
  \tkzDrawLines[add= 0 and .3](A,k A,C)
  \tkzDrawSegments[blue](a',k' a,k)
  \tkzDrawPoints(a,a',k,k',A)
  \tkzDrawCircles(a',k' a,k)
  \tkzLabelPoints(a,a',k,A)
\end{tikzpicture}
\end{tkzexample}


\subsubsection{Example of projection}
\begin{tkzexample}[vbox,small]
\begin{tikzpicture}[scale=1.5]
 \tkzDefPoint(0,0){A}
 \tkzDefPoint(0,4){B}
 \tkzDefTriangle[pythagore](B,A) \tkzGetPoint{C}
 \tkzDefLine[bisector](B,C,A) \tkzGetPoint{c}
 \tkzInterLL(C,c)(A,B)        \tkzGetPoint{D}
 \tkzDefPointBy[projection=onto B--C](D) \tkzGetPoint{G}
 \tkzInterLC(C,D)(D,A) \tkzGetPoints{E}{F}
 \tkzDrawPolygon[teal](A,B,C)
 \tkzDrawSegment(C,D)
 \tkzDrawCircle(D,A)
 \tkzDrawSegment[orange](D,G)
 \tkzMarkRightAngle[fill=orange!20](D,G,B)
 \tkzDrawPoints(A,C,F) \tkzLabelPoints(A,C,F)
 \tkzDrawPoints(B,D,E,G)
 \tkzLabelPoints[above right](B,D,E,G)
 \end{tikzpicture}
 \end{tkzexample}

\subsubsection{Example of symmetry}
\begin{tkzexample}[vbox,small]
\begin{tikzpicture}[scale=1]
  \tkzDefPoint(0,0){O}
  \tkzDefPoint(2,-1){A}
  \tkzDefPoint(2,2){B}
  \tkzDefPointsBy[symmetry=center O](B,A){}
  \tkzDrawLine(A,A')
  \tkzDrawLine(B,B')
  \tkzMarkAngle[mark=s,arc=lll,
      size=2 cm,mkcolor=red](A,O,B)
  \tkzLabelAngle[pos=1,circle,draw,
     fill=blue!10](A,O,B){$60^{\circ}$}
  \tkzDrawPoints(A,B,O,A',B')
  \tkzLabelPoints(A,B,O,A',B')
\end{tikzpicture}
\end{tkzexample}

\subsubsection{Example of rotation}
\begin{tkzexample}[latex=7cm,small]
\begin{tikzpicture}[scale=0.5]
 \tkzDefPoint(0,0){A}
 \tkzDefPoint(5,0){B}
 \tkzDrawSegment(A,B)
 \tkzDefPointBy[rotation=center A angle 60](B)
 \tkzGetPoint{C}
 \tkzDefPointBy[symmetry=center C](A)
 \tkzGetPoint{D}
 \tkzDrawSegment(A,tkzPointResult)
 \tkzDrawLine(B,D)
 \tkzDrawArc[orange,delta=10](A,B)(C)
 \tkzDrawArc[orange,delta=10](B,C)(A)
 \tkzDrawArc[orange,delta=10](C,D)(D)
 \tkzMarkRightAngle(D,B,A)
\end{tikzpicture}
\end{tkzexample}

\subsubsection{Example of rotation in radian}
\begin{tkzexample}[latex=6cm,small]
\begin{tikzpicture}
  \tkzDefPoint["$A$" left](1,5){A}
  \tkzDefPoint["$B$" right](5,2){B}
  \tkzDefPointBy[rotation in rad= center A angle pi/3](B)
  \tkzGetPoint{C}
  \tkzDrawSegment(A,B)
  \tkzDrawPoints(A,B,C)
  \tkzCompass[color=red](A,C)
  \tkzCompass[color=red](B,C)
  \tkzLabelPoints(C)
\end{tikzpicture}
\end{tkzexample}

\subsubsection{Inversion of points}
\begin{tkzexample}[latex=8cm,small]
\begin{tikzpicture}[scale=1.5]
  \tkzDefPoint(0,0){O}
  \tkzDefPoint(1,0){A}
  \tkzDefPoint(-1.5,-1.5){z1}
  \tkzDefPoint(0.35,0){z2}
  \tkzDefPointBy[inversion =%
      center O through A](z1)
  \tkzGetPoint{Z1}
  \tkzDefPointBy[inversion =%
      center O through A](z2)
  \tkzGetPoint{Z2}
  \tkzDrawCircle(O,A)
  \tkzDrawPoints[color=black,
      fill=red,size=4](Z1,Z2)
  \tkzDrawSegments(z1,Z1 z2,Z2)
  \tkzDrawPoints[color=black,
      fill=red,size=4](O,z1,z2)
  \tkzLabelPoints(O,A,z1,z2,Z1,Z2)
\end{tikzpicture}
\end{tkzexample}


\subsubsection{Point Inversion: Orthogonal Circles}
\begin{tkzexample}[latex=8cm,small]
\begin{tikzpicture}[scale=1.5]
  \tkzDefPoint(0,0){O}
  \tkzDefPoint(1,0){A}
  \tkzDrawCircle(O,A)
  \tkzDefPoint(0.5,-0.25){z1}
  \tkzDefPoint(-0.5,-0.5){z2}
  \tkzDefPointBy[inversion = %
     center O through A](z1)
  \tkzGetPoint{Z1}
  \tkzCircumCenter(z1,z2,Z1)
  \tkzGetPoint{c}
  \tkzDrawCircle(c,Z1)
  \tkzDrawPoints[color=black,
     fill=red,size=4](O,z1,z2,Z1,O,A)
\end{tikzpicture}
\end{tkzexample}

\subsection{Transformation of multiple points; \tkzcname{tkzDefPointsBy} }
Variant of the previous macro for defining multiple images.
You must give the names of the images as arguments, or indicate that the names of the images are formed from the names of the antecedents, leaving the argument empty.

\begin{tkzltxexample}[]
\tkzDefPointsBy[translation= from A to A'](B,C){}
\end{tkzltxexample}
The images are $B'$ and $C'$.

\begin{tkzltxexample}[]
\tkzDefPointsBy[translation= from A to A'](B,C){D,E}
\end{tkzltxexample}
The images are $D$ and $E$.

\begin{tkzltxexample}[]
\tkzDefPointsBy[translation= from A to A'](B)
\end{tkzltxexample}
The image is $B'$.
\begin{NewMacroBox}{tkzDefPointsBy}{\oarg{local options}\parg{list of points}\marg{list of points}}%
\begin{tabular}{lll}%
arguments &  examples  &                  \\
\midrule
\TAline{\parg{list of points}\marg{list of pts}}{(A,B)\{E,F\}}{$E$ is the image of $A$ and $F$ is the image of $B$.}   \\
\bottomrule
\end{tabular}

\medskip
If the list of images is empty then the name of the image is the name of the antecedent to which " ' " is added.

\medskip
\begin{tabular}{lll}%
\toprule
options     &     & examples                         \\
\midrule
\TOline{translation = from \#1 to \#2}{}{[translation=from A to B](E)\{\}}
\TOline{homothety = center \#1 ratio \#2}{}{[homothety=center A ratio .5](E)\{F\}}
\TOline{reflection = over \#1--\#2}{}{[reflection=over A--B](E)\{F\}}
\TOline{symmetry = center \#1}{}{[symmetry=center A](E)\{F\}}
\TOline{projection = onto \#1--\#2}{}{[projection=onto A--B](E)\{F\}}
\TOline{rotation = center \#1 angle \#2}{}{[rotation=center  angle 30](E)\{F\}}
\TOline{rotation in rad = center \#1 angle \#2}{}{for instance angle pi/3}
\bottomrule
\end{tabular}

\medskip
The points are only defined and not drawn.
\end{NewMacroBox}

\subsubsection{Example of translation}
\begin{tkzexample}[latex=7cm,small]
\begin{tikzpicture}[>=latex]
 \tkzDefPoint(0,0){A}  \tkzDefPoint(3,1){A'}
 \tkzDefPoint(3,0){B}  \tkzDefPoint(1,2){C}
 \tkzDefPointsBy[translation= from A to A'](B,C){}
 \tkzDrawPolygon[color=blue](A,B,C)
 \tkzDrawPolygon[color=red](A',B',C')
 \tkzDrawPoints[color=blue](A,B,C)
 \tkzDrawPoints[color=red](A',B',C')
 \tkzLabelPoints(A,B,A',B')
 \tkzLabelPoints[above](C,C')
 \tkzDrawSegments[color = gray,->,
              style=dashed](A,A' B,B' C,C')
\end{tikzpicture}
\end{tkzexample}

\endinput
